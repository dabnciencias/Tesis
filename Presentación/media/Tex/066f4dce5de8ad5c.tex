\documentclass[preview]{standalone}

\usepackage[english]{babel}
\usepackage[utf8]{inputenc}
\usepackage[T1]{fontenc}
\usepackage{lmodern}
\usepackage{amsmath}
\usepackage{amssymb}
\usepackage{dsfont}
\usepackage{setspace}
\usepackage{tipa}
\usepackage{relsize}
\usepackage{textcomp}
\usepackage{mathrsfs}
\usepackage{calligra}
\usepackage{wasysym}
\usepackage{ragged2e}
\usepackage{physics}
\usepackage{xcolor}
\usepackage{microtype}
\DisableLigatures{encoding = *, family = * }
\linespread{1}

\begin{document}

\begin{center}
\justifying \textbf{Corolario} Sea $\mathscr{I}\subseteq\mathscr{A}$ una subcategoría aditiva plena de $\mathscr{A}$ tal que $$\mathbb{E}(\mathscr{I}, \mathscr{I})=0.$$ Si $\mathscr{Z}\subseteq\mathscr{A}$ es la subcategoría de los objetos $Z\in\mathscr{A}$ tales que $$\mathbb{E}(Z,I) = 0 = \mathbb{E}(I,Z) \quad \forall \ I\in\mathscr{I},$$ entonces $\mathscr{Z}/\mathscr{I}$ tiene una estructura de categoría extriangulada.
\end{center}

\end{document}
