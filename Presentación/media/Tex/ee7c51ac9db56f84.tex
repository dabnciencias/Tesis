\documentclass[preview]{standalone}

\usepackage[english]{babel}
\usepackage[utf8]{inputenc}
\usepackage[T1]{fontenc}
\usepackage{lmodern}
\usepackage{amsmath}
\usepackage{amssymb}
\usepackage{dsfont}
\usepackage{setspace}
\usepackage{tipa}
\usepackage{relsize}
\usepackage{textcomp}
\usepackage{mathrsfs}
\usepackage{calligra}
\usepackage{wasysym}
\usepackage{ragged2e}
\usepackage{physics}
\usepackage{xcolor}
\usepackage{microtype}
\DisableLigatures{encoding = *, family = * }
\linespread{1}

\begin{document}

\begin{center}
\begin{itemize}
                        \item[$\bullet$] Se conoce una relación entre las categorías exactas y las trianguladas, dada por las categorías de Frobenius y sus categorías estables asociadas[2].
                        \item[$\bullet$] Muchos resultados de naturaleza homológica son válidos tanto en categorías exactas como en categorías trianguladas.
                        \item[$\bullet$] Los procesos de ``adaptación de pruebas'' para transferir resultados entre un tipo de categoría y otra suelen ser difíciles.
                        \end{itemize}
                        El uso de categorías extrianguladas remueve dificultades encontradas durante este proceso. Su definición se obtuvo axiomatizando las propiedades de los bifuntores $\text{Ext}^1$ relevantes para los pares de cotorsión (completos).
\end{center}

\end{document}
