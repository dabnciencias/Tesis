\documentclass[preview]{standalone}

\usepackage[english]{babel}
\usepackage[utf8]{inputenc}
\usepackage[T1]{fontenc}
\usepackage{lmodern}
\usepackage{amsmath}
\usepackage{amssymb}
\usepackage{dsfont}
\usepackage{setspace}
\usepackage{tipa}
\usepackage{relsize}
\usepackage{textcomp}
\usepackage{mathrsfs}
\usepackage{calligra}
\usepackage{wasysym}
\usepackage{ragged2e}
\usepackage{physics}
\usepackage{xcolor}
\usepackage{microtype}
\DisableLigatures{encoding = *, family = * }
\linespread{1}

\begin{document}

\begin{center}
\textbf{Definición} Sean $A,C\in\text{Obj}(\mathscr{A})$. Dos sucesiones de morfismos $A\xrightarrow{x}B\xrightarrow{y}C$ y $A\xrightarrow{x'}B'\xrightarrow{y'}C$ en $\mathscr{A}$ son \emph{equivalentes} si existe un isomorfismo $b\in\text{Hom}_\mathscr{A}(B,B')$ tal que el siguiente diagrama conmuta \break \vspace{5mm}Esto es una relación de equivalencia; denotamos las clases con corchetes ($[ \ ]$).
\end{center}

\end{document}
