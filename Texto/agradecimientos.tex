\documentclass[tesis]{subfiles}
\begin{document}

\chapter*{Agradecimientos}\label{Chap: Agradecimientos}
%\addcontentsline{toc}{chapter}{Agradecimientos}

Quiero agradecer infinitamente a mi familia por tanto apoyo y cariño durante toda mi vida. A pesar de que seamos personas tan divergentes, y sin importar qué tan cerca o qué tan lejos nos encontremos, sepan que siempre están conmigo. \\ %Víctor: Ya van casi tres años y te extraño demasiado, ¡ya regresa de China antes de que empiece \emph{otra} pandemia! \\

Quiero agradecer enormemente al Dr. Octavio Mendoza Hernández, de quien he aprendido mucho sobre matemáticas, tanto a través de sus excelentes clases como de sus increíbles notas, cuya calidad procuro emular en mi propio trabajo. No menos importantes han sido sus enseñanzas sobre cómo lidiar con las matemáticas y su inmensidad; sus consejos al respecto han sido tranquilizadores y esperanzadores, por decir lo menos, y los aprecio muchísimo. Agradezco también a Shaira, con quien colaboré durante los cursos avanzados del Dr. Octavio y de quien también he aprendido mucho; algunos frutos de dicha colaboración se encuentran en esta tesis. \\

Me siento también muy agradecido con el proyecto de animación \href{https://www.youtube.com/playlist?list=PL91agCMqt_mdAgHZkxyn-tscoNpu7ZHvl}{Animathica}, particularmente con Bruno, Pablo, Darío, Vianey, Raúl y Andrea, así como con mis estudiantes de la Facultad de Ciencias y con Javier, por ayudarme a mantener vivo y fuerte mi amor por las matemáticas, aún cuando el mundo parecía estar paralizado. \\

Por último, quisiera agradecer especialmente a Adrián Zenteno Gutiérrez, mi profesor de Cálculo durante mi primer semestre en la Facultad, y gran parte de la razón por la que decidí cambiarme a estudiar matemáticas después de terminar la licenciatura. Adrían: Tu claridad frente al pizarrón y tu humildad como persona son cosas que admiro y a las que aspiro; tuve el honor de llamarte profesor y ahora tengo el privilegio aún más grande de llamarte \emph{amigo}.

\end{document}
