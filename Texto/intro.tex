\documentclass[tesis]{subfiles}
\begin{document}

\chapter*{Introducción}\label{Chap: Introducción} % ¡Reemplazar números con referencias!
\addcontentsline{toc}{chapter}{Introducción}
\markboth{Introducción}{}

Actualmente, las estructuras básicas para el estudio del álgebra homológica son las categorías abelianas, las categorías exactas y las categorías trianguladas, cuya base común son las categorías aditivas. De las tres primeras, las que han sido más extensamente estudiadas en la literatura son las categorías abelianas\cite{Freyd}\cite{Mitchell}\cite{Popescu}, cuya investigación produjo a mediados de los años sesenta el Teorema de Inmersión de Freyd-Mitchell, que afirma que toda categoría abeliana pequeña puede ser sumergida en una subcategoría plena de una categoría de módulos sobre un anillo\footnote{En el presente trabajo, usaremos la palabra \emph{anillo} para referirnos a un anillo asociativo con unidad a menos que se especifique lo contrario.} a través de un funtor fiel, pleno y exacto. Como consecuencia, muchos resultados de álgebra homológica válidos para subcategorías plenas de la categoría $\text{Mod}(R)$ de módulos sobre un anillo pueden ser demostrados para categorías abelianas mediante una inmersión. En particular, esto incluye a los resultados de la forma ``$p$ implica $q$'', donde $p$ es una proposición categórica sobre un diagrama finito y $q$ afirma que existen un número finito de morfismos adicionales entre objetos del diagrama que hacen verificar alguna proposición categórica para el diagrama extendido\cite[{}VI.7.3]{Mitchell}. \\

Existen varias nociones de categoría exacta. La primera surgió en 1962 cuando Puppe\cite{Puppe} determinó cuáles propiedades de la categoría $\text{Mod}(R)$ eran necesarias para poder hablar sobre sucesiones exactas y dio una definición intrínseca de un tipo de categorías no aditivas que cumplían tales propiedades. Posteriormente, Mitchell\cite{Mitchell} estudió las propiedades homológicas de este tipo de categorías y las llamó categorías exactas. Dado que una categoría es abeliana si, y sólo si, es  exacta en el sentido de Puppe y aditiva, las categorías exactas definidas por Puppe dotan a las categorías aditivas de la estructura necesaria para que puedan formar categorías abelianas; sin embargo, tienen algunas deficiencias. En 1972, Quillen\cite{Quillen} dio una definición diferente de categoría exacta, la cual es extrínseca en el sentido de que, dada una categoría aditiva, es necesario especificar una clase de ``sucesiones exactas cortas distinguidas'' en ella, conocida como estructura exacta, para poder formar una categoría exacta. La definición dada por Quillen tiene varias virtudes; en particular, es auto dual, por lo que podemos argumentar usando dualidad, además de que muchos resultados básicos del álgebra homológica tales como el Lema del Cinco y el Tercer Teorema de Isomorfismo de Noether se siguen directamente de sus axiomas. En el presente trabajo, nos enfocaremos en esta segunda noción de categoría exacta por lo que, de ahora en adelante, diremos simplemente \emph{categoría exacta} para referirnos a una categoría exacta en el sentido de Quillen. \\ % ¿Mencionar Teorema de inmersión de Gabriel-Quillen?

Las propiedades homológicas de una categoría abeliana $\mathscr{A}$ también pueden ser estudiadas a través de una categoría derivada $D(\mathscr{A})$ donde, si la categoría abeliana tiene suficientes proyectivos o suficientes inyectivos, se puede aproximar a un objeto de $\mathscr{A}$ mediante construcciones que involucran a sus resoluciones proyectivas e inyectivas de una forma reminiscente a cómo una función diferenciable se puede aproximar en un punto mediante sus derivadas en algún punto cercano. La categoría derivada $D(\mathscr{A})$ y la categoría homotópica de complejos $K(\mathscr{A})$, necesaria para construir la primera, resultan ser aditivas, mas no necesariamente abelianas. Sin embargo, contienen una estructura útil dada por una clase de diagramas, conocidos como triángulos distinguidos, que fungen un rol análogo al de las sucesiones exactas cortas en las categorías abelianas\cite{Flores-Galicia}. En 1963, Verdier\cite{Verdier} axiomatizó esta estructura en su tesis doctoral, bajo la dirección de Grothendieck, desarrollando así la noción de categoría triangulada. \\

Muchos resultados de naturaleza homológica son válidos tanto en categorías exactas como en categorías trianguladas. Sin embargo, el proceso para transferir algunos de estos resultados de un tipo de categoría a otra no es trivial, y la adaptación de algunas pruebas suele ser difícil. En 2019, Nakaoka y Palu\cite{NakaokaPalu} introdujeron una generalización simultánea de las categorías exactas y las categorías trianguladas, y mostraron que con ella se pueden resolver las principales dificultades encontradas durante el proceso de transferencia de resultados entre estas categorías, principalmente para aquellos relacionados con pares de cotorsión. Obtuvieron esta generalización axiomatizando aquellas propiedades de los bifuntores aditivos $\text{Ext}_{}^{1}$ en categorías exactas y categorías trianguladas que son relevantes para el estudio de pares de cotorsión, y la llamaron categoría extriangulada. \\

El presente trabajo se divide en seis capítulos, incluyendo un apéndice. En el Capítulo \ref{Chap: Algunas nociones generales de la teoría de categorías} se da una introducción a la teoría de categorías, limitada a presentar aquellas nociones generales que serán utilizadas en el contexto del álgebra homológica durante el desarrollo principal del texto. La estructuración de este capítulo está hecha de tal forma que refleje el orden en el cual dichas nociones generales de teoría de categorías son requeridas durante los capítulos posteriores. Por ende, este capítulo sirve como referencia para los siguientes. Quienes tengan experiencia previa con el lenguaje categórico podrán empezar a leer a partir del Capítulo \ref{Chap: Categorías aditivas} si así lo desean; en caso contrario, se recomienda leer las secciones \ref{Sec: Definición de categoría} a \ref{Sec: Límites y colímites} antes de iniciar la lectura del Capítulo \ref{Chap: Categorías aditivas}. Posteriormente, para ambos casos, se sugiere consultar el Capítulo \ref{Chap: Algunas nociones generales de la teoría de categorías} siempre que se considere necesario durante el resto de la lectura. \\

El Capítulo \ref{Chap: Categorías aditivas} se enfoca en presentar las categorías aditivas, dado que son el punto de partida común para las categorías trianguladas, exactas y extrianguladas, así como para las abelianas. Su extensión se limita a los temas y resultados más relevantes para el desarrollo de los capítulos siguientes, por lo que el tratamiento no es exhaustivo. El capítulo inicia definiendo las categorías con objeto cero \textemdash de las cuales las categorías aditivas son un caso particular\textemdash, así como el núcleo y el conúcleo, y mostrando algunas equivalencias entre estas nociones y varias de las construcciones presentadas en el Capítulo \ref{Chap: Algunas nociones generales de la teoría de categorías}. Después, se discuten otros tipos de categorías previos a la construcción de categoría aditiva. Luego, se definen las nociones de categoría aditiva, funtor aditivo e ideal de una categoría aditiva, así como su categoría cociente asociada. Finalmente, se demuestran algunos resultados sobre bifuntores aditivos que serán ampliamente utilizados durante el Capítulo \ref{Chap: Categorías extrianguladas}. \\

En el Capítulo \ref{Chap: Categorías exactas} se estudian las categorías exactas. Al inicio, se expone y discute la definición de categoría exacta, que incluye la noción de ``sucesión exacta corta'' en este tipo de categoría dada por una clase particular de pares núcleo-conúcleo. Después, se demuestran algunas propiedades fundamentales en este tipo de estructuras, incluyendo algunos resultados bien conocidos en el área del álgebra homológica, como el Tercer Teorema de Isomorfismo de Noether. El capítulo termina con la definición del bifuntor aditivo $\text{Ext}^1$ en categorías exactas \textemdash el cual será retomado posteriormente en el Capítulo \ref{Chap: Categorías extrianguladas} para mostrar la relación entre las categorías exactas y las categorías extrianguladas\textemdash \ y de los pares de cotorsión correspondientes a dicho bifuntor. \\

El Capítulo \ref{Chap: Categorías trianguladas} tiene como función introducir las categorías trianguladas. Al inicio, se definen las nociones de funtor de traslación sobre una categoría aditiva y la categoría de triángulos asociada, así como los axiomas que debe cumplir una clase especial de triángulos, llamados ``triángulos distinguidos'' \textemdash cuyas propiedades se asemejan de varias maneras a las sucesiones exactas cortas en categorías abelianas\textemdash, para poder formar una categoría triangulada. Después, se demuestran algunas propiedades fundamentales de categorías trianguladas, incluyendo algunos resultados que tienen análogos directos en categorías extrianguladas. Finalmente, se definen los pares de cotorsión en categorías trianguladas, utilizando un bifuntor aditivo $\text{Ext}^1$ apropiado. \\

En el Capítulo \ref{Chap: Categorías extrianguladas} se estudian las categorías extrianguladas. El capítulo inicia discutiendo una conexión entre las categorías exactas y las trianguladas, dada por las categorías de Frobenius y sus categorías estables asociadas, cuya utilidad motiva la idea de categoría extriangulada. Posteriormente, se definen las categorías extrianguladas, se demuestran algunos resultados básicos relacionados con ellas, y se explora su relación con las categorías exactas y las categorías trianguladas, lo cual incluye ejemplos de categorías extrianguladas que las distinguen de las dos anteriores. Al final, se demuestran algunos resultados básicos para objetos proyectivos e inyectivos en categorías extrianguladas, y se presentan los pares de cotorsión en este contexto. \\

Finalmente, el Apéndice \ref{Chap: Categorías abelianas} contiene algunos resultados sobre categorías abelianas, acotados en función de su necesidad para el desarrollo de los Capítulos \ref{Chap: Categorías trianguladas} y \ref{Chap: Categorías extrianguladas}. La primera sección del apéndice presenta algunas nociones generales de la teoría de categorías que, en el contexto del presente trabajo, sólo conciernen a las categorías abelianas. Las siguientes secciones sirven para construir progresivamente la definición de categoría Puppe-exacta, donde se definen las nociones de ``sucesión exacta'' y ``sucesión exacta corta''. En las últimas secciones, se definen las categorías abelianas y se demuestran algunas de sus propiedades fundamentales, varias de las cuales pueden ser utilizadas para trazar analogías con algunos de los resultados expuestos en los Capítulos \ref{Chap: Categorías exactas}, \ref{Chap: Categorías trianguladas} y \ref{Chap: Categorías extrianguladas}. \\

Los contenidos de los Capítulos \ref{Chap: Algunas nociones generales de la teoría de categorías} y \ref{Chap: Categorías aditivas}, así como los del Apéndice \ref{Chap: Categorías abelianas}, se basaron en las notas del curso de Homología Relativa en Categorías Abelianas que impartió el Dr. Octavio Mendoza Hernández en el Programa de Posgrado en Ciencias Matemáticas de la Universidad Nacional Autónoma de México en el año 2021. Así mismo, los contenidos del Capítulo \ref{Chap: Categorías trianguladas} fueron basados en notas del curso de Categorías Trianguladas que impartió el Dr. Octavio en el mismo Programa el año siguiente. Por último, los contenidos de los Capítulos \ref{Chap: Categorías exactas} y \ref{Chap: Categorías extrianguladas} se basaron principalmente en los artículos de Bühler\cite{Bühler} y Nakaoka y Palu\cite{NakaokaPalu}, respectivamente. % Tal vez agregar más referencias bibliográficas

\end{document}
