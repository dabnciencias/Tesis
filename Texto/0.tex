\documentclass[tesis]{subfiles}
\begin{document}

\chapter{Algunas nociones generales de la teoría de categorías} \label{Chap: Algunas nociones generales de la teoría de categorías}    

El lenguaje de la teoría de categorías es de gran utilidad para estudiar el álgebra homológica de manera general y eficiente. En este capítulo, presentaremos los conceptos de teoría de categorías necesarios para la comprensión de los capítulos posteriores, que forman el cuerpo principal del texto. Únicamente las secciones \ref{Sec: Definición de categoría}, \ref{Sec: Morfismos especiales}, \ref{Sec: Funtores}, \ref{Sec: Dualidad} y \ref{Sec: Límites y colímites} son prerrequisitos para leer el Capítulo \ref{Chap: Categorías aditivas}. % 

\section{Definición de categoría} \label{Sec: Definición de categoría}

\begin{Def} \label{Def: Categoría}
    Una \emph{categoría} $\mathscr{C}$ se compone de:
    \begin{itemize}
        \item[$\bullet$] una clase de \emph{objetos} $\text{Obj}(\mathscr{C})$ de $\mathscr{C}$,

        \item[$\bullet$] una clase de \emph{morfismos} $\text{Mor}(\mathscr{C})$ de $\mathscr{C}$,

        \item[$\bullet$] una correspondencia parcialmente definida
            \[
            \circ:\text{Mor}(\mathscr{C})\times\text{Mor}(\mathscr{C})\to \text{Mor}(\mathscr{C}),
            \] 
        que satisfacen las siguientes propiedades:

        \begin{itemize}

            \item[(C1)] la clase de morfismos está determinada como sigue
                \[
                 \text{Mor}(\mathscr{C}) = \bigcup_{(A,B)\in\text{Obj}(\mathscr{C})^2} \text{Hom}_\mathscr{C}(A,B),
                \] 
                %donde $\text{Hom}_\mathscr{C}(A,B)$ es una clase para todo $(A,B)\in\text{Obj}(\mathscr{C})^2$;
                donde $\text{Hom}_\mathscr{C}(A,B)$ es una clase para cualquier par ($A,B$) de objetos en $\mathscr{C}$;

            \item[(C2)] para cualesquiera pares $(A,B),(C,D)$ de objetos en $\mathscr{C}$, se tiene que
                \[
                \text{Hom}_\mathscr{C}(A,B) = \text{Hom}_\mathscr{C}(C,D) \neq \varnothing \implies A=C \ \land \ B=D;
                \] 

            \item[(C3)] para cada terna $(A,B,C)$ de objetos en $\mathscr{C}$, la correspondencia parcial
            \[
            \circ:\text{Mor}(\mathscr{C})\times\text{Mor}(\mathscr{C})\to \text{Mor}(\mathscr{C}),
            \] 
            se restringe a una función bien definida
            \[
                \circ: \text{Hom}_\mathscr{C}(B,C)\times\text{Hom}_\mathscr{C}(A,B)\to \text{Hom}_\mathscr{C}(A,C), \ (f,g)\mapsto f\circ g := \circ(f,g)
            \]
            llamada usualmente ``composición de morfismos'' y que satisface las siguientes propiedades.

            \begin{itemize}
                \item[(i)] Asociatividad: para cualesquiera $f\in\text{Hom}_\mathscr{C}(A,B), g\in\text{Hom}_\mathscr{C}(B,C)$ y $h\in\text{Hom}_\mathscr{C}(C,D)$,
                    \[
                        h\circ(g\circ f) = (h\circ g)\circ f.
                    \] 
                    
                \item[(ii)] Existencia de identidades: para todo $X\in\text{Obj}(\mathscr{C})$, existe $1_X\in\text{Hom}_\mathscr{C}(X,X)$ tal que 
                    \[
                    1_X\circ f = f \quad \land \quad g\circ 1_X = g.
                    \] 
                    para cualesquiera $f\in\text{Hom}_\mathscr{C}(A,X)$ y $g\in\text{Hom}_\mathscr{C}(X,B)$.
            \end{itemize}
        \end{itemize}
    \end{itemize}

    \noindent Cada morfismo $f\in\text{Hom}_\mathscr{C}(A,B)$ se suele representar como $f:A\to B$ o bien como $A\xrightarrow[]{f}B$; en tal caso, $A=:\text{Dom}(f)$ es el \emph{dominio} de $f$ y $B=:\text{Codom}(f)$ es el \emph{codominio} de $f$. Frecuentemente escribiremos $gf$ en lugar de $g\circ f$. Para cualesquiera $A,B\in\text{Obj}(\mathscr{C})$, se suele denotar a $\text{Hom}_\mathscr{C}(A,B)$ como $\mathscr{C}(A,B)$; en particular, definimos $\text{End}_\mathscr{C}(A):= \text{Hom}_\mathscr{C}(A,A)$.
\end{Def}

\begin{Def}\label{Def: Categoría pequeña}
    Una categoría $\mathscr{C}$ es:
    \begin{enumerate}[label=(\alph*)]
    
        \item \emph{pequeña} si las clases $\text{Obj}(\mathscr{C})$ y $\text{Mor}(\mathscr{C})$ son conjuntos;

        \item \emph{localmente pequeña} si $\text{Hom}_\mathscr{C}(A,B)$ es un conjunto para cualesquiera $A,B\in\text{Obj}(\mathscr{C})$.
    \end{enumerate}
\end{Def}

\begin{Ejem}\label{Ejem: Categorías}

    Algunos ejemplos de categorías incluyen:

    \begin{enumerate}[label=(\arabic*)]
    
        \item Sets, donde $\text{Obj}(\text{Sets})$ es la clase de todos los conjuntos, $\text{Mor}(\text{Sets})$ es la clase de todas las funciones entre conjuntos y la composición de morfismos es la composición de funciones;

        \item Top, donde $\text{Obj}(\text{Top})$ es la clase de todos los espacios topológicos, $\text{Mor}(\text{Top})$ es la clase de todas las funciones continuas entre espacios topológicos y la composición de morfismos es la composición de funciones;

        \item $\text{Vect}_K$, donde $K$ es un campo, $\text{Obj}(\text{Vect}_K)$ es la clase de todos los espacios vectoriales sobre $K$, $\text{Mor}(\text{Vect}_K)$ es la clase de todas las transformaciones lineales compatibles con $K$ y la composición de morfismos es la composición de funciones;

        \item Grp, donde $\text{Obj}(\text{Grp})$ es la clase de todos los grupos, $\text{Mor}(\text{Grp})$ es la clase de todos los homomorfismos de grupos y la composición de morfismos es la composición de funciones;

        \item Ab, donde $\text{Obj}(\text{Ab})$ es la clase de todos los grupos abelianos, $\text{Mor}(\text{Ab})$ es la clase de todos los homomorfismos de grupos abelianos y la composición de morfismos es la composición de funciones;

        \item $\text{Mod}(R)$, donde $R$ es un anillo (asociativo con unidad), $\text{Obj}(\text{Mod}(R))$ es la clase de todos los $R$-módulos, $\text{Mor}(\text{Mod}(R))$ es la clase de todos los homomorfismos de $R$-módulos y la composición de morfismos es la composición de funciones;

        \item si $M$ es un monoide, podemos verlo como una categoría pequeña $\mathscr{M}$ con un solo objeto arbitrario (i.e., $\text{Obj}(\mathscr{M})=\{\ast\}$) tal que $\text{Mor}(\mathscr{M})=M$ y la composición de morfismos es el producto en $M$.
    \end{enumerate}

    \noindent En particular, tenemos que $\text{Vect}_K = \text{Mod}(K)$ y $\text{Ab} = \text{Mod}(\mathbb{Z})$. De esta forma, la categoría de módulos sobre un anillo se puede considerar como una generalización simultánea de la categoría de espacios vectoriales y la categoría de grupos abelianos.
\end{Ejem}

\begin{Def} \label{Def: Subcategoría}
    Sean $\mathscr{C}$ y $\mathscr{C}'$ categorías. Decimos que $\mathscr{C}'$ es una \emph{subcategoría} de $\mathscr{C}$ si se satisfacen las siguientes condiciones.

    \begin{itemize}
    
        \item[(SC1)] $\text{Obj}(\mathscr{C}')\subseteq\text{Obj}(\mathscr{C})$.

        \item[(SC2)] $\text{Hom}_{\mathscr{C}'}(A,B) \subseteq \text{Hom}_\mathscr{C}(A,B)$ para cualesquiera $A,B\in\text{Obj}(\mathscr{C}')$.
            
        \item[(SC3)] La composición de morfismos en $\mathscr{C}$, restringida a $\mathscr{C}'$, coincide con la composición en $\mathscr{C}'$.

        \item[(SC4)] Para todo $A\in\text{Obj}(\mathscr{C}')$, si $1'_A$ y $1_A$ son identidades en $\mathscr{C}'$ y $\mathscr{C}$, respectivamente, entonces $1'_A=1_A$.
    \end{itemize}

    \noindent Sea $\mathscr{C}'$ una subcategoría de $\mathscr{C}$. En caso de que la inclusión en (SC2) sea una igualdad, diremos que $\mathscr{C}'$ es una \emph{subcategoría plena} de $\mathscr{C}$, y lo denotaremos por $\mathscr{C}'\subseteq\mathscr{C}$.
\end{Def}

\begin{Ejem}\label{Ejem: Subcategorías}\leavevmode
    \begin{enumerate}[label=(\arabic*)]
    
        \item $\text{Vect}_K, \text{Ab}$ y $\text{Mod}(R)$ son subcategorías de $\text{Sets}$. 

        \item Si $R$ es un subanillo de un campo $K$, entonces podemos considerar a $\text{Vect}_K$ como subcategoría de $\text{Mod}(R)$ mediante la restricción de escalares. Similarmente, si $R$ es un subanillo de $\mathbb{Z}$, podemos considerar a $\text{Ab}$ como subcategoría de $\text{Mod}(R)$. Más generalmente, si $R$ es un subanillo de $S$, podemos considerar a $\text{Mod}(S)$ como subcategoría de $\text{Mod}(R)$.

        %\item Si $R$ es un anillo tal que el campo $K$ es un subanillo de $R$, podemos considerar a $\text{Vect}_K$ como subcategoría de $\text{Mod}(R)$. Similarmente, si $R$ contiene a $\mathbb{Z}$ como subanillo, podemos considerar a Ab como subcategoría de $\text{Mod}(R)$. 

        \item Ab es una subcategoría plena de Grp.

        \item Sean $\mathscr{C}$ una categoría y $\mathcal{X}$ una clase de objetos de $\mathscr{C}$. Entonces, se puede considerar a $\mathcal{X}$ como una subcategoría plena de $\mathscr{C}$, simplemente definiendo $\text{Obj}(\mathcal{X}):=\mathcal{X}$ y $\text{Hom}_\mathcal{X}(X,Y):=\text{Hom}_\mathscr{C}(X,Y)$ para cualesquiera $X,Y\in\mathcal{X}$.
    \end{enumerate}
\end{Ejem}

\section{Morfismos especiales} \label{Sec: Morfismos especiales}

\begin{Def} \label{Def: Morfismos especiales}
    Sea $A\xrightarrow[]{f} B$ un morfismo en una categoría $\mathscr{C}$. Decimos que:

    \begin{enumerate}[label=(\alph*)]
        \item $f$ es un \emph{monomorfismo} si, para cualesquiera $g,h\in\text{Mor}(\mathscr{C})$,
            \[
            fg=fh \implies g=h.
            \] 

        \item $f$ es un \emph{monomorfismo escindible} si existe un morfismo $B\xrightarrow[]{g} A$ en $\mathscr{C}$ tal que
            \[
                gf=1_A.
            \] 
            
        \item $f$ es un \emph{epimorfismo} si, para cualesquiera $g,h\in\text{Mor}(\mathscr{C})$,
            \[
            gf=hf \implies g=h.
            \] 
            
        \item $f$ es un \emph{epimorfismo escindible} si existe un morfismo $B\xrightarrow[]{g} A$ en $\mathscr{C}$ tal que
            \[
                fg=1_B.
            \] 

        \item $f$ es un \emph{isomorfismo} si existe un morfismo $B\xrightarrow[]{g} A$ en $\mathscr{C}$ tal que
            \[
            gf = 1_A \ \land \ fg = 1_B.
            \] 
    \end{enumerate}
\end{Def}

\begin{Obs}\label{Obs: Morfismos especiales}

    Sea $\mathscr{C}$ una categoría.
    \begin{enumerate}[label=(\arabic*)]

        \item Sea $A\xrightarrow[]{f}B\in\text{Mor}(\mathscr{C})$. Si existen $g,g'\in\text{Hom}_\mathscr{C}(B,A)$ tales que $gf=1_A$ y $fg'=1_B$, entonces $g'=g$. En particular, $f$ es un isomorfismo.

            En efecto: Por la asociatividad en la composición de morfismos, tenemos que
            \begin{align*}
                g' &= 1_Ag' \\
                   &= (gf)g' \\
                   &= g(fg') \\
                   &= g1_B \\
                   &= g.
            \end{align*}
            En particular, se sigue que el morfismo inverso de $f$ es único, y se denota por $f^{-1}$. Más aún, por simetría, $f^{-1}$ también es un isomorfismo en $\mathscr{C}$.

        \item La composición de monomorfismos (epimorfismos) es un monomorfismo (epimorfismo).

            En efecto: Sean $A,B,C\in\text{Obj}(\mathscr{C})$ y $A\xrightarrow[]{f} B, B\xrightarrow[]{g} C$ morfismos en $\mathscr{C}$.

            Supongamos que $f$ y $g$ son monomorfismos y que existen $X\in\text{Obj}(\mathscr{C})$ y $h,h'\in\text{Hom}_\mathscr{C}(X,A)$ tales que $(gf)h = (gf)h'$. Entonces
            \begin{align*}
                (gf)h = (gf)h' &\implies g(fh) = g(fh') \\ &\implies fh = fh' \tag{$g$ es monomorfismo} \\
                                   &\implies h = h', \tag{$f$ es monomorfismo}
            \end{align*}
            \noindent por lo que $A\xrightarrow[]{gf} C$ es un monomorfismo.

            Ahora, supongamos que $f$ y $g$ son epimorfismos y que existen $Y\in\text{Obj}(\mathscr{C})$ y $j,j'\in\text{Hom}_\mathscr{C}(C,Y)$ tales que $j(gf) = j'(gf)$. Entonces
            \begin{align*}
            j(gf) = j'(gf) &\implies (jg)f = (j'g)f \\ &\implies jg = j'g \tag{$f$ es epimorfismo} \\
                                   &\implies j = j', \tag{$g$ es epimorfismo}
            \end{align*}
            \noindent por lo que $A\xrightarrow[]{gf} C$ es un epimorfismo.

        \item La composición de isomorfismos es un isomorfismo.

            En efecto: Sean $A\xrightarrow[]{f} B$ y $B\xrightarrow[]{g} C$ isomorfismos en $\mathscr{C}$. Entonces, existen morfismos $B\xrightarrow[]{f^{-1}} A, C\xrightarrow[]{g^{-1}} B$ en $\mathscr{C}$ tales que $f^{-1}f=1_A, ff^{-1}=1_B=g^{-1}g$ y $gg^{-1}=1_C$. Luego, se sigue que el morfismo $C\xrightarrow{f^{-1}g^{-1}} A$ es tal que
            \begin{align*}
                (f^{-1}g^{-1})gf &= f^{-1}(g^{-1}g)f \\
                                 &= f^{-1}f \\
                                 &= 1_A, \\ \\
                (gf)(f^{-1}g^{-1}) &= g(ff^{-1})g^{-1} \\
                                   &= gg^{-1} \\
                                   &= 1_C,
            \end{align*}
            por lo que $gf$ es un isomorfismo y $(gf)^{-1}=f^{-1}g^{-1}$.

        \item Todo monomorfismo (epimorfismo) escindible es un monomorfismo (epimorfismo).

            En efecto: Sea $A\xrightarrow[]{f} B$ un morfismo en $\mathscr{C}$.

            Supongamos que $f$ es un monomorfismo escindible. Entonces existe un morfismo $B\xrightarrow[]{g} A$ en $\mathscr{C}$ tal que $gf = 1_A$. Supongamos que existen $D\in\text{Obj}(\mathscr{C})$ y $j,j'\in\text{Hom}_\mathscr{C}(D,A)$ tales que $fj = fj'$. Entonces
            \begin{align*}
                fj = fj' &\implies g(fj) = g(fj') \\
                &\implies (gf)j = (gf)j' \\
                &\implies 1_A j = 1_A j' \\
                &\implies j = j',
            \end{align*}
            \noindent por lo que $f$ es un monomorfismo.

            Ahora, supongamos que $f$ es un epimorfismo escindible. Entonces existe un morfismo $B\xrightarrow[]{g} A$ en $\mathscr{C}$ tal que $fg = 1_B$. Supongamos que existen $C\in\text{Obj}(\mathscr{C})$ y $h,h'\in\text{Hom}_\mathscr{C}(B,C)$ tales que $hf = h'f$. Entonces
            \begin{align*}
                hf = h'f &\implies (hf)g = (h'f)g \\
                &\implies h(fg) = h'(fg) \\
                               &\implies h 1_B = h' 1_B \\
                               &\implies h = h',
            \end{align*}
            \noindent por lo que $f$ es un epimorfismo.

        \item Si $gf$ es un monomorfismo (epimorfismo), entonces $f$ es un monomorfismo ($g$ es un epimorfismo).

            En efecto: Sean $X\xrightarrow[]{f} Y, Y\xrightarrow[]{g} Z$ morfismos en $\mathscr{C}$.

            Supongamos que $gf$ es un monomorfismo. Si $W\in\text{Obj}(\mathscr{C})$ y $h,h'\in\text{Hom}_\mathscr{C}(W,X)$ son tales que $fh = fh'$, entonces
            \begin{align*}
                fh = fh' &\implies g(fh) = g(fh') \\
                &\implies (gf)h = (gf)h' \\
                               &\implies h = h', \tag{$gf$ es un monomorfismo}
            \end{align*}
            por lo que $f$ es un monomorfismo.

            Ahora, supongamos que $gf$ es un epimorfismo. Si $W\in\text{Obj}(\mathscr{C})$ y $h,h'\in\text{Hom}_\mathscr{C}(Z,W)$ son tales que $hg = h'g$, entonces
            \begin{align*}
                hg = h'g &\implies (hg)f = (h'g)f \\
                &\implies h(gf) = h'(gf) \\
                               &\implies h = h', \tag{$gf$ es un epimorfismo}
            \end{align*}
            por lo que $g$ es un epimorfismo.

        \item Si $f$ es un isomorfismo, entonces es un monomorfismo y un epimorfismo.

            En efecto: Se sigue de (3) pues, por definición, todo isomorfismo es un monomorfismo escindible y un epimorfismo escindible.

        \item $f$ es un isomorfismo si, y sólo si, $f$ es un monomorfismo escindible y un epimorfismo escindible.

            En efecto: Por definición de isomorfismo, basta ver que si $f$ es un monomorfismo escindible y un epimorfismo escindible entonces es un isomorfismo. Supongamos que $A\xrightarrow[]{f} B$ en $\mathscr{C}$ es un monomorfismo escindible y un epimorfismo escindible. Entonces, existen morfismos $g,g':B\to A$ en $\mathscr{C}$ tales que $gf = 1_A$ y $fg' = 1_B$. Luego,
            \begin{align*}
                gf = 1_A &\iff (gf)g' = 1_Ag' \\
                         &\iff g(fg') = g' \\
                         &\iff g1_B = g' \\
                         &\iff g = g',
            \end{align*}
            de donde se sigue que $f$ es un isomorfismo.

        \item Si $f$ es un monomorfismo (epimorfismo) escindible y un epimorfismo (monomorfismo), entonces $f$ es un isomorfismo.

            En efecto: Supongamos que $A\xrightarrow[]{f} B$ un monomorfismo escindible y un epimorfismo. Entonces, por ser un monomorfismo escindible, existe $B\xrightarrow[]{g} A$ en $\mathscr{C}$ tal que $gf=1_A$. Por la existencia de la identidad $1_B$ en $\mathscr{C}$ y la asociatividad de la composición de morfismos, tenemos que
            \begin{align*}
                1_Bf &= f1_A \\
                     &= f(gf) \\
                     &= (fg)f.
            \end{align*}
            Luego, como $f$ es un epimorfismo, se sigue que $fg=1_B$, por lo que $f$ es un isomorfismo.

            \noindent Ahora, supongamos que $A\xrightarrow[]{f} B$ es un epimorfismo escindible y un monomorfismo. Entonces, por ser un epimorfismo escindible, existe $B\xrightarrow[]{g} A$ en $\mathscr{C}$ tal que $fg=1_B$. Por la existencia de la identidad $1_A$ en $\mathscr{C}$ y la asociatividad de la composición de morfismos, tenemos que
            \begin{align*}
                f1_A &= 1_Bf \\
                     &= (fg)f \\
                     &= f(gf). 
            \end{align*}
            Luego, como $f$ es un monomorfismo, se sigue que $gf=1_A$, por lo que $f$ es un isomorfismo.

        \noindent Las flechas $\hookrightarrow, \twoheadrightarrow, \xrightarrow[]{\sim}$ se utilizan para denotar monomorfismos, epimorfismos e isomorfismos, respectivamente.
    \end{enumerate}
\end{Obs}

\begin{Def}\label{Def: retracción y seccióñ}
    Sean $\mathscr{C}$ una categoría y $A\xrightarrow[]{f} B$ un morfismo en $\mathscr{C}$.

    \begin{enumerate}[label=(\alph*)]
    
        \item Una \emph{retracción} de $f$ es un morfismo $B\xrightarrow[]{r} A$ en $\mathscr{C}$ tal que $rf=1_A$.

        \item Una \emph{sección} de $f$ es un morfismo $B\xrightarrow[]{s} A$ en $\mathscr{C}$ tal que $fs=1_B$.
    \end{enumerate}
\end{Def}

\begin{Obs}\label{Obs: retracción y sección}
    Toda retracción es un epimorfismo escindible y, similarmente, toda sección es un monomorfismo escindible. Por ende, hablar de una retracción o una sección sin hacer referencia explícita a un morfismo es equivalente a hablar de epimorfismos escindibles y monomorfismos escindibles, respectivamente.
\end{Obs}

\begin{Def}\label{Def: Relación de isomorfismo}
    Sea $\mathscr{C}$ una categoría. Dos objetos $X,Y\in\text{Obj}(\mathscr{C})$ son \emph{isomorfos}, denotado por $X\simeq Y$, si existe un isomorfismo en $\text{Hom}_\mathscr{C}(X,Y)$ o bien en $\text{Hom}_\mathscr{C}(Y,X)$.
\end{Def}

\begin{Obs}\label{Observaciones sobre la relación de isomorfismo}
    La relación de isomorfismo es una relación de equivalencia en la clase de objetos de una categoría. Más aún, esta noción general de isomorfismo coincide con las nociones usuales de isomorfismo en las categorías $\text{Vect}_K, \text{Grp}, \text{Ab}$ y $\text{Mod}(R)$, y con la noción de homeomorfismo en la categoría Top; en Sets, dos objetos son isomorfos si tienen la misma cardinalidad. Usualmente, esta relación se denota por $\simeq$.

    \vspace{1mm}
    \noindent Demostraremos la primera afirmación: La reflexividad se sigue de observar que todo morfismo identidad es un isomorfismo que se tiene a sí mismo como inverso, la simetría se sigue de la definición de isomorfismo y la transitividad se sigue del inciso (3) de la Observación \ref{Obs: Morfismos especiales}.
\end{Obs}

\section{Funtores} \label{Sec: Funtores}

\begin{Def} \label{Def: Funtor}
    Sean $\mathscr{C}$ y $\mathscr{D}$ categorías. Una correspondencia $F:\mathscr{C}\to \mathscr{D}$ es un \emph{funtor covariante} si satisface las siguientes condiciones:

    \begin{itemize}
    
        \item[(F1)] Si $A\in\text{Obj}(\mathscr{C})$, entonces $F(A)\in\text{Obj}(\mathscr{D})$.

        \item[(F2)] Si $A\xrightarrow[]{f}B\in\text{Mor}(\mathscr{C})$, entonces $F(A)\xrightarrow[]{F(f)} F(B)\in\text{Mor}(\mathscr{D})$.

        \item[(F3)] Si $A\xrightarrow[]{f}B,B\xrightarrow[]{g}C\in\text{Mor}(\mathscr{C})$, entonces $F(gf) = F(g)F(f).$

        \item[(F4)] $F(1_A) = 1_{F(A)}$ para todo $A\in\text{Obj}(\mathscr{C})$.
    \end{itemize}
    En otras palabras, un funtor covariante es una correspondencia entre dos categorías que preserva la composición de morfismos y las identidades. Una correspondencia $F:\mathscr{C}\to \mathscr{D}$ es un \emph{funtor contravariante} si, en vez de las condiciones (F2) y (F3), se satisfacen las siguientes condiciones:

    \begin{itemize}
    
        \item[(FCNT2)] Si $A\xrightarrow[]{f}B\in\text{Mor}(\mathscr{C})$, entonces $F(B)\xrightarrow[]{F(f)}F(A)\in\text{Mor}(\mathscr{D})$.

        \item[(FCNT3)] Si $A\xrightarrow[]{f}B,B\xrightarrow[]{g}C\in\text{Mor}(\mathscr{C})$, entonces $F(gf) = F(f)F(g).$
    \end{itemize}
Por ende, un funtor contravariante es una correspondencia entre dos categorías que invierte la composición de morfismos y preserva las identidades. 
\end{Def}

\begin{Obs}\label{Obs: Funtores}\leavevmode

    \begin{enumerate}[label=(\arabic*)]

        \item Los funtores preservan isomorfismos.

            En efecto: Supongamos que $F:\mathscr{C}\to \mathscr{D}$ es un funtor covariante. Sea $A\xrightarrow[]{f}B$ un isomorfismo en $\mathscr{C}$. Entonces, existe $f^{-1}\in\text{Hom}_\mathscr{C}(B,A)$ tal que $f^{-1}f=1_A$ y $ff^{-1}=1_B$. Luego,
            \begin{align*}
                F(f^{-1})F(f) &= F(f^{-1}f) \\
                              &= F(1_A) \\
                              &= 1_{F(A)}, \\ \\
                F(f)F(f^{-1}) &= F(ff^{-1}) \\
                              &= F(1_B) \\
                              &= 1_{F(B)},
            \end{align*}
            por lo que $F(A)\xrightarrow[]{F(f)}F(B)$ tiene a $F(B)\xrightarrow[]{F(f^{-1})}F(A)$ como inverso. El caso contravariante es análogo.

        \item La restricción de un funtor a cualquier subcategoría de su dominio es un funtor de la misma varianza.

        \item Sean $F:\mathscr{C}\to \mathscr{D}$ y $G:\mathscr{D}\to \mathscr{E}$ funtores. Si definimos la composición $GF:\mathscr{C}\to \mathscr{E}$ como
            \[
                (GF)(\cdot) := G(F(\cdot)),
            \] 
            entonces la composición de funtores $GF$ es un funtor covariante si $F$ y $G$ tienen la misma varianza, y un funtor contravariante si $F$ y $G$ tienen varianza distinta.

   \end{enumerate}
\end{Obs}

\begin{Ejem}\label{Ejem: Funtores}

    Sean $\mathscr{C}$ una categoría y $\mathscr{C}'$ una subcategoría de $\mathscr{C}$.

    \begin{enumerate}[label=(\arabic*)]
    
        \item El funtor identidad $1_\mathscr{C}:\mathscr{C}\to \mathscr{C}, \big(X\xrightarrow[]{f}Y\big)\mapsto \big(X\xrightarrow[]{f}Y\big)$.

        \item El funtor inclusión $\iota_{\mathscr{C}'}:\mathscr{C}'\to \mathscr{C}, \big(X\xrightarrow[]{f}Y\big)\mapsto \big(X\xrightarrow[]{f}Y\big)$, es decir, $\iota_{\mathscr{C}'} = 1_\mathscr{C}\mid_{\mathscr{C}'}$.

        \item El funtor Hom-covariante, si $\mathscr{C}$ es localmente pequeña. Sea $A\in\text{Obj}(\mathscr{C})$. Entonces,
        \[
            \text{Hom}_\mathscr{C}(A,-):\mathscr{C}\to \text{Sets}, \big(X\xrightarrow[]{f}Y\big) \mapsto \big( \text{Hom}_\mathscr{C}(A,X) \xrightarrow[]{\text{Hom}_\mathscr{C}(A,f)} \text{Hom}_\mathscr{C}(A,Y) \big),
        \]
        donde $\text{Hom}_\mathscr{C}(A,f)(\alpha) := f\circ\alpha$ para cualquier $\alpha\in\text{Hom}_\mathscr{C}(A,X)$.

        \item El funtor Hom-contravariante, si $\mathscr{C}$ es localmente pequeña. Sea $A\in\text{Obj}(\mathscr{C})$. Entonces,
        \[
            \text{Hom}_\mathscr{C}(-,A):\mathscr{C}\to \text{Sets}, \big(X\xrightarrow[]{f}Y\big) \mapsto \big( \text{Hom}_\mathscr{C}(Y,A) \xrightarrow[]{\text{Hom}_\mathscr{C}(f,A)} \text{Hom}_\mathscr{C}(X,A) \big),
        \]
        donde $\text{Hom}_\mathscr{C}(f,A)(\beta) := \beta\circ f$ para cualquier $\beta\in\text{Hom}_\mathscr{C}(Y,A)$.
    \end{enumerate}
\end{Ejem}

\begin{Obs}\label{Obs: La categoría Cat}\leavevmode

    \begin{enumerate}[label=(\arabic*)]
    
        \item Podemos considerar a la categoría Cat, donde $\text{Obj}(\text{Cat})$ es la clase de todas las categorías, $\text{Mor}(\text{Cat})$ es la clase de todos los funtores covariantes y la composición de morfismos es la composición de funtores descrita en el inciso (3) de la Observación \ref{Obs: Funtores}. Un funtor covariante $F:\mathscr{C}\to \mathscr{D}$ es un isomorfismo si existe un funtor covariante $G:\mathscr{D}\to \mathscr{C}$ tal que $GF=1_\mathscr{C}$ y $FG=1_\mathscr{D}$.

        \item Por el inciso (3) de la Observación \ref{Obs: Funtores}, no es posible formar una categoría cuyos objetos sean categorías y cuyos morfismos sean funtores contravariantes.
    \end{enumerate}
\end{Obs}

\begin{Def}\label{Def: Isomorfismo de categorías, endofuntor y automorfismo}\leavevmode
    \begin{enumerate}[label=(\alph*)]
    
        \item Un \emph{isomorfismo de categorías} es un funtor covariante que es un isomorfismo en $\text{Cat}$.

        \item Un \emph{endomorfismo} es un funtor covariante de una categoría en sí misma.

        \item Un \emph{automorfismo} es un endomorfismo que es un isomorfismo de categorías.

        \item Un funtor contravariante $F:\mathscr{C}\to \mathscr{D}$ es un \emph{anti-isomorfismo} si existe un funtor contravariante $G:\mathscr{D}\to \mathscr{C}$ tal que
            \[
            GF = 1_\mathscr{C} \quad \land \quad FG=1_\mathscr{D}.
            \] 
    \end{enumerate}
\end{Def}

\begin{Nota}
    En todo lo que sigue, supondremos que los funtores son covariantes, a menos que se especifique lo contrario explícitamente.
\end{Nota}

\begin{Def}\label{Def: Funtor fiel, pleno y denso}
    Para un funtor $F:\mathscr{C}\to \mathscr{D}$, con $\mathscr{D}$ localmente pequeña, decimos que:

    \begin{enumerate}[label=(\alph*)]
    
        \item $F$ es fiel si, para cualesquiera $X,Y\in\text{Obj}(\mathscr{C})$,
            \[
                F:\text{Hom}_\mathscr{C}(X,Y)\to \text{Hom}_\mathscr{D}(FX,FY)
            \] 
            es un monomorfismo en Sets.

        \item $F$ es pleno si, para cualesquiera $X,Y\in\text{Obj}(\mathscr{C})$,
            \[
                F:\text{Hom}_\mathscr{C}(X,Y)\to \text{Hom}_\mathscr{D}(FX,FY)
            \] 
            es un epimorfismo en Sets.

        \item $F$ es denso si, para todo $C\in\text{Obj}(\mathscr{C})$, existe $D\in\text{Obj}(\mathscr{D})$ tal que $F(C)\simeq D$ en $\mathscr{D}$.
    \end{enumerate}
\end{Def}

%\begin{Def}\label{Def: Imagen de un funtor}
%    Sea $F:\mathscr{C}\to \mathscr{D}$ un funtor. La \emph{imagen} de $F$, denotada por $F(\mathscr{C})$, es la colección de objetos y morfismos en $\mathscr{D}$ obtenidos como imágenes de objetos y morfismos en $\mathscr{C}$, bajo $F$.
%\end{Def}
%
%\begin{Obs}\label{Obs: Imagen de un funtor}
%    Sea $F:\mathscr{C}\to \mathscr{D}$ un funtor. Entonces su imagen $F(\mathscr{C})$ es una subcategoría de $\mathscr{D}$.
%\end{Obs}

%\subsection*{Diagramas} \label{Sssec: Diagramas}
%
%\begin{Def} \label{Def: Diagrama}
%    Sean $\mathscr{C},\mathscr{D}$ categorías y $F:\mathscr{C}\to \mathscr{D}$ un funtor. $F$ es un \emph{diagrama} si $\mathscr{C}$ es una categoría pequeña. En particular, $F$ es un \emph{diagrama finito} si $\text{Obj}(\mathscr{C})$ es un conjunto finito.
%\end{Def}
%
%\begin{Obs} \label{Obs: Diagramas}
%    Sean $\mathscr{C},\mathscr{D},\mathscr{E}$ categorías, $F:\mathscr{C}\to \mathscr{D}$ un diagrama y $G:\mathscr{D}\to \mathscr{E}$ un funtor.
%
%    \begin{enumerate}[label=(\arabic*)]
%    
%        \item $F(\mathscr{C})$ es una subcategoría pequeña de $\mathscr{D}$.
%
%            En efecto: Se sigue del inciso (3) de la Observación \ref{Obs: Funtores} y de que $\{F(A)\}_{A\in\text{Obj}(\mathscr{C})}$ es un conjunto, pues $\text{Obj}(\mathscr{C})$ lo es.
%
%        \item La restricción de $G$ a cualquier subcategoría pequeña de $\mathscr{D}$ es un diagrama.
%
%            En efecto: Sea $\mathscr{F}$ una subcategoría pequeña de $\mathscr{D}$. Entonces el funtor inclusión $\iota_\mathscr{F}:\mathscr{F}\to \mathscr{D}$ es un diagrama, por lo que la composición $G\circ \iota_\mathscr{F}:\mathscr{F}\to \mathscr{E}$, que es igual a la restricción de $G$ a $\mathscr{F}$, es un diagrama. 
%
%        \item En vista de (1) y (2), se sigue que se sigue que la restricción $G'$ de $G$ a $F(\mathscr{C})$ es un diagrama. En particular, la composición de funtores $G'F:\mathscr{C}\to \mathscr{E}$ es un diagrama que preserva la composición de morfismos si $G$ es un funtor covariante, y que invierte la composición de morfismos si $G$ es un funtor contravariante.
%    \end{enumerate}
%\end{Obs}
%
%%Ejemplos de diagramas.
%
%A continuación, describimos una forma muy útil de visualizar diagramas a través de sus imágenes: Dado un diagrama $F:\mathscr{C}\to \mathscr{D}$, hacemos un arreglo de la familia de objetos $\{F(A)\}_{A\in\text{Obj}(\mathscr{C})}$ y unimos dichos objetos por los morfismos correspondientes de $\{F(f)\}_{f\in\text{Mor}(\mathscr{C})}$, omitiendo los morfismos identidad y aquellos obtenidos como composición de morfismos entre los objetos de $\{F(A)\}_{A\in\text{Obj}(\mathscr{C})}$. Algunos ejemplos de diagramas finitos visualizados de esta manera son los siguientes:
%\begin{equation} \label{Ejem: Diagramas}
%    \begin{tikzcd}
%        X \arrow[]{rr}[]{h} \arrow[]{dr}[swap]{f} & &Z, & &V \arrow[]{d}[swap]{k} \arrow[]{r}[]{j} &X \arrow[]{d}[]{f} \\
%                                                  &Y \arrow[]{ur}[swap]{g} & & &W \arrow[]{r}[swap]{i} &Y.
%    \end{tikzcd}
%\end{equation}
%De esta manera, los diagramas nos ayudan a mantener un registro de cómo se relacionan distintos morfismos en una categoría a través de sus dominios y codominios. En diagramas, es común denotar a los morfismos identidad con flechas de igualdad, es decir,
%\begin{center}
%    \begin{tikzcd}
%        X \arrow[equals]{r}[]{} &X
%    \end{tikzcd}
%    en vez de
%    \begin{tikzcd}
%        X \arrow[]{r}[]{1_X} &X.
%    \end{tikzcd}
%\end{center}
%
%\noindent Cuando todas las composiciones de morfismos en un diagrama que empiezan en un mismo punto y terminan en un mismo punto son equivalentes, decimos que el diagrama es conmutativo. Por ejemplo, el primer diagrama en (\ref{Ejem: Diagramas}) es conmutativo si $h=gf$, mientras que el segundo es conmutativo si $fj=ik$. Observemos que, si suponemos que los dos diagramas en (\ref{Ejem: Diagramas}) conmutan, entonces se sigue que el diagrama
%\begin{center}
%    \begin{tikzcd}
%        V \arrow[]{d}[swap]{k} \arrow[]{r}[]{j} &X \arrow[]{d}[]{f} \arrow[]{r}[]{h} &Z \\
%        W \arrow[]{r}[swap]{i} &Y \arrow[]{ur}[swap]{g}
%    \end{tikzcd}
%\end{center}
%es conmutativo. De esta forma, los diagramas conmutativos nos pueden ayudar a visualizar sistemas de ecuaciones de morfismos en una categoría.
%
%\begin{Def} \label{Def: Diagrama en una categoría}
%    Sean $\mathscr{C}$ una categoría pequeña, $\mathscr{D}$ una categoría y $F:\mathscr{C}\to \mathscr{D}$ un diagrama. Un \emph{diagrama en} $\mathscr{D}$ es una visualización de $F$ a través de su imagen, como en (\ref{Ejem: Diagramas}). Cuando sea claro en qué categoría se encuentra la imagen, diremos simplemente \emph{diagrama}. Una composición de morfismos en un diagrama es un \emph{camino}, y un diagrama es \emph{conmutativo} si todos los caminos que empiezan en un mismo punto inicial y terminan en un mismo punto final son equivalentes.
%\end{Def}
%
%Los diagramas conmutativos también nos pueden ayudar a enunciar afirmaciones, como veremos en el siguiente ejemplo:

\subsection*{Diagramas} \label{Ssec: Diagramas}

\begin{Def}

    Sea $\mathscr{C}$ una categoría. 
    \begin{enumerate}[label=(\alph*)]
    
        \item Un \emph{diagrama} en $\mathscr{C}$ es un conjunto $D$ formado por objetos y morfismos en $\mathscr{C}$ tal que
            \[
            \{\text{Dom}(f), \text{Codom}(f)\}\subseteq D
            \] 
            para todo $f\in D\cap\text{Mor}(\mathscr{C})$.

        \item Sea $D$ un diagrama en $\mathscr{C}$. Un \emph{camino} $\gamma$ en $D$ es una composición finita de morfismos en $D$, y su \emph{longitud} es el número de flechas que lo componen.

        \item Un diagrama $D$ en $\mathscr{C}$ es \emph{conmutativo} si, para cualesquiera caminos $\gamma$ y $\delta$ en $D$ tales que $\text{Dom}(\gamma) = \text{Dom}(\delta)$ y $\text{Codom}(\gamma) = \text{Codom}(\delta)$, se tiene que $\gamma=\delta$. 
    \end{enumerate}

    Frecuentemente representaremos a un diagrama $D$ en $\mathscr{C}$ como un grafo dirigido con vértices $D\cap\text{Obj}(\mathscr{C})$ y flechas $D\cap\text{Mor}(\mathscr{C})$.
\end{Def}

\begin{Ejem}
    Sea $\mathscr{C}$ una categoría.

    \begin{enumerate}[label=(\arabic*)]
    
        \item Para el diagrama $D$ en $\mathscr{C}$ representado por el grafo dirigido
            \begin{center}
                \begin{tikzcd}
                    X_1 \arrow[shift left]{r}[]{f} \arrow[shift right]{r}[swap]{g} &X_2,
                \end{tikzcd}
            \end{center}
            los morfismos $f$ y $g$ en $\mathscr{C}$ son los únicos caminos. Note que $D$ es conmutativo si $f=g$.

        \item Para el diagrama en $\mathscr{C}$
            \begin{center}
                \begin{tikzcd}
                    &&X_3 \arrow[]{dr}[]{f_3} \\
                    X_1 \arrow[]{r}[]{f_1} &X_2 \arrow[]{ur}[]{f_2} \arrow[]{dr}[swap]{f_4} &&X_4 \\
                                           &&X_5 \arrow[]{ur}[swap]{f_5},
                \end{tikzcd}
            \end{center}
            los caminos posibles son: 

            \begin{itemize}
            
                \item $f_1, \ f_2, \ f_3, \ f_4$ y $f_5$, de longitud 1;

                \item $f_2f_1, \ f_4f_1, \ f_3f_2$ y $f_5f_4$, de longitud 2;

                \item $f_3f_2f_1$ y $f_5f_4f_1$, de longitud 3.
            \end{itemize}

            Note que el diagrama anterior es conmutativo si, y sólo si, $f_3f_2 = f_5f_4$.

        \item Cualquier diagrama sin caminos es conmutativo, por vacuidad.

        \item Sean $\mathscr{D}$ una categoría, $F:\mathscr{C}\to \mathscr{D}$ un funtor y $D$ un diagrama en $\mathscr{C}$. Aplicando el funtor $F$ a cada objeto y morfismo en $D$ obtenemos el diagrama $F(D)$ en $\mathscr{D}$.
            
    \end{enumerate}
\end{Ejem}

Los diagramas nos ayudan a mantener un registro de cómo se relacionan distintos morfismos en una categoría a través de sus dominios y codominios. En particular, por (C3), nos muestran cuáles de los morfismos representados son componibles. Más aún, los diagramas conmutativos nos permiten visualizar sistemas de ecuaciones entre morfismos de una categoría de manera sencilla.

\begin{Nota}
    En diagramas, es común denotar a los morfismos identidad con flechas de igualdad, es decir,
    \begin{center}
        \begin{tikzcd}
            X \arrow[equals]{r}[]{} &X
        \end{tikzcd}
        en vez de
        \begin{tikzcd}
            X \arrow[]{r}[]{1_X} &X.
        \end{tikzcd}
    \end{center}
    Además, comúnmente se utilizan las flechas \begin{tikzcd} {}\arrow[hook]{r}[]{} &{}\end{tikzcd}, \begin{tikzcd} {}\arrow[two heads]{r}[]{} &{}\end{tikzcd} y \begin{tikzcd} {}\arrow[]{r}[]{\sim} &{}\end{tikzcd} para denotar monomorfismos, epimorfismos e isomorfismos, respectivamente, así como flechas punteadas para denotar la existencia del morfismo correspondiente.

\end{Nota}

\begin{Def} \label{Def: Factorización}
    Sean $\mathscr{C}$ una categoría y $A\xrightarrow[]{f}C, B\xrightarrow[]{g}C\in\text{Mor}(\mathscr{C})$. Decimos que el morfismo $f$ se \emph{factoriza a través de} $g$, si existe $A\xrightarrow[]{f'}B\in\text{Mor}(\mathscr{C})$ tal que $f=gf'$; diagramáticamente,
    \begin{equation}\label{eq: Factorización}
        \begin{tikzcd}
            A \arrow[]{rr}[]{f} \arrow[dotted]{dr}[swap]{f'} & &C. \\
                                                             &B \arrow[]{ur}[swap]{g}
        \end{tikzcd}
    \end{equation}
    En tal caso, decimos que $g$ y $f'$ son \emph{factores} de $f$, que $gf'$ es una \emph{factorización} de $f$, y que el factor $f'$ \emph{hace conmutar} el diagrama (\ref{eq: Factorización}). Si $f'$ es el único morfismo con esta propiedad, lo denotamos como
    \begin{center}
        \begin{tikzcd}
            A \arrow[]{rr}[]{f} \arrow[dotted]{dr}[swap]{\exists! \ f'} & &C. \\
                                                             &B \arrow[]{ur}[swap]{g}
        \end{tikzcd}
    \end{center}
\end{Def}

\begin{Def}\label{Def: Propiedad universal}
    Una \emph{propiedad universal} es una afirmación de la existencia de una factorización tal que el factor existente es el único que hace conmutar cierto diagrama. En tal caso, decimos que dicha factorización es la \emph{única} que hace conmutar ese diagrama.
\end{Def}

Una gran cantidad de conceptos categóricos importantes se definen en términos de propiedades universales, como veremos en las sección \ref{Sec: Límites y colímites}. %y \ref{Sec: Localización}.
%Una gran cantidad de conceptos categóricos importantes se definen en términos de propiedades universales, como veremos en las secciones \ref{Ssec: Objetos finales, objetos iniciales y objetos cero}, \ref{Ssec: Igualadores y coigualadores}, \ref{Mendoza-1.2}, \ref{Mendoza-1.8}, \ref{Ssec: Localización} y \ref{Mendoza-1.5}.

\subsection*{Categoría cociente y categoría producto} \label{Ssec: Categoría cociente y categoría producto}

\begin{Def}\label{Def: Relación de congruencia y categoría cociente}
    Sean $\mathscr{C}$ una categoría localmente pequeña y $\cong$ una relación en $\text{Mor}(\mathscr{C})$. Decimos que $\cong$ es una \emph{relación de congruencia} en $\text{Mor}(\mathscr{C})$ si, para cualesquiera $X,Y\in\text{Obj}(\mathscr{C})$, al restringir $\cong$ en cada $\text{Hom}_\mathscr{C}(X,Y)$ se obtiene una relación de equivalencia en el conjunto $\text{Hom}_\mathscr{C}(X,Y)$, la cual denotaremos por $\simeq_{X,Y}$, que respeta la composición de morfismos; es decir, si $f_1,f_2\in\text{Hom}_\mathscr{C}(X,Y)$ y $g_1,g_2\in\text{Hom}_\mathscr{C}(Y,Z)$ son tales que $f_1 \simeq_{X,Y} f_2$ y $g_1 \simeq_{Y,Z} g_2$, entonces $g_1f_1 \simeq_{X,Z} g_2f_2$. Si $\cong$ es una relación de congruencia en $\text{Mor}(\mathscr{C})$, definimos a la \emph{categoría cociente} $\mathscr{C}/\cong$ por
    \begin{align*}
        \text{Obj}(\mathscr{C}/\cong) &:= \text{Obj}(\mathscr{C}), \\
        \text{Hom}_{\mathscr{C}/\cong}(X,Y) &:= \text{Hom}_\mathscr{C}(X,Y)/\simeq_{X,Y} \quad \forall \ X,Y\in\text{Obj}(\mathscr{C}),
    \end{align*}
    con la composición de morfismos inducida por la composición en $\mathscr{C}$. Al funtor $F:\mathscr{C}\to \mathscr{C}/\cong$ que manda a cada morfismo en su clase de equivalencia se le conoce como \emph{funtor cociente}. Se dice que el funtor cociente \emph{desciende} a los morfismos en $\mathscr{C}$ a sus clases de equivalencia en $\mathscr{C}/\cong$.
\end{Def}

\begin{Def}\label{Def: Categoría producto}
    Sean $\mathscr{C}$ y $\mathscr{C}'$ categorías. Definimos a la \emph{categoría producto} $\mathscr{C}'\times \mathscr{C}$ por
    \begin{align*}
        \text{Obj}(\mathscr{C}'\times \mathscr{C}) &:= \text{Obj}(\mathscr{C}')\times \text{Obj}(\mathscr{C}), \\
        \text{Hom}_{\mathscr{C}'\times\mathscr{C}}\big( (X,W), (Y,Z) \big) &:= \text{Hom}_{\mathscr{C}'}(X,Y)\times\text{Hom}_\mathscr{C}(W,Z) \quad \forall \ (X,W),(Y,Z)\in \text{Obj}(\mathscr{C}'\times\mathscr{C}),
    \end{align*}
    con la composición de morfismos definida entrada a entrada.
\end{Def}

\begin{Def}\label{Def: Bifuntor}
    Un \emph{bifuntor} es un funtor cuyo dominio es una categoría producto.
\end{Def}

%\begin{Obs}\label{Observaciones sobre bifuntores}\footnote{Esta observación pretende más adelante ``unir'' al funtor Hom-covariante con el Hom-contravariante, formando algo que no es un bifuntor, y ''corregirlo'' componiéndolo con el producto de un funtor de dualidad y un funtor identidad, que tampoco es un bifuntor. Tal vez valga la pena desecharlo y sólo definir al bifuntor Hom y mostrar que es un bifuntor.}
%    Sean $F:\mathscr{B}'\to \mathscr{B}, G:\mathscr{C}'\to \mathscr{C}$ funtores y $H:\mathscr{B}\times\mathscr{C}\to \mathscr{D}$ un bifuntor. 
%
%    \begin{enumerate}[label=(\arabic*)]
%
%        \item Si $F$ y $G$ tienen la misma varianza, entonces $F\times G:\mathscr{B}'\times\mathscr{C}'\to \mathscr{B}\times \mathscr{C}$ es un bifuntor de la misma varianza, definido entrada a entrada.
%
%        \item Si $F$ y $G$ tienen la misma varianza que $H$, entonces las composiciones $H\circ(F\times 1_\mathscr{C}):\mathscr{B}'\times\mathscr{C}\to \mathscr{D}$ y $H\circ(1_\mathscr{B}\times G):\mathscr{B}\times\mathscr{C}'\to \mathscr{D}$ son bifuntores.
%    \end{enumerate}
%\end{Obs}

\begin{Ejem}\label{Ejem: Bifuntor}
    Sea $\mathscr{C}$ una categoría localmente pequeña. Entonces, tenemos\footnote{Ver la Definición \ref{Def: Categoría opuesta} al inicio de la siguiente sección.} el bifuntor Hom
    \[
        \text{Hom}(-,-):\mathscr{C}^\text{op}\times\mathscr{C}\to \text{Sets}, \ \big( X\xrightarrow[]{f^\text{op}}W, Y\xrightarrow[]{g}Z \big) \mapsto \big( \text{Hom}_\mathscr{C}(X,Y)\xrightarrow[]{\text{Hom}(f^\text{op},g)} \text{Hom}_\mathscr{C}(W,Z) \big),
    \] 
    donde $\text{Hom}(f^\text{op},g)(\alpha) := g\circ\alpha\circ f$ para cualquier $\alpha\in\text{Hom}_\mathscr{C}(X,Y)$. En particular, para cualquier $A\in\text{Obj}(\mathscr{C})$, tenemos que $\text{Hom}(1_A{}^\text{op},-):\mathscr{C}\to \text{Sets}$ y $\text{Hom}(-,1_A):\mathscr{C}^\text{op}\to \text{Sets}$ son funtores; el primero coincide con el funtor Hom-covariante $\text{Hom}_\mathscr{C}(A,-):\mathscr{C}\to \text{Sets}$ introducido en el Ejemplo \ref{Ejem: Funtores}.
\end{Ejem}

\section{Dualidad} \label{Sec: Dualidad}

\subsection*{Categoría opuesta} \label{Ssec: Categoría opuesta}

\begin{Def} \label{Def: Categoría opuesta}
    
    Sea $\mathscr{C}$ una categoría. La \emph{categoría opuesta} $\mathscr{C}^{\text{op}}$ de $\mathscr{C}$ se define como sigue:
    
    \begin{itemize}
        \item[$\bullet$] $\text{Obj}(\mathscr{C}^{\text{op}}):=\text{Obj}(\mathscr{C})$.
    
        \item[$\bullet$] $\text{Hom}_{\mathscr{C}^{\text{op}}}(B,A):=\text{Hom}_\mathscr{C}(A,B)$.
    
        \item[$\bullet$] La composición $\circ^{\text{op}}:\text{Mor}(\mathscr{C}^{\text{op}})\times\text{Mor}(\mathscr{C}^{\text{op}})\to \text{Mor}(\mathscr{C}^{\text{op}})$ es ``la opuesta'' a la composición $\circ:\text{Mor}(\mathscr{C})\times\text{Mor}(\mathscr{C})\to \text{Mor}(\mathscr{C})$, es decir, $f\circ^{\text{op}} g:= g\circ f$.
    \end{itemize}
\end{Def}
    
\begin{Nota}
    Sea $f:A\to B$ un morfismo en $\mathscr{C}$. Para distinguirlo del mismo morfismo $f\in\mathscr{C}^\text{op}$, es conveniente denotar a este último por $f^{\text{op}}:B\to A$. Con dicha notación, la composición de
    \[
        C\xrightarrow[]{g^{\text{op}}}B\xrightarrow[]{f^{\text{op}}}A
    \] 
    en $\mathscr{C}^{\text{op}}$ se escribe como sigue:
    \[
        f^{\text{op}}\circ^{\text{op}}g^{\text{op}} := (g\circ f)^{\text{op}}.
    \] 
    Para simplificar aún más lo anterior, escribimos
    \[
        f^{\text{op}}g^{\text{op}} = (gf)^{\text{op}}.
    \] 
\end{Nota}

\begin{Obs}\label{Obs: Categoría opuesta}
    Sean $\mathscr{C}$ una categoría y $\mathscr{C}^\text{op}$ su categoría opuesta.

    \begin{enumerate}[label=(\arabic*)]
    
        \item Existe un funtor contravariante dado por la correspondencia
            \begin{align*}
                D_\mathscr{C}:\mathscr{C}&\to \mathscr{C}^\text{op}, \\
                X &\mapsto X, \\
                A\xrightarrow[]{f} B &\mapsto B\xrightarrow[]{f^\text{op}} A,
            \end{align*}
                
        conocido como \emph{funtor de dualidad}.

    \item Note que $(\mathscr{C}^{\text{op}})^{\text{op}} = \mathscr{C}$. Por lo tanto, las composiciones de los funtores $D_\mathscr{C}:\mathscr{C}\to \mathscr{C}^\text{op}$ y $D_{\mathscr{C}^\text{op}}:\mathscr{C}^\text{op}\to \mathscr{A}$ satisfacen
        \[
            D_{\mathscr{C}^\text{op}} D_\mathscr{C} = 1_\mathscr{C} \quad \text{y} \quad D_\mathscr{C} D_{\mathscr{C}^\text{op}} = 1_{\mathscr{C}^\text{op}}.
        \] 
        Es decir, $D_\mathscr{C}:\mathscr{C}\to \mathscr{C}^\text{op}$ es un anti-isomorfismo de categorías.

    \item Sea $F:\mathscr{C}\to \mathscr{D}$ un funtor de cualquier varianza. Entonces, las composiciones de funtores
        \begin{align*}
            F_\text{op}&:=F\circ D_{\mathscr{C}^\text{op}}:\mathscr{C}^\text{op}\to \mathscr{D}, \\
            F^\text{op}&:=D_\mathscr{D}\circ F:\mathscr{C}\to \mathscr{D}^\text{op},
        \end{align*}
    \end{enumerate}
    son funtores de varianza opuesta a $F$. Notamos que $(F_\text{op})^\text{op} = (F^\text{op})_\text{op} =: F_\text{op}^\text{op}:\mathscr{C}^\text{op}\to \mathscr{D}^\text{op}$ es un funtor con la misma varianza que $F$.
\end{Obs}

\subsection*{Principio de dualidad} \label{Ssec: Principio de dualidad}

\begin{Def}\label{Def: Proposición categórica}
    Una \emph{proposición categórica} es una afirmación que involucra conceptos categóricos, tales como morfismos, funtores, diagramas, etcétera, y es una proposición en el sentido lógico; es decir, se puede determinar sin ambigüedad si es exclusivamente verdadera o falsa.
\end{Def}

%\begin{Ejem}\label{Relaciones de dualidad}
%
%    Sean $\mathscr{C}$ una categoría y $A\xrightarrow[]{f} B$ un morfismo en $\mathscr{C}$. Consideremos el morfismo opuesto $B\xrightarrow{f^{\text{op}}} A$ en $\mathscr{C}^{op}$. De \ref{Def: Morfismos especiales} podemos ver que: 
%    \begin{enumerate}[label=(\arabic*)]
%    
%        \item $f$ es un monomorfismo en $\mathscr{C}$ si, y sólo si, $f^{\text{op}}$ es un epimorfismo en $\mathscr{C}^{\text{op}}$;
%
%        \item $f$ es un monomorfismo escindible en $\mathscr{C}$ si, y sólo si, $f^{\text{op}}$ es un epimorfismo escindible en $\mathscr{C}^{\text{op}}$;
%
%        \item $f$ es un isomorfismo en $\mathscr{C}$ si, y sólo si, $f^{\text{op}}$ es un isomorfismo en $\mathscr{C}^{\text{op}}$.
%    \end{enumerate}
%
%    \noindent Para hacer referencia a las proposiciones categóricas duales de (1), informalmente decimos que ``monomorfismo'' y ``epimorfismo'' son ``nociones categóricas duales''. Análogamente, por (2), decimos que las nociones de ``monomorfismo escindible'' y ``epimorfismo escindible'' son duales. Más aún, de (3) se sigue que la noción de isomorfismo es dual a sí misma, o ``auto dual''. 
%\end{Ejem}

%En general, podemos intentar encontrar nociones categóricas duales siguiendo un procedimiento que primero ilustraremos para uno de los ejemplos anteriores y luego enunciaremos. Sea $\mathscr{C}$ una categoría. Supongamos que $B\xrightarrow{f^\text{op}} A$ es un epimorfismo en $\mathscr{C}^\text{op}$. Entonces $f^{\text{op}}$ es tal que, para cualesquiera $g^{\text{op}},h^{\text{op}}\in\text{Mor}(\mathscr{C}^{\text{op}})$, se tiene que
%    \[
%    g^{\text{op}} f^{\text{op}}= h^{\text{op}} f^{\text{op}}\implies g^{\text{op}}=h^{\text{op}}.
%    \] 
%    Aplicando el funtor de dualidad $D_{\mathscr{C}^\text{op}}:\mathscr{C}^\text{op}\to \mathscr{C}$ ``a la proposición categórica anterior'', tenemos que
%    \[
%    fg = fh \implies g = h,
%    \] 
%    es decir, que $f$ es un monomorfismo. Por lo tanto, la noción dual de ``epimorfismo'' es ``monomorfismo''.

%    Sean $\mathscr{C}$ una categoría y $P$ una proposición categórica válida en $\mathscr{C}$. Para encontrar la proposición categórica dual de $P$ hay que hacer lo siguiente:
%
%\begin{enumerate}[label=(\arabic*)]
%    
%    \item Escribir la proposición $P$ en $\mathscr{C}^\text{op}$. Denotaremos a la proposición categórica resultante por $P'$.
%
%    \item Aplicar el funtor de dualidad $D_{\mathscr{C}^\text{op}}:\mathscr{C}^\text{op}\to \mathscr{C}$ a la proposición\footnote{Recordemos que una proposición lógica puede estar compuesta de otras proposiciones lógicas unidas por operadores lógicos. En particular, una proposición categórica en una categoría $\mathscr{C}$ puede estar compuesta de otras proposiciones categóricas en $\mathscr{C}$ más ``elementales''. En este sentido, si $P$ es una proposición categórica en $\mathscr{C}$, entonces entenderemos el ``aplicar el funtor de dualidad $D_\mathscr{C}$ a $P$'' como aplicarlo a todas las proposiciones ``elementales'' de $P$ donde tenga sentido hacerlo, sin modificar los operadores lógicos que las unen.} $P'$. 
%\end{enumerate}
%
%\noindent Luego, la proposición categórica $Q:= D_{\mathscr{C}^\text{op}}(P')$ es la dual a $P$.

%\noindent En otras palabras, lo que hacemos es enunciar una proposición en la categoría opuesta, pasarla a la categoría original e interpretarla ahí, y observar relación de dualidad que surge entre las nociones correspondientes. Notamos que, para poder hacer el procedimiento anterior, es crucial que la proposición $Q$ que buscamos dualizar tenga sentido tanto en una categoría $\mathscr{C}$ como en su categoría opuesta $\mathscr{C}^\text{op}$. 

Recordemos que una proposición lógica puede estar compuesta de otras proposiciones lógicas unidas por operadores lógicos. En particular, una proposición categórica en una categoría $\mathscr{C}$ puede estar compuesta de otras proposiciones categóricas en $\mathscr{C}$ más ``elementales''. En este sentido, si $P$ es una proposición categórica en $\mathscr{C}$, entonces entenderemos el ``aplicar el funtor de dualidad $D_\mathscr{C}$ a $P$'' como aplicarlo a todas las proposiciones ``elementales'' de $P$ donde tenga sentido hacerlo, sin modificar los operadores lógicos que las unen, y denotaremos a la proposición categórica resultante, cometiendo un abuso de notación, por $D_{\mathscr{C}}(P)$. Al procedimiento para obtener $D_\mathscr{C}(P)$ se le conoce informalmente como ``darle vuelta a las flechas''.

\begin{Def}
    Sean $\mathscr{C}$ una categoría y $P$ una proposición categórica en $\mathscr{C}$. 

    \begin{enumerate}[label=(\alph*)]
    
        \item La proposición categórica $\overline{P}$ es aquella que se obtiene al escribir la proposición $P$ en términos de conceptos (morfismos, funtores, diagramas, etcétera) en la categoría $\mathscr{C}^\text{op}$.

        \item La \emph{proposición dual} de $P$ se define como $P^\ast:=D_{\mathscr{C}^\text{op}}( \overline{P})$. 

        %\item $P$ es \emph{dualizable en} $\mathscr{C}$ si $P^\ast$ también se verifica en $\mathscr{C}$.

        \item $P$ es \emph{auto dual} si $P^\ast = P$.
    \end{enumerate}
\end{Def}

\begin{Obs}
    Sean $\mathscr{C}$ una categoría y $P$ una proposición categórica en $\mathscr{C}$. Entonces, $(P^\ast)^\ast = P$.
    \vspace{1mm}

    En efecto: Se sigue de notar que
    \[
        \overline{(D_{\mathscr{C}^\text{op}}(\overline{P}))} \text{ en } \mathscr{C}^\text{op} \iff D_\mathscr{C}(P) \text{ en } \mathscr{C}^\text{op},
    \]
    y luego aplicar $D_{\mathscr{C}^\text{op}}$ a la proposición categórica bicondicional anterior.
\end{Obs}

\begin{Ejem}
    Sean $\mathscr{C}$ una categoría y $P$ la proposición categórica ``$f$ es un monomorfismo'', con $f\in\text{Mor}(\mathscr{C})$. Entonces, $ \overline{P}$ es la proposición categórica ``$f^\text{op}$ es un monomorfismo'', con $f^\text{op}\in\text{Mor}(\mathscr{C}^\text{op})$. Para obtener la proposición dual $P^\ast$, partimos de $ \overline{P}$, que podemos reescribir como
    \[
    f^\text{op}g^\text{op} = f^\text{op}h^\text{op} \implies g^\text{op} = h^\text{op}.
    \] 
    Aplicando el funtor $D_{\mathscr{C}^\text{op}}$ a la proposición anterior, obtenemos
    \[
    gf = hf \implies g=h,
    \] 
    lo que equivale a la proposición categórica ``$f$ es un epimorfismo'', con $f\in\text{Mor}(\mathscr{C})$. Por lo tanto, ``$f$ es un epimorfismo'' es la proposición dual de ``$f$ es un monomorfismo''. Más aún, análogamente, se puede ver que ``$f$ es un monomorfismo'' es la proposición dual de ``$f$ es un epimorfismo'', verificándose $(P^\ast)^\ast = P$.
\end{Ejem}

El hecho de que una proposición categórica $P$ se verifique en una categoría $\mathscr{C}$ no implica que $P^\ast$ se verifique en $\mathscr{C}$. Esto se debe a que algunas proposiciones categóricas pueden involucrar conceptos que, para ser definidos, requieran que en una categoría se cumplan condiciones adicionales que \emph{a priori} no necesariamente deben cumplirse en su categoría opuesta. Esto significa que la dualizabilidad de proposiciones categóricas depende del contexto en que se esté trabajando. Para hacer más precisa esta idea, introducimos los siguientes conceptos.

\begin{Def}\label{Def: Universos}\leavevmode
    
    \begin{enumerate}[label=(\alph*)]
    
        \item Un \emph{universo} $\mathfrak{U}$ es una colección de categorías.

        \item Un universo $\mathfrak{U}$ es \emph{dualizante} si $\mathscr{C}^\text{op}\in \mathfrak{U}$ para toda $\mathscr{C}\in \mathfrak{U}$.
    \end{enumerate}
\end{Def}

\begin{Ejem}\label{Ejem: Universos}\leavevmode
    \begin{enumerate}[label=(\arabic*)]
    
        \item El universo de todas las categorías es dualizante.

        \item El universo de las categorías de módulos sobre un anillo no es dualizante\footnote{Se puede demostrar que, para una familia arbitraria de módulos no triviales en una categoría de módulos sobre un anillo, el morfismo canónico entre su coproducto y su producto es un monomorfismo mas no un epimorfismo. Por ende, si $R\neq0$, suponer que $\text{Mod}(R)^\text{op}$ es equivalente a una categoría de módulos sobre un anillo lleva a una contradicción.}.
    \end{enumerate}
\end{Ejem}

%\begin{Obs}
%    Toda proposición categórica válida en las categorías de un universo dualizante es dualizable. 
%    \vspace{1mm}
%
%    En efecto: Sean $U$ un universo dualizable, $\mathscr{C}$ una categoría en $U$ y $P$ una afirmación categórica. Como $P$ es válida en todas las categorías en $U$, en particular, $P$ es válida en $\mathscr{C}^\text{op}$.
%\end{Obs}

%    Sean $\mathscr{C}$ una categoría y $P$ una proposición válida en $\mathscr{C}$ y dualizable. El principio de dualidad consiste en aplicar el procedimiento descrito anteriormente para obtener una proposición $P^\text{op}$ válida en $\mathscr{C}^\text{op}$, y puede ser aplicado a definiciones, axiomas, teoremas, propiedades universales, etcétera, por lo que puede ser de gran utilidad si se utiliza con el debido cuidado. En particular, si la categoría $\mathscr{C}$ es auto dual, entonces se sigue que la proposición dual $P^\text{op}$ también es válida en $\mathscr{C}$. \\
%
%    Por ejemplo, si una proposición P es válida en una categoría $\mathscr{C}$ que cumple ciertos axiomas adicionales $\{A_i\}_{i\in I}$, con $I$ finito, entonces se sigue que la proposición P\textsuperscript{op} es válida en cualquier categoría $\mathscr{D}$ que cumpla los axiomas $\{A_i^\text{op}\}_{i\in I}$. En particular, si $A_i^\text{op}=A_i$ para cada $i\in I$, entonces P\textsuperscript{op} es válida en $\mathscr{C}$.

\begin{Def}
    Sea $\mathfrak{U}$ un universo dualizante. Una proposición categórica $P$ es \emph{dualizable} en $\mathfrak{U}$ si se verifica en todas las categorías de $\mathfrak{U}$.
\end{Def}

\begin{Obs}[Principio de dualidad]
    Sea $\mathfrak{U}$ un universo dualizante. Entonces, para toda proposición categórica $P$ dualizable en $\mathfrak{U}$, se tiene que $P^\ast$ es dualizable en $\mathfrak{U}$.
    \vspace{1mm}

    En efecto: Sean $\mathscr{C}\in \mathfrak{U}$ y $P$ una proposición categórica dualizable en $\mathfrak{U}$. Entonces, $P$ se verifica en $\mathscr{C}$ y, en particular, $\overline{P}$ se verifica en $\mathscr{C}^\text{op}$. Aplicando el funtor de dualidad $D_{\mathscr{C}^\text{op}}$ a $\overline{P}$, obtenemos que $P^\ast$ se verifica en $\mathscr{C}$.
\end{Obs}

El principio de dualidad nos dice que, para cualquier resultado que se pueda demostrar en las categorías de un universo dualizante, se sigue que existe un resultado dual válido para las categorías del mismo universo. Este principio reduce algunas demostraciones a observar que ya se ha demostrado un resultado dual en una categoría que pertenece a un universo dualizante; sin embargo, sólo le da validez a la demostración en categorías del mismo universo. \\

\begin{Prop}\label{Prop: Universo dualizante}
    Sea $\mathfrak{U}$ un universo dado por las categorías que cumplen ciertos axiomas adicionales $\{A_i\}_{i\in I}$, con $I$ finito. Entonces, si $\{A_i\}_{i\in I} = \{A_j{}^\ast\}_{j\in I}$, el universo $\mathfrak{U}$ es dualizante.
\end{Prop}

\begin{proof}
    Supongamos que $\{A_i\}_{i\in I} = \{A_j{}^\ast\}_{j\in I}$. Sea $\mathscr{C}\in \mathfrak{U}$. Entonces, los axiomas $\{A_j{}^\ast\}_{j\in I}$ se cumplen en $\mathscr{C}$, por lo que $\mathscr{C}^\text{op}$ verifica los axiomas
    \begin{align*}
        \{D_\mathscr{C}(A_j{}^\ast)\}_{j\in I} &= \{D_{\mathscr{C}}(D_{\mathscr{C}^\text{op}}(\overline{A_j}))\} \\
                                               &= \{\overline{A_j}\}_{j\in I},
    \end{align*}
    lo que implica que $\mathscr{C}^\text{op}\in\mathfrak{U}$.
\end{proof}

Más adelante, utilizaremos la caracterización de los universos dualizantes dada por la Proposición \ref{Prop: Universo dualizante} para observar que los universos formados, respectivamente, por las categorías aditivas, exactas, trianguladas, extrianguladas y abelianas son dualizantes. Esto nos permitirá argumentar en dichos contextos utilizando el principio de dualidad.

\subsection*{Subobjetos y objetos cociente} \label{Ssec: Subobjetos y objetos cociente}

\begin{Def}\label{Def: Subobjeto}
    Sean $\mathscr{C}$ una categoría y $X\xhookrightarrow{\alpha} A$ un monomorfismo en $\mathscr{C}$. Entonces, decimos que $X$ es un \emph{subobjeto} de $A$, vía $\alpha$, y que $\alpha$ es una inclusión de $X$ en $A$; denotamos esta relación por $X\subseteq A$. Si $f:A\to B$ es un morfismo en $\mathscr{C}$, decimos que la composición $f\mid_X := f\alpha:X\to B$ es la \emph{restricción} de $f$ en $X$ vía $\alpha$.
\end{Def}

\begin{Def}\label{Def: Categoría de subobjetos}
    Sean $\mathscr{C}$ una categoría y $A\in\text{Obj}(\mathscr{C})$. La \emph{categoría de subobjetos} de $A$, denotada por $\text{Mon}_\mathscr{C}(-,A)$, se define como sigue:

    \begin{itemize}
    
        \item[$\bullet$] $\text{Obj}(\text{Mon}_\mathscr{C}(-,A)) := \{\alpha\in\text{Mor}(\mathscr{C}) \mid \alpha \text{ es un monomorfismo y } \text{Codom}(\alpha)=A\}$.

        \item[$\bullet$] Dados $X\xhookrightarrow{\alpha}A$ y $Y\xhookrightarrow{\beta}A$, se define
            \[
                \text{Mon}_\mathscr{C}(-,A)(\alpha,\beta) := \{h\in \text{Hom}_\mathscr{C}(X,Y) \mid \beta h=\alpha\}.
            \] 
            Es decir, $\alpha\xrightarrow[]{h}\beta\in\text{Mon}_\mathscr{C}(-,A)(\alpha,\beta)$ si $h\in\text{Hom}_\mathscr{C}(X,Y)$ y hace conmutar el siguiente diagrama
            \begin{center}
                \begin{tikzcd}
                    X \arrow[]{dd}[swap]{h} \arrow[hook]{dr}[]{\alpha} \\
                    &A. \\
                    Y \arrow[hook]{ur}[swap]{\beta}
                \end{tikzcd}
            \end{center}

        \item[$\bullet$] La composición en $\text{Mon}_\mathscr{C}(-,A)$ es la inducida de $\mathscr{C}$.
    \end{itemize}
\end{Def}

\begin{Obs}\label{Mendoza-1.1.2}

    Sean $\mathscr{C}$ una categoría y $A\in\text{Obj}(\mathscr{C})$.

    \begin{enumerate}[label=(\arabic*)]
    
        \item Para cualesquiera $\alpha,\beta\in\text{Mon}_\mathscr{C}(-,A)$, la clase $\text{Mon}_\mathscr{C}(-,A)(\alpha,\beta)$ tiene a lo sumo un elemento y, en caso de que exista, es un monomorfismo en $\mathscr{C}$.

        \item $\text{Mon}_\mathscr{C}(-,A)(\alpha,\beta)\neq\varnothing$ y $\text{Mon}_\mathscr{C}(-,A)(\beta,\alpha)\neq\varnothing$ si, y sólo si, $\alpha\simeq\beta$ en $\text{Mon}_\mathscr{C}(-,A)$.

            En efecto: Sean $\alpha\xrightarrow[]{h} \beta, \beta\xrightarrow[]{j} \alpha$ en $\text{Mon}_\mathscr{C}(-,A)$. Dado que $\beta\xrightarrow[]{hj} \beta$ y $1_\beta=1_{\text{Dom}(\beta)}$, por (1), se tiene que $hj=1_\beta$. Análogamente, se ve que $jh=1_\alpha$.

        \item La relación $\leq$ en la clase $\text{Obj}(\text{Mon}_\mathscr{C}(-,A))$ dada por
            \[
                \alpha\leq\beta \iff \text{Mon}_\mathscr{C}(-,A)(\alpha,\beta)\neq\varnothing
            \] 
            es una relación de preorden en $\text{Obj}(\text{Mon}_\mathscr{C}(-,A))$.

        \item Sea $\overline{\text{Mon}}_\mathscr{C}(-,A):=\text{Obj}(\text{Mon}_\mathscr{C}(-,A))/\simeq$. Dado que por (2) tenemos que $\alpha\leq\beta$ y $\beta\leq\alpha$ si, y sólo si, $\alpha\simeq\beta$, entonces el preorden $\leq$ en $\text{Obj}(\text{Mon}_\mathscr{C}(-,A))$ induce un orden parcial en $\overline{\text{Mon}}_\mathscr{C}(-,A)$ dado por
            \[
                [x]\leq[y] \iff x\leq y.
            \] 
    \end{enumerate}
\end{Obs}

%Dado que la definición de subobjeto depende únicamente de la noción de monomorfismo, la cual es dualizable en cualquier categoría y tiene a ``epimorfismo'' como noción dual, procederemos a definir la noción dual de subobjeto, así como la categoría correspondiente a esta nueva noción.

\begin{Def}\label{Def: Objeto cociente}
    Sean $\mathscr{C}$ una categoría y $A\xtwoheadrightarrow{\beta} X$ un epimorfismo en $\mathscr{C}$. Entonces, decimos que $X$ es un \emph{objeto cociente} de $A$ vía $\beta$.
\end{Def}

\begin{Def}\label{Def: Categoría de objetos cociente}
    Sean $\mathscr{C}$ una categoría y $A\in\text{Obj}(\mathscr{C})$. La \emph{categoría de objetos cociente} de $A$, denotada por $\text{Epi}_\mathscr{C}(A,-)$, se define como sigue:

    \begin{itemize}
    
        \item[$\bullet$] $\text{Obj}(\text{Epi}_\mathscr{C}(A,-)) := \{\alpha\in\text{Mor}(\mathscr{C}) \mid \alpha \text{ es un epimorfismo y } \text{Dom}(\alpha) = A\}$.

        \item[$\bullet$] Dados \begin{tikzcd} A \arrow[two heads]{r}[]{\alpha} &X \end{tikzcd} y \begin{tikzcd} A \arrow[two heads]{r}[]{\beta} &Y \end{tikzcd}, se define
            \[
                \text{Epi}_\mathscr{C}(A,-)(\alpha,\beta) := \{h\in\text{Hom}_\mathscr{C}(X,Y) \mid h\alpha=\beta \}.
            \] 
            Es decir, $\alpha\xrightarrow[]{h}\beta\in\text{Epi}_\mathscr{C}(A,-)(\alpha,\beta)$ si $h\in\text{Hom}_\mathscr{C}(X,Y)$ y hace conmutar el siguiente diagrama
            \begin{center}
                \begin{tikzcd}
                    &X \arrow[]{dd}[]{h} \\
                    A \arrow[two heads]{ur}[]{\alpha} \arrow[two heads]{dr}[swap]{\beta} \\
                    &Y.
                \end{tikzcd}
            \end{center}
            
        \item[$\bullet$] La composición en $\text{Epi}_\mathscr{C}(A,-)$ es la inducida de $\mathscr{C}$.
    \end{itemize}
\end{Def}

\begin{Obs}\label{Observaciones sobre la categoría de objetos cociente}
    
    Sean $\mathscr{C}$ una categoría y $A\in\text{Obj}(\mathscr{C})$.

    \begin{enumerate}[label=(\arabic*)]

        \item Para cualesquiera $\alpha,\beta\in\text{Epi}_\mathscr{C}(A,-)$, la clase $\text{Epi}_\mathscr{C}(A,-)(\alpha,\beta)$ tiene a lo sumo un elemento y, en caso de que exista, es un epimorfismo en $\mathscr{C}$.

        \item $\text{Epi}_\mathscr{C}(A,-)(\alpha,\beta)\neq\varnothing$ y $\text{Epi}_\mathscr{C}(A,-)(\beta,\alpha)\neq\varnothing$ si, y sólo si, $\alpha\simeq\beta$ en $\text{Epi}_\mathscr{C}(A,-)$.

        \item La relación $\leq$ en la clase $\text{Obj}(\text{Epi}_\mathscr{C}(A,-))$ dada por
            \[
                \alpha\leq\beta \iff \text{Epi}_\mathscr{C}(A,-)(\beta,\alpha)\neq\varnothing
            \] 
            es una relación de preorden en $\text{Obj}(\text{Epi}_\mathscr{C}(A,-))$. Notemos que, en el diagrama conmutativo
            \begin{center}
                \begin{tikzcd}
                    &X \\
                    A \arrow[two heads]{ur}[]{\alpha} \arrow[two heads]{dr}[swap]{\beta} \\
                    &Y \arrow[]{uu}[swap]{h},
                \end{tikzcd}
                \hspace{3mm} i.e., $\alpha\le\beta$,
            \end{center}
            se tiene que $h$ es un epimorfismo (i.e., que $Y$ ``cubre'' a $X$, vía $h$), y en este sentido es que interpretamos la desigualdad $\alpha\le\beta$.

        \item Sea $\overline{\text{Epi}}_\mathscr{C}(A,-):=\text{Obj}(\text{Epi}_\mathscr{C}(A,-))/\simeq$. Dado que por (2) tenemos que $\alpha\leq\beta$ y $\beta\leq\alpha$ si, y sólo si, $\alpha\simeq\beta$, entonces el preorden $\leq$ en $\text{Obj}(\text{Epi}_\mathscr{C}(-,A))$ induce un orden parcial en $\overline{\text{Epi}}_\mathscr{C}(-,A)$ dado por
            \[
                [x]\leq[y] \iff x\leq y.
            \] 

        \item Las categorías $\text{Mon}_\mathscr{C}(-,A)$ y $\text{Epi}_{\mathscr{C}^\text{op}}(A,-)$ son anti-isomorfas.

    \end{enumerate}
\end{Obs}

\section{Límites y colímites} \label{Sec: Límites y colímites}

Muchas de las construcciones categóricas que estudiaremos posteriormente como los objetos finales, igualadores, productos fibrados y productos, son ejemplos de límites. Así mismo, los objetos iniciales, coigualadores, sumas fibradas y coproductos, que son construcciones duales a las mencionadas anteriormente, son ejemplos de colímites. Tanto los límites como los colímites tienen la bondad de ser únicos hasta isomorfismos en una categoría apropiada. Por lo tanto, en vez de demostrar dicha unicidad hasta isomorfismos para cada una de las construcciones mencionadas anteriormente, demostraremos que esto es válido en general para límites y colímites, y más adelante nos limitaremos a hacer observaciones cuando una construcción sea un límite o un colímite. \\

Las definiciones de límite y colímite son muy abstractas, por lo que se recomienda a quienes no conozcan ya ejemplos de ellos que empiecen esta sección por leer los que aparecen a continuación, mismos que fueron mencionados en el párrafo anterior, y que vuelvan a estas definiciones cuantas veces sea necesario para ir construyendo un panorama general.

\begin{Def} \label{Def: Cono} 
    Sea $F:\mathscr{C}\to \mathscr{D}$ un funtor. Un \emph{cono} de $F$ es un par $(D,\psi)$, donde $D\in\text{Obj}(\mathscr{D})$ y $\psi$ es una familia $\{D\xrightarrow[]{\psi_X} F(X)\}_{X\in\text{Obj}(\mathscr{C})}$ de morfismos en $\mathscr{D}$ tales que, para cualesquiera $X,Y\in\text{Obj}(\mathscr{C})$ y $X\xrightarrow[]{f} Y\in\text{Mor}(\mathscr{C})$, se tiene que $F(f)\psi_X=\psi_Y$; diagramáticamente, el cono $(D,\psi)$ se puede ver como el siguiente diagrama conmutativo
    \begin{equation}\label{eq: Cono}
        \begin{tikzcd}
            &D \arrow[]{dl}[swap]{\psi_X} \arrow[]{dr}[]{\psi_Y} &&&&&\empty{} \arrow[phantom]{d}[]{\forall \ X,Y\in\text{Obj}(\mathscr{C}), X\xrightarrow[]{f} Y\in\text{Mor}(\mathscr{C}).} \\
            F(X) \arrow[]{rr}[swap]{F(f)} &&F(Y) &&&&\empty{}
        \end{tikzcd}
    \end{equation}
\end{Def}

\begin{Obs}\label{Obs: Cono}
    El término ``cono'' es una mnemotecnia que hace referencia a los diagramas conmutativos (\ref{eq: Cono}) que representan su definición, pues podemos imaginar que cada uno de dichos diagramas es una rebanada de un cono que tiene en el borde de la base a todos los objetos de la categoría $F(\mathscr{C})$.
\end{Obs}

\begin{Def}\label{Def: Morfismo de conos}
    Sean $F:\mathscr{C}\to \mathscr{D}$ un funtor y $(D,\psi), (D',\psi')$ conos de $F$. Un \emph{morfismo de conos} $\alpha:(D,\psi)\to (D',\psi')$ es un morfismo $D\xrightarrow[]{\alpha} D'$ en $\mathscr{D}$ que hace conmutar el siguiente diagrama
    \begin{center}
        \begin{tikzcd}
            D \arrow[]{dr}[swap]{\psi_X} \arrow[]{rr}[]{\alpha} &&D' \arrow[]{dl}[]{\psi'_X} \\
                                                                &F(X)
        \end{tikzcd}
        $\forall \ X\in\text{Obj}(\mathscr{C})$.
    \end{center}
\end{Def}

\begin{Obs}\label{Obs: Morfismo de conos}
    Sea $F:\mathscr{C}\to \mathscr{D}$ un funtor. Las clases de conos de $F$ y de morfismos de conos de $F$, junto con la composición de morfismos inducida de $\mathscr{D}$, forman una categoría llamada la \emph{categoría de conos de} $F$, que denotamos por $\text{Cone}(F)$. %En particular, el morfismo identidad de un cono $(D,\psi)$ en la categoría de conos de $F$ es el morfismo identidad $1_D\in\text{Mor}(\mathscr{D})$. Esto es, $1_{(D,\psi)}=1_D$.
\end{Obs}

\begin{Def} \label{Def: límite}
    Sea $F:\mathscr{C}\to \mathscr{D}$ un funtor. Un \emph{cono universal} de $F$ es un cono de $F$ a través del cual se factorizan todos los conos de $F$ de forma única. Es decir, un cono $(U,\phi)$ de $F$ es un cono universal de $F$ si para cualquier cono $(D,\psi)$ de $F$ existe un único morfismo de conos $(D,\psi)\xrightarrow[]{\varphi} (U,\phi)$. A un cono universal de $F$ también se le conoce como un \emph{límite} del funtor $F$, y la propiedad descrita anteriormente se conoce como la \emph{propiedad universal del límite del funtor} $F$.
\end{Def}

\begin{Obs} \label{Unicidad de límites hasta isomorfismos}
    Sea $F:\mathscr{C}\to \mathscr{D}$ un funtor. Entonces los límites de $F$, si existen, son únicos hasta isomorfismos en la categoría de conos de $F$.

    \vspace{1mm}
    En efecto: Sean $(U,\phi)$ y $(U',\phi')$ límites de $F$. Dado que son conos universales y, en particular, conos de $F$, por la propiedad universal del límite de $F$ existen morfismos únicos $U'\xrightarrow[]{\varphi} U, U\xrightarrow[]{\varphi'} U'$ en $\mathscr{D}$ tales que, para cualquier $X\in\text{Obj}(\mathscr{C})$, el siguiente diagrama conmuta
    \begin{equation}\label{eq: Unicidad de límites}
        \begin{tikzcd}
            U' \arrow[dotted, shift left]{rr}[]{\varphi} \arrow[]{dr}[swap]{\phi'_X} &&U. \arrow[]{dl}[]{\phi_X} \arrow[dotted, shift left]{ll}[]{\varphi'} \\
                                                                         &F(X)
        \end{tikzcd}
    \end{equation}
    Ahora, del diagrama conmutativo (\ref{eq: Unicidad de límites}) tenemos que los siguientes diagramas conmutan para cualquier $X\in\text{Obj}(\mathscr{C})$,
    \begin{center}
        \begin{tikzcd}
            U \arrow[]{rr}[]{\varphi\varphi'} \arrow[]{dr}[swap]{\phi_X} &&U \arrow[]{dl}[]{\phi_X} &&U' \arrow[]{dr}[swap]{\phi'_X} \arrow[]{rr}[]{\varphi'\varphi} &&U' \arrow[]{dl}[]{\phi'_X} \\
                                              &F(X) &&&&F(X)
        \end{tikzcd}
    \end{center}
    Por otro lado, los siguientes diagramas conmutan para cualquier $X\in\text{Obj}(\mathscr{C})$,
    \begin{center}
        \begin{tikzcd}
            U \arrow[]{rr}[]{1_U} \arrow[]{dr}[swap]{\phi_X} &&U \arrow[]{dl}[]{\phi_X} &&U' \arrow[]{dr}[swap]{\phi'_X} \arrow[]{rr}[]{1_{U'}} &&U' \arrow[]{dl}[]{\phi'_X} \\
                                              &F(X) &&&&F(X)
        \end{tikzcd}
    \end{center}
    Por ende, de la propiedad universal del límite de $F$ se sigue que $\varphi\varphi'=1_U$ y $\varphi'\varphi=1_{U'}$. Por lo tanto, concluimos que los límites de $F$ son únicos hasta isomorfismos en la categoría de conos de $F$.
\end{Obs}

%Dado que la definición de límite depende de la noción de cono, la cual es dualizable en cualquier categoría, procederemos a definir la noción dual del cono, llamada cocono, así como la noción dual del límite, llamada colímite\footnote{Es costumbre utilizar el prefijo co- para indicar dualidad en contextos categóricos; sin embargo, dicho prefijo no siempre implica dualidad, sobre todo en otros contextos matemáticos, por lo que siempre debemos tener cuidado de no suponer dualidades que no existen.}.

\begin{Def} \label{Def: Cocono} 
    Sea $F:\mathscr{C}\to \mathscr{D}$ un funtor. Un \emph{cocono} de $F$ es un par $(\psi,D)$, donde $D\in\text{Obj}(\mathscr{D})$ y $\psi$ es una familia $\{F(X)\xrightarrow[]{\psi_X} D\}_{X\in\text{Obj}(\mathscr{C})}$ de morfismos en $\mathscr{D}$ tales que, para cualesquiera $X,Y\in\text{Obj}(\mathscr{C})$ y $X\xrightarrow[]{f} Y\in\text{Mor}(\mathscr{C})$, se tiene que $\psi_YF(f)=\psi_X$; diagramáticamente, el cocono $(\psi,D)$ se puede ver como el siguiente diagrama conmutativo
    \begin{equation}\label{eq: Cocono}
        \begin{tikzcd}
            F(X) \arrow[]{dr}[swap]{\psi_X} \arrow[]{rr}[]{F(f)} &&F(Y) \arrow[]{dl}[]{\psi_Y} &&&&\empty{} \arrow[phantom]{d}[]{\forall \ X,Y\in\text{Obj}(\mathscr{C}), X\xrightarrow[]{f} Y\in\text{Mor}(\mathscr{C}).} \\
                                                                 &D &&&&&\empty{} 
        \end{tikzcd}
    \end{equation}
\end{Def}

\begin{Obs}\label{Observaciones del cocono}
    El diagrama (\ref{eq: Cocono}) puede obtenerse invirtiendo las flechas del diagrama (\ref{eq: Cono}), reordenando los elementos, y reetiquetando a los objetos arbitrarios $X$ y $Y$. Similarmente, si invertimos las flechas del diagrama (\ref{eq: Cocono}), reordenamos los objetos y reetiquetamos a $X$ y a $Y$, podemos obtener el diagrama (\ref{eq: Cono}). Esto muestra que las nociones de cono y cocono son duales entre sí. Es decir, que si $F:\mathscr{C}\to \mathscr{D}$ es un funtor, $D$ es un objeto en $\mathscr{D}$ y $\psi$ es una familia $\{D\xrightarrow[]{\psi_X}F(X)\}_{X\in\text{Obj}(\mathscr{C})}$ de morfismos en $\mathscr{D}$, entonces
    \[
        (D,\psi) \text{ es un cono de } F \iff (\psi^\text{op},D) \text{ es un cocono de } F,
    \] 
    donde $\psi^\text{op} = \{F(X)\xrightarrow[]{\psi_X{}^\text{op}}D\}_{X\in\text{Obj}(\mathscr{C})}$.
\end{Obs}

\begin{Def}\label{Def: Morfismo de coconos}
    Sean $F:\mathscr{C}\to \mathscr{D}$ un funtor y $(\psi,D), (\psi',D')$ coconos de $F$. Un \emph{morfismo de coconos} $\beta:(\psi,D)\to (\psi',D')$ es un morfismo $D\xrightarrow[]{\beta} D'$ en $\mathscr{D}$ que hace conmutar el siguiente diagrama
    \begin{center}
        \begin{tikzcd}
            &F(X) \arrow[]{dl}[swap]{\psi_X} \arrow[]{dr}[]{\psi'_X} \\
            D \arrow[]{rr}[swap]{\beta} &&D'.
        \end{tikzcd}
        $\forall \ X\in \text{Obj}(\mathscr{C})$.
    \end{center}
\end{Def}

\begin{Obs}\label{Obs: morfismos de coconos}
    Sea $F:\mathscr{C}\to \mathscr{D}$ un funtor. Las clases de coconos de $F$ y de morfismos de coconos de $F$, junto con la composición de morfismos inducida de $\mathscr{D}$, forman una categoría llamada la \emph{categoría de coconos de} $F$, que denotamos por $\text{CoCone}(F)$. %En particular, el morfismo identidad de un cocono $(\psi,D)$ en la categoría de coconos de $F$ es el morfismo identidad $1_D\in\text{Mor}(\mathscr{D})$. Esto es, $1_{(\psi,D)}=1_D$.
\end{Obs}

\begin{Def} \label{Def: colímite}
    Sea $F:\mathscr{C}\to \mathscr{D}$ un funtor. Un \emph{cocono universal} de $F$ es un cocono de $F$ a través del cual se factorizan todos los coconos de $F$ de forma única. Es decir, un cocono $(\phi,U)$ de $F$ es un cocono universal si para cualquier cocono $(\psi,D)$ de $F$ existe un único morfismo de coconos $(\phi,U)\xrightarrow[]{\varphi} (\psi,D)$. A un cocono universal de $F$ también se le conoce como un \emph{colímite} del funtor $F$, y la propiedad descrita anteriormente se conoce como la \emph{propiedad universal del colímite del funtor} $F$.
\end{Def}

\begin{Obs} \label{Unicidad de colímites hasta isomorfismos}
    Sea $F:\mathscr{C}\to \mathscr{D}$ un funtor. Entonces los colímites de $F$, si existen, son únicos hasta isomorfismos en la categoría de coconos de $F$.

    \vspace{1mm}
    En efecto: Sean $(\phi,U)$ y $(\phi',U')$ colímites de $F$. Dado que son coconos universales y, en particular, coconos de $F$, por la propiedad universal del colímite de $F$ existen morfismos únicos $U\xrightarrow[]{\varphi} U', U'\xrightarrow[]{\varphi'} U$ en $\mathscr{D}$ tales que, para cualquier $X\in\text{Obj}(\mathscr{C})$, el siguiente diagrama conmuta
    \begin{equation}\label{eq: Unicidad de colímites}
        \begin{tikzcd}
                                                                         &F(X) \arrow[]{dl}[swap]{\phi_X} \arrow[]{dr}[]{\phi'_X} \\
            U \arrow[dotted, shift left]{rr}[]{\varphi} &&U'. \arrow[dotted, shift left]{ll}[]{\varphi'}
        \end{tikzcd}
    \end{equation}
    Ahora, del diagrama conmutativo (\ref{eq: Unicidad de colímites}) tenemos que los siguientes diagramas conmutan para cualquier $X\in\text{Obj}(\mathscr{C})$,
    \begin{center}
        \begin{tikzcd}
            &F(X) \arrow[]{dl}[swap]{\phi_X} \arrow[]{dr}[]{\phi_X} \\
            U \arrow[]{rr}[swap]{\varphi'\varphi} &&U
        \end{tikzcd}
        \hspace{1cm}
        \begin{tikzcd}
            &F(X) \arrow[]{dl}[swap]{\phi_X} \arrow[]{dr}[]{\phi_X} \\
            U' \arrow[]{rr}[swap]{\varphi\varphi'} &&U'
        \end{tikzcd}
    \end{center}
    Por otro lado, los siguientes diagramas conmutan para cualquier $X\in\text{Obj}(\mathscr{C})$,
    \begin{center}
        \begin{tikzcd}
            &F(X) \arrow[]{dl}[swap]{\phi_X} \arrow[]{dr}[]{\phi_X} \\
            U \arrow[]{rr}[swap]{1_U} &&U
        \end{tikzcd}
        \hspace{1cm}
        \begin{tikzcd}
            &F(X) \arrow[]{dl}[swap]{\phi_X} \arrow[]{dr}[]{\phi_X} \\
            U' \arrow[]{rr}[swap]{1_{U'}} &&U'
        \end{tikzcd}
    \end{center}
    Por ende, de la propiedad universal del colímite de $F$ se sigue que $\varphi'\varphi=1_U$ y $\varphi\varphi'=1_{U'}$. Por lo tanto, concluimos que los colímites de $F$ son únicos hasta isomorfismos en la categoría de coconos de $F$.
\end{Obs}

\subsection*{Objetos finales, objetos iniciales y objetos cero} \label{Ssec: Objetos finales, objetos iniciales y objetos cero} %\section{Objetos finales, objetos iniciales y objetos cero} \label{Sec: Objetos finales, objetos iniciales y objetos cero}

\begin{Nota}
    Dado un conjunto $Z$, denotaremos por $|Z|$ al cardinal de dicho conjunto.
\end{Nota}

\begin{Def}\label{Def: Objeto final}
    Sea $\mathscr{C}$ una categoría. Un objeto $F$ en $\mathscr{C}$ es un \emph{objeto final} si
    \[
    |\text{Hom}_\mathscr{C}(X,F)| = 1
    \] 
    para todo $X\in\text{Obj}(\mathscr{C})$.
\end{Def}

\begin{Obs} \label{Unicidad de objetos finales hasta isomorfismo}
    Los objetos finales en una categoría $\mathscr{C}$, si existen, son únicos hasta isomorfismos en $\mathscr{C}$.

    \vspace{1mm}
    En efecto: Sean $F,F'$ objetos finales en una categoría $\mathscr{C}$. Entonces, tenemos que
    \begin{align}
        |\text{Hom}_\mathscr{C}(F',F)| &= 1 = |\text{Hom}_\mathscr{C}(F,F')|, \label{eq: Objetos finales 1} \\
        |\text{Hom}_\mathscr{C}(F,F)| &= 1 = |\text{Hom}_\mathscr{C}(F',F')|. \label{eq: Objetos finales 2}
    \end{align}
    Por (\ref{eq: Objetos finales 1}), existen $h\in\text{Hom}_\mathscr{C}(F,F')$ y $j\in\text{Hom}_\mathscr{C}(F',F)$. Por axiomas de categoría, tenemos que $jh\in\text{Hom}_\mathscr{C}(F,F), hj\in\text{Hom}_\mathscr{C}(F',F'), 1_F\in\text{Hom}_\mathscr{C}(F,F)$ y $1_{F'}\in\text{Hom}_\mathscr{C}(F',F')$. Por (\ref{eq: Objetos finales 2}), se sigue que $jh=1_F$ y $hj=1_{F'}$, por lo que $F\simeq F'$.
\end{Obs}

%\begin{Obs}\label{Observaciones de los objetos finales}
%    Los objetos finales en una categoría $\mathscr{C}$, si existen, son únicos hasta isomorfismos en $\mathscr{C}$.
%
%    \vspace{1mm}
%    En efecto: Notemos que, por definición, un objeto final es un límite sobre el diagrama vacío $F:\varnothing\to \mathscr{C}$. Por la Observación \ref{Unicidad de límites hasta isomorfismo}, tenemos que un objeto final es único hasta isomorfismos en la categoría de conos del diagrama vacío. Como en tal caso la categoría de conos del diagrama vacío es una subcategoría de $\mathscr{C}$, de la Observación \ref{Observaciones de las subcategorías} concluimos que los objetos finales en $\mathscr{C}$ son únicos hasta isomorfismos en $\mathscr{C}$.
%\end{Obs}

\begin{Def}\label{Def: Objeto inicial}
    Sea $\mathscr{C}$ una categoría. Un objeto $I$ en $\mathscr{C}$ es un \emph{objeto inicial} si
    \[
    |\text{Hom}_\mathscr{C}(I,X)| = 1
    \] 
    para todo $X\in\text{Obj}(\mathscr{C})$.
\end{Def}

\begin{Obs}\leavevmode\label{Obs: Objetos iniciales}
    \begin{enumerate}[label=(\arabic*)]
    
        \item Los objetos iniciales en una categoría $\mathscr{C}$, si existen, son únicos hasta isomorfismos en $\mathscr{C}$.

    \vspace{1mm}
    En efecto: Sean $I,I'$ objetos iniciales en una categoría $\mathscr{C}$. Entonces, tenemos que
    \begin{align}
        |\text{Hom}_\mathscr{C}(I,I')| &= 1 = |\text{Hom}_\mathscr{C}(I',I)|, \label{eq: Objetos iniciales 1} \\
        |\text{Hom}_\mathscr{C}(I,I)| &= 1 = |\text{Hom}_\mathscr{C}(I',I')|. \label{eq: Objetos iniciales 2}
    \end{align}
    Por (\ref{eq: Objetos iniciales 1}), existen $f\in\text{Hom}_\mathscr{C}(I,I')$ y $g\in\text{Hom}_\mathscr{C}(I',I)$. Por axiomas de categoría, tenemos que $gf\in\text{Hom}_\mathscr{C}(I,I), fg\in\text{Hom}_\mathscr{C}(I',I'), 1_I\in\text{Hom}_\mathscr{C}(I,I)$ y $1_{I'}\in\text{Hom}_\mathscr{C}(I',I')$. Por (\ref{eq: Objetos iniciales 2}), se sigue que $gf=1_I$ y $fg=1_{I'}$, por lo que $I\simeq I'$.

        \item Las nociones de objeto final e inicial son duales entre sí. Es decir, si $\mathscr{C}$ es una categoría y $X$ es un objeto en $\mathscr{C}$, entonces
            \[
                X \text{ es un objeto final en } \mathscr{C} \iff X \text{ es un objeto inicial en } \mathscr{C}^\text{op}.
            \] 
    \end{enumerate}
\end{Obs}

%\begin{Obs}\label{Observaciones de los objetos iniciales}
%    Los objetos iniciales en una categoría $\mathscr{C}$, si existen, son únicos hasta isomorfismos en $\mathscr{C}$.
%
%    \vspace{1mm}
%    En efecto: Notemos que, por definición, un objeto inicial es un colímite sobre el diagrama vacío $F:\varnothing\to \mathscr{C}$. Por la Observación \ref{Unicidad de colímites hasta isomorfismo}, tenemos que un objeto inicial es único hasta isomorfismos en la categoría de coconos del diagrama vacío. Como en tal caso la categoría de coconos del diagrama vacío es una subcategoría de $\mathscr{C}$, de la Observación \ref{Observaciones de las subcategorías} concluimos que los objetos iniciales en $\mathscr{C}$ son únicos hasta isomorfismos en $\mathscr{C}$.
%\end{Obs}

\begin{Def}\label{Def: Objeto cero}
    Sea $\mathscr{C}$ una categoría. Un objeto $0$ en $\mathscr{C}$ es un \emph{objeto cero} si es un objeto final e inicial, es decir, si
    \[
    |\text{Hom}_\mathscr{C}(X,0)| = 1 = |\text{Hom}_\mathscr{C}(0,X)|
    \] 
    para todo $X\in\text{Obj}(\mathscr{C})$.
\end{Def}

\begin{Obs}\label{Obs: Objeto cero}\leavevmode

    \begin{enumerate}[label=(\arabic*)]
    
        \item Los objetos cero en una categoría $\mathscr{C}$, si existen, son únicos hasta isomorfismos en $\mathscr{C}$.

        \item La noción de objeto cero es auto dual. Es decir, si $\mathscr{C}$ es una categoría y $X$ es un objeto en $\mathscr{C}$, entonces
            \[
            X \text{ es un objeto cero en } \mathscr{C} \iff X \text{ es un objeto cero en } \mathscr{C}^\text{op}.
            \] 
    \end{enumerate}
\end{Obs}

\subsection*{Igualadores y coigualadores} \label{Ssec: Igualadores y coigualadores}
%\section{Igualadores y coigualadores} \label{Sec: Igualadores y coigualadores}

\begin{Def} \label{Def: Igualador}
    Sean $\alpha,\beta:A\to B$ morfismos en una categoría $\mathscr{C}$. Un morfismo $I\xrightarrow[]{\iota} A$ en $\mathscr{C}$ es un \emph{igualador} para el par $(\alpha,\beta)$ si las siguientes condiciones se satisfacen.

    \begin{itemize}
        \item[(I1)] $\beta\iota=\alpha\iota$.

        \item[(I2)] Propiedad universal del igualador: para todo morfismo $X\xrightarrow[]{f} A$ en $\mathscr{C}$ tal que $\beta f=\alpha f$, existe un único morfismo $X\xrightarrow[]{f'} I$ tal que $\iota f' = f$; diagramáticamente,
            \begin{center}
                \begin{tikzcd}
                    &X \arrow[dotted]{dl}[swap]{\exists! \ f'} \arrow[]{d}[]{f} \\
                    I \arrow[]{r}[swap]{\iota} &A \arrow[shift left]{r}[]{\alpha} \arrow[shift right]{r}[swap]{\beta} &B.
                \end{tikzcd}
            \end{center}
    \end{itemize}
\end{Def}

\begin{Obs} \label{Mendoza-1.1.4}
    Sean $\alpha,\beta:A\to B$ en una categoría $\mathscr{C}$.
    \begin{enumerate}[label=(\arabic*)]
        \item Si $I\xrightarrow[]{\iota} A$ es un igualador de $(\alpha,\beta)$, entonces $\iota\in\text{Mon}_\mathscr{C}(-,A)$.

        \item Si $I\xhookrightarrow{\iota} A$ y $I'\xhookrightarrow{\iota'} A$ son igualadores de $(\alpha,\beta)$, entonces $\iota\simeq\iota'$ en $\text{Mon}_\mathscr{C}(-,A)$.

            \item En virtud de (1) y (2), y en caso de que $(\alpha,\beta)$ admita un igualador, denotaremos por $\text{Equ}(\alpha,\beta)\xhookrightarrow[]{\iota}A$ a la elección de uno de ellos, siguiendo la notación preponderante derivada del término \emph{equalizer} en inglés.

            %\item Por definición, un igualador de $(\alpha,\beta)$ es un límite\footnote{Esto me parece muy informal. Discutir con Octavio. Tal vez debería poner al igualador y coigualador antes de la sección de límites y colímites.} sobre el diagrama conmutativo en $\mathscr{C}$ \begin{tikzcd} A \arrow[shift left]{r}[]{\alpha} \arrow[shift right]{r}[swap]{\beta} &B \end{tikzcd}.
        \end{enumerate}
    \end{Obs}

    \begin{Def} \label{Def: Coigualador}
        Sean $\alpha,\beta:A\to B$ morfismos en una categoría $\mathscr{C}$. Un morfismo $B\xrightarrow[]{\nu} C$ en $\mathscr{C}$ es un \emph{coigualador} para el par $(\alpha,\beta)$ si las siguientes condiciones se satisfacen.

        \begin{itemize}
            \item[(CI1)] $\nu\beta=\nu\alpha$.

            \item[(CI2)] Propiedad universal del coigualador: para todo morfismo $B\xrightarrow[]{f} Y$ en $\mathscr{C}$ tal que $f\beta=f\alpha$, existe un único morfismo $C\xrightarrow[]{f'} Y$ tal que $f'\nu = f$; diagramáticamente,
                \begin{center}
                    \begin{tikzcd}
                        A \arrow[shift left]{r}[]{\alpha} \arrow[shift right]{r}[swap]{\beta} &B \arrow[]{d}[swap]{f} \arrow[]{r}[]{\nu} &C. \arrow[dotted]{dl}[]{\exists! \ f'} \\
                                                                                              &Y
                    \end{tikzcd}
                \end{center}
        \end{itemize}
    \end{Def}

    \begin{Obs} \label{Mendoza-1.1.4}
        Sean $\alpha,\beta:A\to B$ en una categoría $\mathscr{C}$.
        \begin{enumerate}[label=(\arabic*)]
            \item Si $\nu\to C$ es un coigualador de $(\alpha,\beta)$, entonces $\nu\in\text{Epi}_\mathscr{C}(B,-)$.

            \item Si $B\xtwoheadrightarrow{\nu} C$ y $B\xtwoheadrightarrow{\nu'} C'$ son coigualadores de $(\alpha,\beta)$, entonces $\nu\simeq\nu'$ en $\text{Epi}_\mathscr{C}(B,-)$.

            \item En virtud de (1) y (2), y en caso de que $(\alpha,\beta)$ admita un coigualador, denotaremos por $B\xtwoheadrightarrow[]{\nu}\text{CoEq}(\alpha,\beta)$ a la elección de uno de ellos.

            %\item Por definición, un coigualador de $(\alpha,\beta)$ es un colímite\footnote{Esto me parece muy informal. Discutir con Octavio. Tal vez debería poner al igualador y coigualador antes de la sección de límites y colímites.} sobre el diagrama conmutativo en $\mathscr{C}$ \begin{tikzcd} A \arrow[shift left]{r}[]{\alpha} \arrow[shift right]{r}[swap]{\beta} &B \end{tikzcd}.
    \end{enumerate}
\end{Obs}

\subsection*{Productos fibrados, sumas fibradas y cuadrados bicartesianos} \label{Mendoza-1.2}
%\section{Productos fibrados, sumas fibradas y cuadrados bicartesianos} \label{Mendoza-1.2}

\begin{Def}
    Sean $A_1\xrightarrow[]{\alpha_1} A$ y $A_2\xrightarrow[]{\alpha_2} A$ morfismos en una categoría $\mathscr{C}$. Un \emph{producto fibrado} o \emph{pull-back} para el diagrama $A_1\xrightarrow[]{\alpha_1}A\xleftarrow[]{\alpha_2}A_2$ es un diagrama conmutativo en $\mathscr{C}$
    \begin{center}
        \begin{tikzcd}
            P \arrow[]{d}[swap]{\beta_1} \arrow[]{r}[]{\beta_2} \arrow[phantom]{dr}[]{\text{PB}} &A_2 \arrow[]{d}[]{\alpha_2} \\
            A_1 \arrow[]{r}[swap]{\alpha_1} &A
        \end{tikzcd}
    \end{center}
    que satisface la siguiente propiedad universal: para cualesquiera morfismos $P'\xrightarrow[]{\beta_1'} A_1, P'\xrightarrow[]{\beta_2'} A_2$ en $\mathscr{C}$ tales que $\alpha_2\beta_2' = \alpha_1\beta_1'$, se tiene que existe un único morfismo $P'\xrightarrow[]{\gamma} P$ en $\mathscr{C}$ tal que $\beta_1\gamma = \beta_1'$ y $\beta_2\gamma = \beta_2'$; diagramáticamente,
    \begin{center}
        \begin{tikzcd}
            P' \arrow[bend right = 30]{ddr}[swap]{\beta_1'} \arrow[bend left = 30]{drr}[]{\beta_2'} \arrow[dotted]{dr}[]{\exists! \ \gamma} \\
            &P \arrow[]{d}[swap]{\beta_1} \arrow[]{r}[]{\beta_2} \arrow[phantom]{dr}[]{\text{PB}} &A_2 \arrow[]{d}[]{\alpha_2} \\
            &A_1 \arrow[]{r}[swap]{\alpha_1} &A.
        \end{tikzcd}
    \end{center}
\end{Def}

\begin{Obs} \label{Mendoza-Ejer.7}
    Los productos fibrados, si existen, son únicos hasta isomorfismos.
    \vspace{1mm}

    En efecto: Sean $A_1\xrightarrow[]{\alpha_1}A$ y $A_2\xrightarrow[]{\alpha_2}A$ morfismos en una categoría $\mathscr{C}$. Supongamos que los siguientes diagramas conmutativos en $\mathscr{C}$ son productos fibrados para el diagrama $A_1\xrightarrow[]{\alpha_1}A\xleftarrow[]{\alpha_2}A_2$
    \begin{center}
        \begin{tikzcd}
            P \arrow[]{d}[swap]{\beta_1} \arrow[]{r}[]{\beta_2} \arrow[phantom]{dr}[]{\text{PB}} &A_2 \arrow[]{d}[]{\alpha_2} & &P' \arrow[]{d}[swap]{\beta_1'} \arrow[]{r}[]{\beta_2'} \arrow[phantom]{dr}[]{\text{PB}} &A_2 \arrow[]{d}[]{\alpha_2} \\
            A_1 \arrow[]{r}[swap]{\alpha_1} &A & &A_1\arrow[]{r}[swap]{\alpha_1} &A.
        \end{tikzcd}
    \end{center}
    Por la propiedad universal de los productos fibrados, existen morfismos únicos $P\xrightarrow[]{\gamma} P', P'\xrightarrow[]{\gamma'} P$ en $\mathscr{C}$ tales que el siguiente diagrama conmuta
    \begin{center}
        \begin{tikzcd}
            P'\arrow[bend right = 30]{ddr}[swap]{\beta_1'}\arrow[bend left = 30]{drr}{\beta_2'}\arrow[dotted, shift left]{dr}{\gamma'} & & \\
                                                                                                                                       &P \arrow[dotted, shift left]{ul}{\gamma} \arrow{r}{\beta_2} \arrow{d}{\beta_1} &A_2 \arrow{d}{\alpha_2} \\
                                                                                                                                       &A_1 \arrow{r}[swap]{\alpha_1} &A.
        \end{tikzcd}
    \end{center}
    Más aún, de la unicidad en la propiedad universal del producto fibrado se sigue que $\gamma'\gamma=1_P$ y $\gamma\gamma'=1_{P'}$, por lo que $P\simeq P'$.
\end{Obs}

\begin{Coro} \label{Mendoza-Ejer.8(a)}
    Sean $\mathscr{C}$ una categoría y el diagrama conmutativo en $\mathscr{C}$
    \begin{center}
        \begin{tikzcd}
            P \arrow[]{d}[swap]{\beta_1} \arrow[]{r}[]{\beta_2} \arrow[phantom]{dr}[]{\text{PB}} &A_2 \arrow[]{d}[]{\alpha_2} \\
            A_1 \arrow[]{r}[swap]{\alpha_1} &A
        \end{tikzcd}
    \end{center}
    un producto fibrado. Entonces, se cumplen las siguientes condiciones.

    \begin{enumerate}[label=(\alph*)]
    
        \item Si $\alpha_1$ es un monomorfismo, entonces $\beta_2$ también lo es.

        \item Si $\alpha_1$ es un isomorfismo, entonces $\beta_2$ también lo es.
    \end{enumerate}
\end{Coro}

\begin{proof}\leavevmode
    \begin{enumerate}[label=(\alph*)]
    
    \item Supongamos que $\alpha_1$ es un monomorfismo y que existen $X\in\text{Obj}(\mathscr{C})$ y $\gamma,\gamma':X\to P\in\text{Mor}(\mathscr{C})$ tales que $\beta_2\gamma = \beta_2\gamma'$. Como por hipótesis $\alpha_1\beta_1=\alpha_2\beta_2$, tenemos que
            \begin{align*}
                \alpha_1\beta_1\gamma &= \alpha_2\beta_2\gamma \\
                                        &= \alpha_2\beta_2\gamma' \\
                                        &= \alpha_1\beta_1\gamma'.
            \end{align*}
            Dado que $\alpha_1$ es un monomorfismo, se sigue que $\beta_1\gamma = \beta_1\gamma'$. Por lo tanto, $\gamma$ y $\gamma'$ son tales que el siguiente diagrama conmuta
            \begin{center}
                \begin{tikzcd}
                    X \arrow[bend right = 30]{ddr}[swap]{\beta_1\gamma} \arrow[bend left = 30]{drr}{\beta_2\gamma'} \arrow[shift right]{dr}[swap]{\gamma} \arrow[shift left]{dr}{\gamma'} & & \\
                                                                                                                                                                                          &P \arrow{r}{\beta_2} \arrow{d}[swap]{\beta_1} \arrow[phantom]{dr}[]{\text{PB}} &A_2 \arrow{d}{\alpha_2} \\
                                                                                                                                                                                          &A_1 \arrow{r}[swap]{\alpha_1} &A.
                \end{tikzcd}
            \end{center}
            Por la propiedad universal del producto fibrado se sigue que $\gamma=\gamma'$. Por lo tanto, $\beta_2$ es un monomorfismo. Observemos que por simetría se sigue que, si $\alpha_2$ es un monomorfismo, entonces $\beta_1$ también lo es.

        \item Supongamos que $\alpha_1$ es un isomorfismo. Observemos que el siguiente diagrama en $\mathscr{C}$ conmuta
    \begin{center}
        \begin{tikzcd}
            A_2 \arrow[bend left = 30]{drr}[]{1_{A_2}} \arrow[bend right = 30]{ddr}[swap]{\alpha_1^{-1}\alpha_2} \\
            &P \arrow[]{d}[swap]{\beta_1} \arrow[]{r}[]{\beta_2} \arrow[phantom]{dr}[]{\text{PB}} &A_2 \arrow[]{d}[]{\alpha_2} \\
            &A_1 \arrow[]{r}[swap]{\alpha_1} &A.
        \end{tikzcd}
    \end{center}
    Por la propiedad universal del producto fibrado, existe un único morfismo $A_2\xrightarrow[]{\gamma} P$ tal que $\beta_1\gamma=\alpha_1^{-1}\alpha_2$ y $\beta_2\gamma=1_{A_2}$. Luego, tenemos que
    \begin{align*}
        \beta_2\gamma\beta_2 &= 1_{A_2}\beta_2 \\
                             &= \beta_2 \\
                             &= \beta_21_P, \\ \\
        \beta_1\gamma\beta_2 &= \alpha_1^{-1}\alpha_2\beta_2 \\
                             &= \alpha_1^{-1}\alpha_1\beta_1 \\
                             &= 1_{A_1}\beta_1 \\
                             &= \beta_1 \\
                             &= \beta_11_P.
    \end{align*}
    De la unicidad en la propiedad universal del producto fibrado se sigue que $\gamma\beta_2=1_P$, de donde concluimos que $\beta_2$ es un isomorfismo.
    \end{enumerate}
\end{proof}

\begin{Prop}\label{Mendoza-Ejer.9}
    Sean $\mathscr{C}$ una categoría y consideremos un diagrama en $\mathscr{C}$
    \begin{center}
        \begin{tikzcd}
            P\arrow{r}{g}\arrow{d}[swap]{\beta_1} &B'\arrow{d}{\theta_1} &Q\arrow{l}[swap]{\alpha_2} \arrow{d}{\theta_2} \\
            A\arrow{r}[swap]{f} &B &I\arrow{l}{\gamma_2}
        \end{tikzcd}
    \end{center}
    tal que ambos cuadrados son productos fibrados. Si $\theta_1$ y $\gamma_2$ son monomorfismos en $\mathscr{C}$ y existe $A\xrightarrow[]{\gamma_1} I$ tal que $f = \gamma_2\gamma_1$, entonces existe $P\xrightarrow[]{\alpha_1} Q$ tal que $\alpha_2\alpha_1=g$ y el siguiente diagrama es un producto fibrado

    \begin{center}
        \begin{tikzcd}
            P \arrow{d}[swap]{\beta_1} \arrow{r}{\alpha_1} &Q\arrow{d}{\theta_2} \\
            A \arrow{r}[swap]{\gamma_1} &I.
        \end{tikzcd}
    \end{center}
\end{Prop}

\begin{proof}

    Supongamos que $\theta_1$ y $\gamma_2$ son monomorfismos en $\mathscr{C}$ y que existe $A\xrightarrow[]{\gamma_1} I$ tal que $f=\gamma_2\gamma_1$. Entonces, tenemos que el siguiente diagrama en $\mathscr{C}$ conmuta
    \begin{equation}\label{eq: Ejer.9-1}
        \begin{tikzcd}
            P\arrow{r}{g}\arrow{d}[swap]{\beta_1} &B'\arrow{d}{\theta_1} &Q\arrow{l}[swap]{\alpha_2} \arrow{d}{\theta_2} \\
            A\arrow{r}[swap]{f} \arrow[bend right = 60]{rr}{\gamma_1} &B &I.\arrow{l}{\gamma_2}
        \end{tikzcd}
    \end{equation}
    En particular, observemos que $\gamma_2(\gamma_1\beta_1) = \theta_1 g$. Reordenando el diagrama \ref{eq: Ejer.9-1}, obtenemos el diagrama conmutativo en $\mathscr{C}$
    \begin{center}
        \begin{tikzcd}
            P \arrow[bend right = 30]{ddr}[swap]{\gamma_1\beta_1} \arrow[bend left = 30]{drr}{g} & & \\
                                                                                                 &Q \arrow{d}[swap]{\theta_2} \arrow{r}{\alpha_2} &B' \arrow{d}{\theta_1}\\
                                                                                                 &I \arrow{r}[swap]{\gamma_2} &B.
        \end{tikzcd}
    \end{center}
    Dado que el cuadrado conmutativo es un producto fibrado por hipótesis, de la propiedad universal del producto fibrado se sigue que existe un único morfismo $P\xrightarrow[]{\alpha_1} Q$ tal que $\alpha_2\alpha_1 = g$ y $\theta_2\alpha_1 = \gamma_1\beta_1$. Veamos que el diagrama conmutativo dado por la segunda igualdad
    \begin{center}
        \begin{tikzcd}
            P \arrow{d}[swap]{\beta_1} \arrow{r}{\alpha_1} &Q \arrow{d}{\theta_2} \\
            A \arrow{r}[swap]{\gamma_1} &I
        \end{tikzcd}
    \end{center}
    es un producto fibrado para $A\xrightarrow[]{\gamma_1} I \xleftarrow[]{\theta_2} Q$. En efecto, sean $P'\in\text{Obj}(\mathscr{C})$ y $P'\xrightarrow[]{\varphi_1} A, P'\xrightarrow[]{\varphi_2} Q$ morfismos en $\mathscr{C}$ tales que $\gamma_1\varphi_1=\theta_2\varphi_2$. Entonces, como $\gamma_2\theta_2 = \theta_1\alpha_2$, tenemos que
    \begin{align*}
        f\varphi_1 &= \gamma_2\gamma_1\varphi_1 \\
                   &= \gamma_2\theta_2\varphi_2 \\
                   &= \theta_1\alpha_2\varphi_2,
    \end{align*}
    por lo que tenemos el diagrama conmutativo en $\mathscr{C}$
    \begin{equation}\label{eq: Ejer.9-2}
        \begin{tikzcd}
            P' \arrow[bend left = 30]{rrd}[]{\alpha_2\varphi_2} \arrow[bend right = 30]{ddr}[swap]{\varphi_1} \\
            &P \arrow[]{d}[swap]{\beta_1} \arrow[]{r}[]{g} &B' \arrow[]{d}[]{\theta_1} \\
            &I \arrow[]{r}[swap]{f} &B.
        \end{tikzcd}
    \end{equation}
    Como el cuadrado del diagrama \ref{eq: Ejer.9-2} es un producto fibrado, existe $P'\xrightarrow[]{\varphi}P$ en $\mathscr{C}$ tal que $g\varphi=\alpha_2\varphi_2$ y $\beta_1\varphi=\varphi_1$. Ahora, como $\gamma_2$ es un monomorfismo y el diagrama en $\mathscr{C}$
    \begin{center}
        \begin{tikzcd}
            Q \arrow[]{d}[swap]{\theta_2} \arrow[]{r}[]{\alpha_2} &B' \arrow[]{d}[]{\theta_1} \\
            I \arrow[]{r}[swap]{\gamma_2} &B
        \end{tikzcd}
    \end{center}
    es un producto fibrado, se sigue de la Proposición \ref{Mendoza-Ejer.8(a)} que $\alpha_2$ también es un monomorfismo. Por ende, de $\alpha_2\varphi_2 = g\varphi = \alpha_2\alpha_1\varphi$ se sigue que $\varphi_2=a\alpha_1\varphi$, por lo que el diagrama en $\mathscr{C}$
    \begin{center}
        \begin{tikzcd}
            P' \arrow[dotted]{dr}[swap]{\varphi} \arrow[bend right = 30]{ddr}[swap]{\varphi_1} \arrow[bend left = 30]{rrd}[]{\varphi_2} \\
            &P \arrow[]{d}[swap]{\beta_1} \arrow[]{r}[]{\alpha_1} &Q \arrow[]{d}[]{\theta_2} \\
            &A \arrow[]{r}[swap]{\gamma_1} &I
        \end{tikzcd}
    \end{center}
    conmuta. Como $\theta_2$ y $\alpha_1$ son monomorfismos, entonces por el inciso (2) de la Observación \ref{Obs: Morfismos especiales} sabemos que $\theta_2\alpha_1$ es un monomorfismo. Por lo tanto, si existe $P'\xrightarrow[]{\varphi'} P$ en $\mathscr{C}$ tal que $\alpha_1\varphi'=\varphi_2$ y $\beta_1\varphi' = \varphi_1$, entonces tenemos que $\theta_2\alpha_1\varphi' = \theta_2\alpha_1\varphi$, de donde se sigue que $\varphi'=\varphi$.
\end{proof}

\begin{Def}
    Sea $\mathscr{C}$ una categoría. Decimos que $\mathscr{C}$ es una \emph{categoría con productos fibrados} si para cualquier diagrama $A_1\xrightarrow[]{\alpha_1}A\xleftarrow[]{\alpha_2}A_2$ en $\mathscr{C}$ existe el producto fibrado correspondiente.
\end{Def}

%La noción dual del producto fibrado es la siguiente.

\begin{Def} \label{Mendoza-Ejer.10}
    Sean $A\xrightarrow[]{\alpha_1} A_1$ y $A\xrightarrow[]{\alpha_2} A_2$ morfismos en una categoría $\mathscr{C}$. Una \emph{suma fibrada} o \emph{push-out} para el diagrama $A_1 \xleftarrow{\alpha_1} A \xrightarrow{\alpha_2}A_2$ es un diagrama conmutativo en $\mathscr{C}$
    \begin{center}
        \begin{tikzcd}
            A \arrow{d}[swap]{\alpha_1} \arrow{r}{\alpha_2} & A_2 \arrow{d}{\beta_2} \\
            A_1 \arrow{r}[swap]{\beta_1} & S \arrow[phantom]{ul}[]{\text{PO}}
        \end{tikzcd}
    \end{center}
    que satisface la siguiente propiedad universal: para cualesquiera morfismos $A_1\xrightarrow[]{\beta_1'} S',A_2\xrightarrow[]{\beta_2'} S'$ en $\mathscr{C}$ tales que $\beta_1'\alpha_1 = \beta_2'\alpha_2$, se tiene que existe un único morfismo $S\xrightarrow[]{\gamma} S'$ tal que $\gamma\beta_1 = \beta_1'$ y $\gamma\beta_2 = \beta_2'$; diagramáticamente, 
    \begin{center}
        \begin{tikzcd}
            A \arrow{d}[swap]{\alpha_1} \arrow{r}{\alpha_2} &A_2 \arrow{d}{\beta_2} \arrow[bend left = 30]{ddr}{\beta_2'} & \\
            A_1 \arrow{r}[swap]{\beta_1} \arrow[bend right = 30]{drr}[swap]{\beta_1'} & S\arrow[dotted]{dr}[swap]{\exists! \ \gamma} \arrow[phantom]{ul}[]{\text{PO}} & \\
              &    &S'.
        \end{tikzcd}
    \end{center}
\end{Def}

\begin{Obs}\leavevmode\label{Obs: Sumas fibradas}
    \begin{enumerate}[label=(\arabic*)]
    
        \item Las sumas fibradas, si existen, son únicas hasta isomorfismos.

        En efecto: Sean $A\xrightarrow[]{\alpha_1} A_1$ y $A\xrightarrow[]{\alpha_2} A_2$ morfismos en una categoría $\mathscr{C}$. Supongamos que los siguientes diagramas conmutativos en $\mathscr{C}$ son sumas fibradas para el diagrama $A_1\xleftarrow[]{\alpha_1}A\xrightarrow[]{\alpha_2}A_2$
        \begin{center}
            \begin{tikzcd}
                A \arrow{d}[swap]{\alpha_1} \arrow{r}{\alpha_2} &A_2 \arrow{d}{\beta_2} & &A \arrow[]{d}[swap]{\alpha_1} \arrow[]{r}[]{\alpha_2} &A_2 \arrow[]{d}[]{\beta_2'} \\
                A_1 \arrow{r}[swap]{\beta_1} &S \arrow[phantom]{ul}[]{\text{PO}} & &A_1 \arrow[]{r}[swap]{\beta_1'} &S'. \arrow[phantom]{ul}[]{\text{PO}}
            \end{tikzcd}
        \end{center}
        Por la propiedad universal de las sumas fibradas, existen morfismos únicos $S\xrightarrow[]{\gamma} S',S'\xrightarrow[]{\gamma'} S$ en $\mathscr{C}$ tales que el siguiente diagrama conmuta
        \begin{center}
            \begin{tikzcd}
                A \arrow{d}[swap]{\alpha_1} \arrow{r}{\alpha_2} &A_2 \arrow{d}{\beta_2} \arrow[bend left = 30]{ddr}{\beta_2'} & \\
                A_1 \arrow{r}[swap]{\beta_1} \arrow[bend right = 30]{drr}[swap]{\beta_1'} & S\arrow[shift left, dotted]{dr}{\gamma} & \\
                                                                                    &    &S'. \arrow[shift left, dotted]{ul}{\gamma'}
            \end{tikzcd}
        \end{center}
        Más aún, de la unicidad en la propiedad universal de la suma fibrada se sigue que $\gamma'\gamma=1_S$ y $\gamma\gamma'=1_{S'}$, por lo que $S\simeq S'$.

    \item Las nociones de producto fibrado y suma fibrada son duales entre sí. Es decir, si $\mathscr{C}$ es una categoría y $A_1\xrightarrow[]{\alpha_1}A, A_2\xrightarrow[]{\alpha_2}A$ son morfismos en $\mathscr{C}$, entonces las siguientes condiciones son equivalentes.
        %\begin{center}
        %    \adjustbox{scale=0.97}{%
        %        \begin{tikzcd}
        %            X \arrow[]{d}[swap]{\beta_1} \arrow[]{r}[]{\beta_2} &A_2 \arrow[]{d}[]{\alpha_2} \\
        %            A_1 \arrow[]{r}[swap]{\alpha_1} &A
        %        \end{tikzcd}
        %        es un producto fibrado en $\mathscr{C}$ $\iff$
        %        \begin{tikzcd}
        %            A \arrow[]{d}[swap]{\alpha_2{}^\text{op}} \arrow[]{r}[]{\alpha_1{}^\text{op}} &A_1 \arrow[]{d}[]{\beta_1{}^\text{op}} \\
        %            A_2 \arrow[]{r}[swap]{\beta_2{}^\text{op}} &X
        %        \end{tikzcd}
        %    es una suma fibrada en $\mathscr{C}^\text{op}$.}
        %\end{center}
        \begin{itemize}
           \item[(i)] \begin{tikzcd}
               X \arrow[]{d}[swap]{\beta_1} \arrow[]{r}[]{\beta_2} &A_2 \arrow[]{d}[]{\alpha_2} \\
               A_1 \arrow[]{r}[swap]{\alpha_1} &A
           \end{tikzcd}
           es un producto fibrado para el diagrama $A_1\xrightarrow[]{\alpha_1}A\xleftarrow[]{\alpha_2}A_2$ en $\mathscr{C}$.
           \item[(ii)] \begin{tikzcd}
               A \arrow[]{d}[swap]{\alpha_2{}^\text{op}} \arrow[]{r}[]{\alpha_1{}^\text{op}} &A_1 \arrow[]{d}[]{\beta_1{}^\text{op}} \\
               A_2 \arrow[]{r}[swap]{\beta_2{}^\text{op}} &X
           \end{tikzcd}
           es una suma fibrada para el diagrama $A_1\xleftarrow[]{\alpha_1{}^\text{op}}A\xrightarrow[]{\alpha_2{}^\text{op}}A_2$ en $\mathscr{C}^\text{op}$.
        \end{itemize}
    \end{enumerate}
\end{Obs}

\begin{Coro} \label{Mendoza-Ejer.8(a)*}
    Sean $\mathscr{C}$ una categoría y el diagrama conmutativo en $\mathscr{C}$
    \begin{center}
        \begin{tikzcd}
            A \arrow[]{d}[swap]{\alpha_1} \arrow[]{r}[]{\alpha_2} \arrow[phantom]{dr}[]{\text{PO}} &A_2 \arrow[]{d}[]{\beta_2} \\
            A_1 \arrow[]{r}[swap]{\beta_1} &S
        \end{tikzcd}
    \end{center}
    una suma fibrada. Entonces, se cumplen las siguientes condiciones.

    \begin{enumerate}[label=(\alph*)]
    
        \item Si $\alpha_1$ es un epimorfismo, entonces $\beta_2$ también lo es.

        \item Si $\alpha_1$ es un isomorfismo, entonces $\beta_2$ también lo es.
    \end{enumerate}
\end{Coro}

\begin{proof}\leavevmode
    \begin{enumerate}[label=(\alph*)]

        \item Supongamos que $\alpha_1$ es un epimorfismo y que existen $X\in\text{Obj}(\mathscr{C})$ y $\gamma,\gamma':S\to X\in\text{Mor}(\mathscr{C})$ tales que $\gamma\beta_2 = \gamma'\beta_2$. Como por hipótesis $\beta_1\alpha_1=\beta_2\alpha_2$, tenemos que
            \begin{align*}
                \gamma\beta_1\alpha_1 &= \gamma\beta_2\alpha_2 \\
                                        &= \gamma'\beta_2\alpha_2 \\
                                        &= \gamma'\beta_1\alpha_1.
            \end{align*}
            Dado que $\alpha_1$ es un epimorfismo, se sigue que $\gamma\beta_1 = \gamma'\beta_1$. Por lo tanto, $\gamma$ y $\gamma'$ son tales que el siguiente diagrama conmuta
    \begin{center}
        \begin{tikzcd}
            A \arrow{d}[swap]{\alpha_1} \arrow{r}{\alpha_2} &A_2 \arrow{d}{\beta_2} \arrow[bend left = 30]{ddr}{\gamma\beta_2} & \\
            A_1 \arrow{r}[swap]{\beta_1} \arrow[bend right = 30]{drr}[swap]{\gamma\beta_1} & S\arrow[shift left]{dr}{\gamma} \arrow[shift right]{dr}[swap]{\gamma'} \arrow[phantom]{ul}[]{\text{PO}} & \\
              &    &S'.
        \end{tikzcd}
    \end{center}
            Por la propiedad universal de la suma fibrada se sigue que $\gamma=\gamma'$. Por lo tanto, $\beta_2$ es un epimorfismo. Observemos que por simetría se sigue que, si $\alpha_2$ es un epimorfismo, entonces $\beta_1$ también lo es.

        \item Supongamos que $\alpha_1$ es un isomorfismo. Observemos que el siguiente diagrama en $\mathscr{C}$ conmuta
    \begin{center}
        \begin{tikzcd}
            A \arrow[]{d}[swap]{\alpha_1} \arrow[]{r}[]{\alpha_2} \arrow[phantom]{dr}[]{\text{PO}} &A_2 \arrow[]{d}[]{\beta_2} \arrow[bend left = 30]{ddr}[]{1_{A_2}} \\
            A_1 \arrow[]{r}[swap]{\beta_1} \arrow[bend right = 30]{drr}[swap]{\alpha_2\alpha_1^{-1}} &S \\
                                            &&A_2.
        \end{tikzcd}
    \end{center}
    Por la propiedad universal de la suma fibrada, existe un único morfismo $S\xrightarrow[]{\gamma} A_1$ tal que $\gamma\beta_1=\alpha_2\alpha_1^{-1}$ y $\gamma\beta_2=1_{A_2}$. Luego, tenemos que
    \begin{align*}
        \beta_2\gamma\beta_2 &= \beta_21_{A_2} \\
                             &= \beta_2 \\
                             &= 1_S\beta_2, \\ \\
        \beta_2\gamma\beta_1 &= \beta_2\alpha_2\alpha_1^{-1} \\
                             &= \beta_1\alpha_1\alpha_1^{-1} \\
                             &= \beta_11_{A_1} \\
                             &= \beta_1 \\
                             &= 1_S\beta_1.
    \end{align*}
    De la unicidad en la propiedad universal del producto fibrado se sigue que $\beta_1\gamma=1_S$, de donde concluimos que $\beta_2$ es un isomorfismo.

    \end{enumerate}
\end{proof}

\begin{Def}
    Sea $\mathscr{C}$ una categoría. Decimos que $\mathscr{C}$ es una \emph{categoría con sumas fibradas} si para cualquier diagrama $A_1\xleftarrow[]{\alpha_1}A\xrightarrow[]{\alpha_2}A_2$ en $\mathscr{C}$ existe la suma fibrada correspondiente.
\end{Def}

\begin{Def}
    Un \emph{cuadrado bicartesiano} en una categoría $\mathscr{C}$ es un diagrama conmutativo en $\mathscr{C}$
    \begin{center}
        \begin{tikzcd}
            A \arrow[]{r}[]{\alpha} \arrow[]{d}[swap]{\gamma} &B \arrow[]{d}[]{\beta} \\
            C \arrow[]{r}[swap]{\delta} &D \arrow[phantom]{ul}[]{\text{BC}}
        \end{tikzcd}
    \end{center}
    tal que es un producto fibrado y una suma fibrada.
\end{Def}

\begin{Obs}
    La noción de cuadrado bicartesiano es auto dual.
    \vspace{1mm}

    En efecto: Se sigue del inciso (2) de la Observación \ref{Obs: Sumas fibradas}.
\end{Obs}

\subsection*{Productos y coproductos} \label{Mendoza-1.8}
%\section{Productos y coproductos} \label{Mendoza-1.8}

\begin{Def} \label{Def: Producto}
    Sean $\mathscr{C}$ una categoría y $\{A_i\}_{i\in I}$ una familia de objetos en $\mathscr{C}$. Un \emph{producto} en $\mathscr{C}$ para $\{A_i\}_{i\in I}$ es un objeto $P\in\text{Obj}(\mathscr{C})$ junto con una familia $\{P\xrightarrow[]{\pi_i} A_i\}_{i\in I}$ de morfismos en $\mathscr{C}$ tales que se satisfacen la siguiente propiedad universal: para cualesquiera $Q\in\text{Obj}(\mathscr{C})$ y $\{Q\xrightarrow[]{\alpha_i} A_i\}_{i\in I}$ en $\mathscr{C}$, existe un único morfismo $Q\xrightarrow[]{\alpha} P$ en $\mathscr{C}$ tal que $\pi_i\alpha=\alpha_i$ para todo $i\in I$; diagramáticamente,
    \begin{center}
        \begin{tikzcd}
            Q \arrow[]{dr}[swap]{\alpha_i} \arrow[dotted]{rr}[]{\exists! \ \alpha} & &P \arrow[]{dl}[]{\pi_i} &\empty{} \\
                                                                                 &A_i & &\empty{} \arrow[phantom]{u}[]{\forall \ i\in I.}
        \end{tikzcd}
    \end{center}
\end{Def}

\begin{Obs} \label{Observaciones del producto}
    Sean $\mathscr{C}$ una categoría, $\{A_i\}_{i\in I}$ una familia de objetos en $\mathscr{C}$ y $\{P\xrightarrow[]{\pi_i} A_i\}_{i\in I}$ un producto para $\{A_i\}_{i\in I}$.

    \begin{enumerate}[label=(\arabic*)]

        \item Si $\{P'\xrightarrow[]{\pi_i'} A_i\}_{i\in I}$ y $P'\in\text{Obj}(\mathscr{C})$ son otro producto para $\{A_i\}_{i\in I}$, entonces existe $\lambda:P\xrightarrow[]{\sim} P'$ en $\mathscr{C}$ tal que $\pi'_i\lambda=\pi_i$ para todo $i\in I$; diagramáticamente,
            \begin{center}
                \begin{tikzcd}
                    P \arrow[]{dr}[swap]{\pi_i} \arrow[dotted]{rr}{\lambda}[swap]{\sim} &&P' \arrow[]{dl}[]{\pi'_i} & &\empty{} \\
                                                                                                 &A_i & & &\empty{} \arrow[phantom]{u}[]{\forall \ i\in I.}
                \end{tikzcd}
            \end{center}

            En efecto: Por la propiedad universal del producto, existen $P\xrightarrow[]{\lambda} P'$ y $P'\xrightarrow[]{\mu} P$ tales que el siguiente diagrama conmuta
            \begin{center}
                \begin{tikzcd}
                    P \arrow[]{dr}[swap]{\pi_i} \arrow[]{r}[]{\lambda} &P' \arrow[]{d}[]{\pi'_i} \arrow[]{r}[]{\mu} &P \arrow[]{dl}[]{\pi_i} & &\empty{} \\
                                                                       &A_i & & &\empty{} \arrow[phantom]{u}[]{\forall \ i\in I.}
                \end{tikzcd}
            \end{center}

            Luego, para todo $i\in I$ se tienen los siguientes diagramas conmutativos
            \begin{center}
                \begin{tikzcd}
                    P \arrow[]{dr}[swap]{\pi_i} \arrow[]{rr}[]{\mu\lambda} & &P \arrow[]{dl}[]{\pi_i} & &P' \arrow[]{dr}[swap]{\pi'_i} \arrow[]{rr}[]{\lambda\mu} & &P'. \arrow[]{dl}[]{\pi'_i} \\
                                                                           &A_i & & & &A_i
                \end{tikzcd}
            \end{center}
            De la unicidad en la propiedad universal del producto se sigue que $\mu\lambda = 1_P$ y $\lambda\mu = 1_{P'}$.
            
        \item En vista de (1), y en caso de que exista, denotaremos por $\{\prod_{i\in I}A_i\xrightarrow[]{\pi_i} A_i\}_{i\in I}$ a la elección de un producto en $\mathscr{C}$ para $\{A_i\}_{i\in I}$. En tal caso, el morfismo $\prod_{i\in I}A_i\xrightarrow[]{\pi_i} A_i$ se conoce como la $i$-ésima proyección natural de $\prod_{i\in I}$ en $A_i$.

        \item Sean $\mathscr{C}$ localmente pequeña e $I$ un conjunto. Una familia $\{A\xrightarrow[]{p_i} A_i\}_{i\in I}$ en $\mathscr{C}$ es un producto para $\{A_i\}_{i\in I}$ si, y sólo si, para todo $B\in\text{Obj}(\mathscr{C})$, el morfismo canónico en Sets
            \[
                \varphi_B:\text{Hom}_\mathscr{C}(B,A)\to \prod_{i\in I}\text{Hom}_\mathscr{C}(B,A_i), \beta\mapsto (p_i\beta)_{i\in I}
            \] 
            es un isomorfismo en Sets. 

        \item Si $I=\varnothing$, entonces $P$ es un objeto final en $\mathscr{C}$. Recíprocamente, cualquier objeto final en $\mathscr{C}$ es un producto en $\mathscr{C}$ para una familia vacía de objetos en $\mathscr{C}$.
            
    \end{enumerate}
\end{Obs}

\begin{Prop}\label{Mendoza-1.8.6}
    Sean $\mathscr{C}$ una categoría, $A_1,A_2\in\text{Obj}(\mathscr{C})$ tales que existe un producto $A_1\prod A_2$ en $\mathscr{C}$, y el diagrama
    \begin{equation}\label{eq: Mendoza-1.8.6-1}
        \begin{tikzcd}
            P \arrow[]{d}[swap]{\beta_1} \arrow[]{r}[]{\beta_2} &A_2 \arrow[]{d}[]{\alpha_2} \\
            A_1 \arrow[]{r}[swap]{\alpha_1} &A.
        \end{tikzcd}
    \end{equation}
    en $\mathscr{C}$. Por la propiedad universal del producto, podemos considerar un morfismo $P\xrightarrow[]{\beta}A_1\prod A_2$ tal que hace conmutar los triángulos superiores izquierdos del siguiente diagrama en $\mathscr{C}$
    \begin{equation}\label{eq: Mendoza-1.8.6-2}
        \begin{tikzcd}
            P \arrow[]{dd}[swap]{\beta_1} \arrow[]{rr}[]{\beta_2} \arrow[]{dr}[]{\beta} &&A_2 \arrow[]{dd}[]{\alpha_2} \\
                                                                                        &A_1\prod A_2 \arrow[]{dl}[]{\pi_1} \arrow[]{ur}[swap]{\pi_2} \\
            A_1 \arrow[]{rr}[swap]{\alpha_1} &&A.
        \end{tikzcd}
    \end{equation}
    Entonces, las siguientes condiciones son equivalentes.

    \begin{enumerate}[label=(\alph*)]
    
        \item El diagrama (\ref{eq: Mendoza-1.8.6-1}) es un producto fibrado en $\mathscr{C}$.

        \item $\beta = \text{Equ}(\alpha_2\pi_2,\alpha_1\pi_1)$.
    \end{enumerate}
\end{Prop}

\begin{proof}\leavevmode

    (a)$\Rightarrow$(b) Por hipótesis, tenemos que el cuadrado del diagrama (\ref{eq: Mendoza-1.8.6-2}) conmuta. Por ende,
    \begin{align*}
        (\alpha_1\pi_1)\beta &= \alpha_1(\pi_1\beta) \\
                             &= \alpha_1\beta_1 \\
                             &= \alpha_2\beta_2 \\
                             &= \alpha_2(\pi_2\beta) \\
                             &= (\alpha_2\pi_2)\beta.
    \end{align*}
    Ahora, sea $X\xrightarrow[]{\delta}A_1\prod A_2$ en $\mathscr{C}$ tal que $\alpha_1\pi_1\delta = \alpha_2\pi_2\delta$. Entonces, como el diagrama (\ref{eq: Mendoza-1.8.6-1}) es un producto fibrado, por la propiedad universal del producto fibrado tenemos que existe $X\xrightarrow[]{\theta}P$ en $\mathscr{C}$ tal que $\beta_2\theta = \pi_2\delta$ y $\beta_1\theta = \pi_1\delta$. Luego, como
    \begin{align*}
        \pi_1(\beta\theta) &= (\pi_1\beta)\theta \\
                           &= \beta_1\theta \\
                           &= \pi_1\delta, \\ \\
        \pi_2(\beta\theta) &= (\pi_2\beta)\theta \\
                           &= \beta_2\theta \\
                           &= \pi_2\delta,
    \end{align*}
    de la propiedad universal del producto, se sigue que $\beta\theta=\delta$. Supongamos que existe $X\xrightarrow[]{\theta'} P$ en $\mathscr{C}$ tal que $\beta\theta' = \delta$. Entonces, para $i\in\{1,2\}$, tenemos que
    \begin{align*}
        \beta_i\theta' &= (\pi_i\beta)\theta \\
                       &= \pi_i(\beta\theta') \\
                       &= \pi_i\delta
    \end{align*}
    y, de la propiedad universal del producto fibrado, se sigue que $\theta'=\theta$. \\

    (b)$\Rightarrow$(a) Observemos que
    \begin{align*}
        \alpha_2\beta_2 &= \alpha_2\pi_2\beta \\
                        &= \alpha_1\pi_1\beta \tag{$\beta = \text{Equ}(\alpha_1\pi_1,\alpha_2\pi_2)$} \\
                        &= \alpha_1\beta_1.
    \end{align*}
    Sean $X\xrightarrow[]{\delta_1} A_1$ y $X\xrightarrow[]{\delta_2} A_2$ en $\mathscr{C}$ tales que $\alpha_2\delta_2 = \alpha_1\delta_1$. Por la propiedad universal del producto, existe $X\xrightarrow[]{\delta} A_1\prod A_2$ en $\mathscr{\mathscr{C}}$ tal que $\pi_1\delta = \delta_1$ y $\pi_2\delta = \delta_2$. Ahora, como
    \begin{align*}
        (\alpha_1\pi_1)\delta &= \alpha_1(\pi_1\delta) \\
                              &= \alpha_1\delta_1 \\
                              &= \alpha_2\delta_2 \\
                              &= \alpha_2(\pi_2\delta) \\
                              &= (\alpha_2\pi_2)\delta
    \end{align*}
    y $\beta = \text{Equ}(\alpha_1\pi_1,\alpha_2\pi_2)$, por la propiedad universal del ecualizador, existe $X\xrightarrow[]{\psi} P$ en $\mathscr{C}$ tal que $\beta\psi=\delta$. Más aún, para $i\in\{1,2\}$, tenemos que
    \begin{align*}
        \beta_i\psi &= \pi_i\beta\psi \\
                    &= \pi_i\delta \\
                    &= \delta_i.
    \end{align*}
    Supongamos que existe $X\xrightarrow[]{\psi'}P$ en $\mathscr{C}$ tal que $\beta_i\psi' = \delta_i$, para $i\in\{1,2\}$. Dado que, para $i\in\{1,2\}$, tenemos que
    \begin{align*}
        \pi_i(\beta\psi') &= (\pi_i\beta)\psi' \\
                          &= \beta_i\psi' \\
                          &= \delta_i \\
                          &= \pi_i\delta,
    \end{align*}
    de la propiedad universal del producto se sigue que $\beta\psi' = \delta$. Finalmente, por la propiedad universal del ecualizador, tenemos que $\psi' = \psi$.
\end{proof}

\begin{Def}\label{Def: Categoría con productos finitos}
    Sea $\mathscr{C}$ una categoría. Decimos que $\mathscr{C}$ es una \emph{categoría con productos finitos} si para cualquier familia $\{A_i\}_{i\in I}$ de objetos en $\mathscr{C}$ con $I$ finito existe el producto correspondiente.
\end{Def}

\begin{Def}\label{Def: Coproducto}
    Sean $\mathscr{C}$ una categoría y $\{A_i\}_{i\in I}$ una familia de objetos en $\mathscr{C}$. Un \emph{coproducto} en $\mathscr{C}$ para $\{A_i\}_{i\in I}$ es un objeto $C$ en $\mathscr{C}$ junto con una familia $\{A_i\xrightarrow[]{\mu_i} C\}_{i\in I}$ de morfismos en $\mathscr{C}$ tales que satisfacen la siguiente propiedad universal: para cualesquiera $B\in\text{Obj}(\mathscr{C})$ y $\{A_i\xrightarrow[]{\beta_i} B\}_{i\in I}$ en $\mathscr{C}$, existe un único morfismo $C\xrightarrow[]{\beta} B$ en $\mathscr{C}$ tal que $\beta\mu_i = \beta_i$ para todo $i\in I$; diagramáticamente,
    \begin{center}
        \begin{tikzcd}
            &A_i \arrow[]{dl}[swap]{\mu_i} \arrow[]{dr}[]{\beta_i} & &\empty{} \\
            C \arrow[dotted]{rr}[swap]{\exists! \ \beta} & &B &\empty{} \arrow[phantom]{u}[]{\forall \ i\in I.}
        \end{tikzcd}
    \end{center}
\end{Def}

\begin{Obs} \label{Mendoza-1.8.3}
    Sean $\mathscr{C}$ una categoría, $\{A_i\}_{i\in I}$ una familia de objetos en $\mathscr{C}$, $C\in\text{Obj}(\mathscr{C})$ y $\{A_i\xrightarrow[]{\mu_i} C\}_{i\in I}$ un coproducto en $\mathscr{C}$ para $\{A_i\}_{i\in I}$.

    \begin{enumerate}[label=(\arabic*)]

        \item Si $\{A_i\xrightarrow[]{\mu'_i} C'\}_{i\in I}$ y $C'\in\text{Obj}(\mathscr{C})$ son otro coproducto en $\mathscr{C}$ para $\{A_i\}_{i\in I}$, entonces existe $\theta:C'\xrightarrow[]{\sim} C$ en $\mathscr{C}$ tal que $\theta\mu'_i = \mu_i$ para todo $i\in I$; diagramáticamente,
            \begin{center}
                \begin{tikzcd}
                    C' \arrow[dotted]{rr}[]{\theta} \arrow[dotted]{rr}[swap]{\sim} & &C &\empty{} \\
                                                                                   &A_i \arrow[]{ul}[]{\mu'_i} \arrow[]{ur}[swap]{\mu_i} & &\empty{} \arrow[phantom]{u}[]{\forall \ i\in I.}
                \end{tikzcd}
            \end{center}

            En efecto: Por la propiedad universal del coproducto, existen $C'\xrightarrow[]{\theta} C$ y $C\xrightarrow[]{\nu} C'$ tales que el siguiente diagrama en conmuta
            \begin{center}
                \begin{tikzcd}
                    C' \arrow[]{r}[swap]{\theta} &C \arrow[]{r}[swap]{\nu} &C' &\empty{} \\
                                             &A_i \arrow[]{ul}[]{\mu_i'} \arrow[]{u}[]{\mu_i} \arrow[]{ur}[swap]{\mu_i'} &&\empty{} \arrow[phantom]{u}[]{\forall \ i\in I.}
                \end{tikzcd}
            \end{center}
            Luego, para todo $i\in I$, se tienen los siguientes diagramas conmutativos
            \begin{center}
                \begin{tikzcd}
                    C' \arrow[]{rr}[]{\nu\theta} &&C' &&C \arrow[]{rr}[]{\theta\nu} &&C. \\
                                                 &A_i \arrow[]{ul}[]{\mu_i'} \arrow[]{ur}[swap]{\mu_i'} &&&&A_i \arrow[]{ul}[]{\mu_i} \arrow[]{ur}[swap]{\mu_i}
                \end{tikzcd}
            \end{center}
            De la unicidad en la propiedad universal del coproducto se sigue que $\nu\theta=1_{C'}$ y $\theta\nu=1_C$.

        \item En vista de (1), y en caso de que exista, denotaremos por $\{A_i\xrightarrow[]{\mu_i} \coprod_{i\in I}A_i\}_{i\in I}$ a la elección de un coproducto en $\mathscr{C}$ para $\{A_i\}_{i\in I}$. En tal caso, el morfismo $A_i\xrightarrow[]{\mu_i} \coprod_{i\in I}A_i$ se conoce como la $i$-ésima inclusión natural de $A_i$ en $\coprod_{i\in I}A_i$.
            
        \item Sean $\mathscr{C}$ localmente pequeña e $I$ un conjunto. Una familia $\{A_i\xrightarrow[]{\nu_i} B\}_{i\in I}$ y $B\in\text{Obj}(\mathscr{C})$ son un coproducto de $\{A_i\}_{i\in I}$ si, y sólo si, para todo $H\in\text{Obj}(\mathscr{C})$, el morfismo canónico en Sets
            \[
                \psi_H:\text{Hom}_\mathscr{C}(B,H)\to \prod_{i\in I}\text{Hom}_\mathscr{C}(A_i,H), \beta\mapsto (\beta\nu_i)_{i\in I}
            \] 
            es un isomorfismo en Sets. En particular, 
            \[
            \psi_H:\text{Hom}_\mathscr{C}\bigg(\coprod_{i\in I}A_i,H\bigg) \xrightarrow[]{\sim} \prod_{i\in I}\text{Hom}_\mathscr{C}(A_i,H)
            \] 
            en $\text{Sets}$ para todo $H\in\text{Obj}(\mathscr{C})$.

        \item Si $I=\varnothing$, entonces $C$ es un objeto inicial en $\mathscr{C}$. Recíprocamente, cualquier objeto inicial en $\mathscr{C}$ es un coproducto en $\mathscr{C}$ para una familia vacía de objetos en $\mathscr{C}$.

        \item Las nociones de producto y coproducto son duales entre sí. Es decir, si $\mathscr{C}$ es una categoría, $\{A_i\}_{i\in I}$ es una familia de objetos en $\mathscr{C}$, $P$ es un objeto en $\mathscr{C}$ y $\pi$ es una familia $\{P\xrightarrow[]{\pi_i}A_i\}_{i\in I}$ de morfismos en $\mathscr{C}$, entonces %$\{P\xrightarrow[]{\pi_i}A_i\}_{i\in I}$ es un producto para $\{A_i\}_{i\in I}$ en $\mathscr{C}$ si, y sólo si, $\{A_i\xrightarrow[]{\pi_i{}^\text{op}}P\}_{i\in I}$ es un coproducto para $\{A_i\}_{i\in I}$ en $\mathscr{C}^\text{op}$.
            \[
                P \text{ y } \pi \text{ son un producto en } \mathscr{C} \text{ para } \{A_i\}_{i\in I} \iff P \text{ y } \pi^\text{op} \text{ son un coproducto en } \mathscr{C}^\text{op} \text{ para } \{A_i\}_{i\in I},
            \] 
            donde $\pi^\text{op} = \{A_i\xrightarrow[]{\pi_i{}^\text{op}}P\}_{i\in I}$.
    \end{enumerate}
\end{Obs}

\begin{Prop}\label{Mendoza-Ejer.42}
    Sean $\mathscr{C}$ una categoría, $\{A_i\xrightarrow[]{\mu_i} \coprod_{i\in I}A_i\}_{i\in I}$ un coproducto en $\mathscr{C}, \ C\in\text{Obj}(\mathscr{C})$ y $\{A_i\xrightarrow[]{\nu_i} C\}_{i\in I}$ una familia de morfismos en $\mathscr{C}$. Entonces, las siguientes condiciones son equivalentes.

    \begin{enumerate}[label=(\alph*)]
    
        \item $C$ y $\{A_i\xrightarrow[]{\nu_i} C\}_{i\in I}$ son un coproducto de $\{A_i\}_{i\in I}$.

        \item Existe un isomorfismo $\varphi:\coprod_{i\in I}A_i\xrightarrow[]{\sim}C$ tal que $\varphi\mu_i=\nu_i$ para todo $i\in I$.
    \end{enumerate}
\end{Prop}

\begin{proof}

    Si $I=\varnothing$, entonces por el inciso (1) de la Observación \ref{Obs: Objetos iniciales} y el inciso (4) de la Observación \ref{Mendoza-1.8.3} se sigue que (a) y (b) son equivalentes, por lo que podemos suponer que $I\neq\varnothing$. \\

    $(a)\Rightarrow(b)$ Supongamos que $C$ y $\{A_i\xrightarrow[]{\nu_i} C\}_{i\in I}$ son un coproducto de $\{A_i\}_{i\in I}$. Entonces, por la propiedad universal del coproducto, se sigue que existe un único morfismo $C\xrightarrow[]{\psi} \coprod_{i\in I}A_i$ tal que $\psi\nu_i=\mu_i$ para cada $i\in I$. Como por hipótesis $\coprod_{i\in I}A_i$ y $\{A_i\xrightarrow[]{\mu_i} \coprod_{i\in I}A_i\}_{i\in I}$ también son un coproducto de $\{A_i\}_{i\in I}$, entonces existe un único morfismo $\coprod_{i\in I}A_i\xrightarrow[]{\varphi} C$ tal que $\varphi\mu_i=\nu_i$ para cada $i\in I$. Observemos que, para cada $i\in I$,
    \begin{align*}
        (\psi\varphi)\mu_i &= \psi(\varphi\mu_i) \\
                           &= \psi\nu_i \\
                           &= \mu_i \\
                           &= 1_{\coprod_{i\in I}A_i}\mu_i, \\ \\
        (\varphi\psi)\nu_i &= \varphi(\psi\nu_i) \\
                           &= \varphi\mu_i \\
                           &= \nu_i \\
                           &= 1_C\nu_i.
    \end{align*}
    De la unicidad en la propiedad universal del coproducto, se sigue que $\psi\varphi=1_{\coprod_{i\in I}A_i}$ y $\varphi\psi=1_C$. Por ende, existe $\varphi:\coprod_{i\in I}A_i\xrightarrow[]{\sim}C$ tal que $\varphi\mu_i=\nu_i$ para cada $i\in I$. \\

    $(b)\Rightarrow(a)$ Supongamos que existe $\varphi:\coprod_{i\in I}A_i\xrightarrow[]{\sim}C$ tal que $\varphi\mu_i=\nu_i$, para cada $i\in I$. Sean $B\in\text{Obj}(\mathscr{C})$ y $\{A_i\xrightarrow[]{\beta_i} B\}_{i\in I}$ una familia de morfismos en $\mathscr{C}$. Como por hipótesis $\coprod_{i\in I}A_i$ y $\{A_i\xrightarrow[]{\mu_i} \coprod_{i\in I}A_i\}_{i\in I}$ son un coproducto de $\{A_i\}_{i\in I}$, entonces existe un único morfismo $\coprod_{i\in I}A_i\xrightarrow[]{\beta} B$ tal que $\beta\mu_i=\beta_i$ para cada $i\in I$. Observemos que $C\xrightarrow[]{\beta\varphi^{-1}} B$ es un morfismo en $\mathscr{C}$ tal que, para cada $i\in I$,
    \begin{align*}
        (\beta\varphi^{-1})\nu_i &= (\beta\varphi^{-1})(\varphi\mu_i) \\
                                 &= \beta(\varphi^{-1}\varphi)\mu_i \\
                                 &= \beta\mu_i \\
                                 &= \beta_i.
    \end{align*}
    Más aún, de la unicidad de la propiedad universal del coproducto, se sigue que $\beta\varphi^{-1}$ es el único morfismo de $C$ a $B$ que cumple lo anterior. Por ende, $C$ y $\{A_i\xrightarrow[]{\nu_i} C\}_{i\in I}$ son un coproducto de $\{A_i\}_{i\in I}$.
\end{proof}

\begin{Prop}\label{Mendoza-Ejer.42*}
    Sean $\mathscr{C}$ una categoría, $\{\prod_{i\in I}A_i\xrightarrow[]{\pi_i} A_i\}_{i\in I}$ un producto en $\mathscr{C}, C\in\text{Obj}(\mathscr{C})$ y $\{C\xrightarrow[]{\rho_i} A_i\}_{i\in I}$ una familia de morfismos en $\mathscr{C}$. Entonces, las siguientes condiciones son equivalentes.

    \begin{enumerate}[label=(\alph*)]
    
        \item $C$ y $\{C\xrightarrow[]{\rho_i} A_i\}_{i\in I}$ son un producto de $\{A_i\}_{i\in I}$.

        \item Existe un isomorfismo $\varphi:C\xrightarrow[]{\sim}\prod_{i\in I}A_i$ tal que $\pi_i\varphi=\rho_i$ para todo $i\in I$.
    \end{enumerate}
\end{Prop}

\begin{proof}

    Haremos esta demostración utilizando el principio de dualidad. Observemos que, por el inciso (5) de la Observación \ref{Mendoza-1.8.3}, tenemos que $\{A_i\xrightarrow[]{\pi_i{}^\text{op}} \prod_{i\in I}A_i\}_{i\in I}$ es un coproducto en $\mathscr{C}^\text{op}$. \\

    (a)$\Rightarrow$(b) Supongamos que $C$ y $\{C\xrightarrow[]{\rho_i} A_i\}_{i\in I}$ son un producto de $\{A_i\}_{i\in I}$ en $\mathscr{C}$. Entonces, por dualidad, tenemos que $C$ y $\{A_i\xrightarrow[]{\rho_i{}^\text{op}} C\}_{i\in I}$ son un coproducto de $\{A_i\}_{i\in I}$ en $\mathscr{C}^\text{op}$. Aplicando la Proposición \ref{Mendoza-Ejer.42}, tenemos que existe un isomorfismo $\varphi^\text{op}:\prod_{i\in I}A_i\xrightarrow[]{\sim}C$ tal que $\varphi^\text{op}\pi_i{}^\text{op}=\rho_i{}^\text{op}$ para todo $i\in I$. Luego, tenemos que
    \begin{align*}
        \varphi^\text{op}\pi_i{}^\text{op}=\rho_i{}^\text{op} \quad \forall \ i\in I &\implies D_{\mathscr{C}^\text{op}}(\varphi^\text{op}\pi_i{}^\text{op}) = D_{\mathscr{C}^\text{op}}(\rho_i{}^\text{op}) \quad \forall \ i\in I \\
                                                                                     &\implies D_{\mathscr{C}^\text{op}}(\pi_i^\text{op})D_{\mathscr{C}^\text{op}}(\varphi^\text{op}) = D_{\mathscr{C}^\text{op}}(\rho_i{}^\text{op}) \quad \forall \ i\in I \\
                                                                                     &\implies \pi_i\varphi = \rho_i \quad \forall \ i\in I,
    \end{align*}
    donde $\varphi:C\xrightarrow[]{\sim}\prod_{i\in I}A_i$ es un isomorfismo por el inciso (1) de la Observación \ref{Obs: Funtores}. \\

    (b)$\Rightarrow$(a) Supongamos que existe un isomorfismo $\varphi:C\xrightarrow[]{\sim}\prod_{i\in I}A_i$ tal que $\pi_i\varphi=\rho_i$ para todo $i\in I$. Observemos que
    \begin{align*}
        \pi_i\varphi=\rho \quad \forall \ i\in I &\implies D_\mathscr{C}(\pi_i\varphi) = D_\mathscr{C}(\rho_i) \quad \forall \ i\in I \\
                                                 &\implies D_\mathscr{C}(\varphi)D_\mathscr{C}(\pi_i) = D_\mathscr{C}(\rho_i) \quad i\in I \\
                                                 &\implies \varphi^\text{op}\pi_i{}^\text{op} = \rho_i{}^\text{op} \quad \forall \ i\in I,
    \end{align*}
    donde $\varphi^\text{op}:\prod_{i\in I}A_i\xrightarrow[]{\sim}C$ es un isomorfismo por el inciso (1) de la Observación \ref{Obs: Funtores}. Aplicando la Proposición \ref{Mendoza-Ejer.42}, tenemos que $C$ y $\{A_i\xrightarrow[]{\rho_i{}^\text{op}} C\}_{i\in I}$ son un coproducto de $\{A_i\}_{i\in I}$ en $\mathscr{C}^\text{op}$. Por el inciso (5) de la Observación \ref{Mendoza-1.8.3}, se sigue que $C$ y $\{C\xrightarrow[]{\rho_i} A_i\}_{i\in I}$ son un producto de $\{A_i\}_{i\in I}$ en $\mathscr{C}$.
\end{proof}


\begin{Def}\label{Def: Mat}
    Sean $\mathscr{C}$ una categoría localmente pequeña y $A=\coprod_{j\in J}A_j, B=\prod_{i\in I}B_i$ en $\mathscr{C}$, con $I$ y $J$ conjuntos. Denotaremos por $\text{Mat}_{I\times J}(A,B)$ al conjunto de matrices $\alpha$, de orden $I\times J$, con entradas $[\alpha]_{i,j}\in\text{Hom}_\mathscr{C}(A_j,B_i)$. Para $\alpha,\beta\in\text{Mat}_{I\times J}(A,B)$, definimos
    \[
        \alpha = \beta \iff [\alpha]_{i,j} = [\beta]_{i,j} \quad \forall \ (i,j)\in I\times J.
    \] 
\end{Def}

\begin{Prop}\label{Mendoza-1.8.11}
    Sean $\mathscr{C}$ una categoría localmente pequeña y $A=\coprod_{j\in J}A_j, B=\prod_{i\in I}B_i$ en $\mathscr{C}$, con $I$ y $J$ conjuntos. Entonces, la correspondencia
    \[
        \varphi = \varphi_{B,A}:\text{Hom}_\mathscr{C}(A,B) \to \text{Mat}_{I\times J}(A,B),
    \]
    dada por $[\varphi(f)]_{i,j} := \pi_i^Bf\mu_j^A$ para todo $(i,j)\in I\times J$, es un isomorfismo en Sets.
\end{Prop}

\begin{proof}

    Primero, veamos que $\varphi$ es inyectiva. Sean $f,g\in\text{Hom}_\mathscr{C}(A,B)$ tales que $\varphi(f)=\varphi(g)$. Para cada $i\in I$, tenemos que, para todo $j\in J$,
    \begin{align*}
        (\pi_i^Bf)\mu_j^A &= [\varphi(f)]_{i,j} \\
                          &= [\varphi(g)]_{i,j} \\
                          &= (\pi_i^Bg)\mu_j^A.
    \end{align*}
    Luego, por la propiedad universal del coproducto, tenemos que
    \[
    \pi_i^Bf = \pi_i^Bg \quad \forall \ i\in I;
    \] 
    de la unicidad en la propiedad universal del producto, se sigue que $f=g$. \\
    
    Ahora, veamos que $\varphi$ es suprayectiva. Sea $\alpha\in \text{Mat}_{I\times J}(A,B)$. Para cada $j\in J$ fijo, se tiene la familia de morfismos $\{A_j\xrightarrow[]{[\alpha]_{i,j}} B_i\}_{i\in I}$. Luego, por la propiedad universal del producto, para cada $j\in J$, existe un morfismo $A_j\xrightarrow[]{f_j} B$ en $\mathscr{C}$ tal que $\pi_i^Bf_j=[\alpha]_{i,j}$ para todo $i\in I$. Además, por la propiedad universal del coproducto, existe un morfismo $A\xrightarrow[]{f} B$ en $\mathscr{C}$ tal que $f\mu_j^A=f_j$ para todo $i\in I$. Diagramáticamente, tenemos que
    \begin{center}
        \begin{tikzcd}
            A \arrow[]{r}[]{f} &B \arrow[]{d}[]{\pi_i^B} &&\empty{} \\
            A_j \arrow[]{u}[]{\mu_j^A} \arrow[]{ur}[swap]{f_j} \arrow[]{r}[swap]{[\alpha]_{i,j}} &B_i &&\empty{} \arrow[phantom]{u}[]{\forall \ (i,j)\in I\times J.}
        \end{tikzcd}
    \end{center}
    Por lo tanto, $[\varphi(f)]_{i,j} = \pi_i^Bf\mu_j^A = [\alpha]_{i,j}$ para todo $(i,j)\in I\times J$, por lo que $\varphi(f)=\alpha$.
\end{proof}

\begin{Def}\label{Def: Categoría con coproductos finitos}
    Sea $\mathscr{C}$ una categoría. Decimos que $\mathscr{C}$ es una \emph{categoría con coproductos finitos} si para cualquier familia $\{A_i\}_{i\in I}$ de objetos en $\mathscr{C}$, con $I$ finito, existe el coproducto correspondiente.
\end{Def}

\begin{Def}\label{Def: Morfismo diagonal y codiagonal}
    Sea $\mathscr{C}$ una categoría localmente pequeña con productos y coproductos finitos. Para cada $A\in\text{Obj}(\mathscr{C})$, se definen los morfismos:

    \begin{enumerate}[label=(\alph*)]
    
        \item Diagonal $\Delta_A:A\to A\prod A$, con $\varphi(\Delta_A) := \big( \begin{smallmatrix} 1_A \\ 1_A \end{smallmatrix} \big)$;

        \item Codiagonal $\nabla_A:A\coprod A\to A$, con $\varphi(\nabla_A) := (\begin{smallmatrix} 1_A &1_A \end{smallmatrix})$.
    \end{enumerate}
\end{Def}

%\section{Objetos proyectivos e inyectivos} \label{Sec: Objetos proyectivos e inyectivos}
%
%\begin{Def}\label{Def: Objeto proyectivo}
%    Sea $\mathscr{C}$ una categoría. Un objeto $P$ en $\mathscr{C}$ es \emph{proyectivo} si para todo diagrama en $\mathscr{C}$ de la forma
%    \begin{center}
%        \begin{tikzcd}
%            &P \arrow[]{d}[]{f} \\
%            X \arrow[two heads]{r}[swap]{g} &Y
%        \end{tikzcd}
%    \end{center}
%    se tiene que $f$ se factoriza a través de $g$; esto es
%    \begin{equation}\label{eq: Proyectivo}
%        \begin{tikzcd}
%            &P \arrow[]{d}[]{f} \arrow[dotted]{dl}[swap]{\exists \ f'} \\
%            X \arrow[two heads]{r}[swap]{g} &Y.
%        \end{tikzcd}
%    \end{equation}
%    Denotaremos por $\text{Proj}(\mathscr{C})$ a la clase de todos los objetos proyectivos en $\mathscr{C}$, siguiendo la notación preponderante derivada del término \emph{projective} en inglés.
%\end{Def}
%
%%El siguiente resultado es una caracterización de objetos proyectivos en una categoría arbitraria.
%
%%\begin{Prop}\label{Prop: Caracterización de objetos proyectivos}
%%    Sea $\mathscr{C}$ una categoría. Entonces $P\in\text{Obj}(\mathscr{C})$ es proyectivo si, y sólo si, el funtor Hom-covariante $\text{Hom}_\mathscr{C}(P,-)$ preserva epimorfismos.\footnote{Revisar a qué se refiere que ``preserve epimorfismos'' en MacLane (1978) p.118. Pista: es que manda epimorfismos en $\mathscr{C}$ en epimorfismos en Sets.}.
%%\end{Prop}
%%
%%\begin{proof}
%%
%%\end{proof}
%
%\begin{Lema}\label{Mendoza-1.10.19}
%    Sean $\mathscr{C}$ una categoría y $P\xhookrightarrow{\alpha} Q$ un monomorfismo escindible en $\mathscr{C}$, con $Q$ proyectivo. Entonces, $P$ es proyectivo.
%\end{Lema}
%
%\begin{proof}
%
%    Sea $Q\xrightarrow[]{\beta} P$ tal que $\beta\alpha=1_P$. Consideremos un diagrama en $\mathscr{C}$ de la forma
%    \begin{center}
%        \begin{tikzcd}
%            &P \arrow[]{d}[]{f} \\
%            X \arrow[two heads]{r}[swap]{g} &Y.
%        \end{tikzcd}
%    \end{center}
%    Como $Q\in\text{Proj}(\mathscr{C})$, tenemos el siguiente diagrama conmutativo en $\mathscr{C}$
%    \begin{center}
%        \begin{tikzcd}
%            Q \arrow[dotted]{d}[swap]{\exists \ f'} \arrow[]{r}[]{\beta} &P \arrow[]{d}[]{f} \\
%            X \arrow[two heads]{r}[swap]{g} &Y.
%        \end{tikzcd}
%    \end{center}
%    Veamos que el diagrama en $\mathscr{C}$
%    \begin{center}
%        \begin{tikzcd}
%            &P \arrow[]{d}[]{f} \arrow[]{dl}[swap]{f'\alpha} \\
%            X \arrow[two heads]{r}[swap]{g} &Y.
%        \end{tikzcd}
%    \end{center}
%    conmuta. En efecto,
%    \begin{align*}
%        g(f'\alpha) &= (gf')\alpha \\
%                    &= (f\beta)\alpha \\
%                    &= f(\beta\alpha) \\
%                    &= f1_P \\
%                    &= f.
%    \end{align*}
%\end{proof}
%
%\begin{Prop}\label{Mendoza-1.10.21}
%    Sean $\mathscr{C}$ una categoría y $\{P_i\}_{i\in I}$ una familia de objetos en $\mathscr{C}$ tal que existe el coproducto $\coprod_{i\in I}P_i$ en $\mathscr{C}$. Entonces,
%    \[
%    \coprod_{i\in I}P_i\in \text{Proj}(\mathscr{C}) \iff P_i\in \text{Proj}(\mathscr{C}) \quad \forall \ i\in I.
%    \] 
%\end{Prop}
%
%\begin{proof}
%    Sean $\{P_i\xrightarrow[]{\mu_i} \coprod_{i\in I}P_i\}_{i\in I}$ las inclusiones naturales y $\{\coprod_{i\in I}P_i\xrightarrow[]{\pi_i} P_i\}_{i\in I}$ las proyecciones naturales. \\
%
%    $(\Rightarrow)$ Sea $i\in I$. Dado que $\pi_i\mu_i=1_{P_i}$ y $\coprod_{i\in I}P_i\in\text{Proj}(\mathscr{C})$, por el Lema \ref{Mendoza-1.10.19} tenemos que $P_i\in\text{Proj}(\mathscr{C})$. \\
%
%    $(\Leftarrow)$ Consideremos un diagrama en $\mathscr{C}$ de la forma
%    \begin{center}
%        \begin{tikzcd}
%            &\coprod_{i\in I}P_i \arrow[]{d}[]{f} \\
%            X \arrow[two heads]{r}[swap]{g} &Y.
%        \end{tikzcd}
%    \end{center}
%    Luego, para cada $i\in I$, de $P_i\in\text{Proj}(\mathscr{C})$ se sigue que
%    \begin{center}
%        \begin{tikzcd}
%            P_i \arrow[dotted]{d}[swap]{\exists \ f_i} \arrow[]{r}[]{\mu_i} &\coprod_{i\in I}P_i \arrow[]{d}[]{f} \\
%            X \arrow[two heads]{r}[swap]{g} &Y.
%        \end{tikzcd}
%    \end{center}
%    Ahora, considerando la familia $\{P_i\xrightarrow[]{f_i} X\}_{i\in I}$ y usando la propiedad universal del coproducto, tenemos que el siguiente diagrama conmuta
%    \begin{center}
%        \begin{tikzcd}
%            P_i \arrow[]{rr}[]{\mu_i} \arrow[]{dr}[swap]{f_i} &&\coprod_{i\in I}P_i \arrow[dotted]{dl}[]{\exists \ f'} \\
%                                                              &X
%        \end{tikzcd}
%        $\forall \ i\in I$.
%    \end{center}
%    Luego, como para cualquier $i\in I$ tenemos que
%    \begin{align*}
%        (gf')\mu_i &= g(f'\mu_i) \\
%                   &= gf_i \\
%                   &= f\mu_i,
%    \end{align*}
%    de la propiedad universal del producto se sigue que $gf'=f$. Por ende, concluimos que $\coprod_{i\in I}P_i\in\text{Proj}(\mathscr{C})$.
%\end{proof}
%
%%\begin{Prop}\label{Mendoza-1.10.20}
%%    Para una categoría abeliana $\mathscr{A}$ y $P\in\text{Obj}(\mathscr{A})$, las siguientes condiciones son equivalentes.
%%
%%    \begin{enumerate}[label=(\alph*)]
%%    
%%        \item $P\in\text{Proj}(\mathscr{A})$.
%%
%%        \item Todo epimorfismo $\alpha:X\twoheadrightarrow P$ en $\mathscr{A}$ es escindible.
%%    \end{enumerate}
%%\end{Prop}
%%
%%\begin{proof}
%%    $(a)\Rightarrow(b)$ Sea \begin{tikzcd} X \arrow[two heads]{r}[]{\alpha} &Y \end{tikzcd}. Luego, por ser $P$ proyectivo, se tiene
%%    \begin{center}
%%        \begin{tikzcd}
%%            &P \arrow[]{d}[]{1_P} \arrow[dotted]{dl}[swap]{\exists \ \alpha'} \\
%%            X \arrow[two heads]{r}[swap]{\alpha} &P,
%%        \end{tikzcd}
%%    \end{center}
%%    es decir, $\alpha\alpha'=1_P$. Por ende, $\alpha$ es un epimorfismo escindible. \\
%%
%%    $(b)\Rightarrow(a)$ Consideremos el diagrama en $\mathscr{A}$
%%    \begin{center}
%%        \begin{tikzcd}
%%           &P \arrow[]{d}[]{g} \\
%%            M \arrow[two heads]{r}[swap]{f} &N.
%%        \end{tikzcd}
%%    \end{center}
%%    Como $\mathscr{A}$ tiene productos fibrados, existe el producto fibrado de $f$ y $g$
%%    \begin{center}
%%        \begin{tikzcd}
%%            X \arrow[]{d}[swap]{g'} \arrow[]{r}[]{f'} &P \arrow[]{d}[]{g} \\
%%            M \arrow[two heads]{r}[swap]{f} &N.
%%        \end{tikzcd}
%%    \end{center}
%%    Ahora, como $f$ es un epimorfismo, por el inciso a de ?? \ref{Mendoza-1.10.15} tenemos que $f':X\to P$ es un epimorfismo. Entonces, por hipótesis, $f'$ es un epimorfismo escindible y por el inciso (b) de \ref{Mendoza-1.10.15} tenemos que $f'$ se factoriza a través de $f$. Por ende, se sigue que $P$ es proyectivo.
%%\end{proof}
%
%\begin{Def}\label{Def: Suficientes proyectivos}
%    Una categoría $\mathscr{C}$ tiene \emph{suficientes proyectivos} si para cualquier $X\in\text{Obj}(\mathscr{C})$ existe un epimorfismo $P\twoheadrightarrow X$ con $P\in\text{Proj}(\mathscr{C})$.
%\end{Def}
%
%\begin{Def}\label{Def: Objeto inyectivo}
%    Sea $\mathscr{C}$ una categoría. Un objeto $Q$ en $\mathscr{C}$ es \emph{inyectivo} si para todo diagrama en $\mathscr{C}$ de la forma
%    \begin{center}
%        \begin{tikzcd}
%            X \arrow[hook]{r}[]{g} \arrow[]{d}[swap]{f} &Y \\
%            Q
%        \end{tikzcd}
%    \end{center}
%    se tiene que $f$ se factoriza a través de $g$; esto es
%    \begin{equation}\label{eq: Inyectivo}
%        \begin{tikzcd}
%            X \arrow[]{d}[swap]{f} \arrow[hook]{r}[]{g} &Y \arrow[dotted]{dl}[]{\exists \ f'} \\
%            Q.
%        \end{tikzcd}
%    \end{equation}
%    Denotaremos por $\text{Inj}(\mathscr{C})$ a la clase de todos los objetos inyectivos en $\mathscr{C}$, siguiendo la notación preponderante derivada del término \emph{injective} en inglés.
%\end{Def}
%
%\begin{Obs}\label{Obs: Objetos inyectivos}
%    Si invertimos el sentido de las flechas en el diagrama (\ref{eq: Proyectivo}) y reemplazamos el epimorfismo por un monomorfismo \textemdash su noción dual\textemdash, obtenemos el diagrama (\ref{eq: Inyectivo}). Haciendo el proceso análogo con el diagrama (\ref{eq: Inyectivo}), se obtiene el diagrama (\ref{eq: Proyectivo}). Esto muestra que las nociones de objeto proyectivo y objeto inyectivo son duales entre sí. Es decir, si $\mathscr{C}$ es una categoría y $Q\in\text{Obj}(\mathscr{C})$, entonces
%    \[
%    Q\in\text{Proj}(\mathscr{C}) \iff Q\in\text{Inj}(\mathscr{C}^\text{op}).
%    \] 
%    
%\end{Obs}
%
%%El siguiente resultado es una caracterización de objetos proyectivos en una categoría abeliana.
%
%%\begin{Prop}\label{Prop: Caracterización de objetos proyectivos}
%%    Sea $\mathscr{C}$ una categoría. Entonces $Q\in\text{Obj}(\mathscr{C})$ es inyectivo si, y sólo si, el funtor Hom-contravariante $\text{Hom}_\mathscr{C}(-,Q):\mathscr{C}\to \text{Sets}$ manda monomorfismos en $\mathscr{C}$ en epimorfismos en Sets\footnote{Revisar MacLane (1978) p.118.}.
%%\end{Prop}
%%
%%\begin{proof}
%%
%%\end{proof}
%
%En vista de la Observación anterior, las demostraciones de los siguientes dos resultados se siguen de aplicar el principio de dualidad al Lema \ref{Mendoza-1.10.19} y la Proposición \ref{Mendoza-1.10.21}, respectivamente.
%
%\begin{Lema}\label{Mendoza-Ej.69}
%    Sean $\mathscr{C}$ una categoría y $Q\xtwoheadrightarrow[]{\beta} M$ un epimorfismo escindible en $\mathscr{C}$, con $Q$ inyectivo. Entonces $M$ es inyectivo.
%\end{Lema}
%
%\begin{Prop}\label{Mendoza-Ej.71}
%    Sean $\mathscr{C}$ una categoría y $\{Q_i\}_{i\in I}$ una familia de objetos en $\mathscr{C}$ tal que existe el producto $\prod_{i\in I}Q_i$ en $\mathscr{C}$. Entonces,
%    \[
%    \prod_{i\in I}Q_i\in \text{Inj}(\mathscr{C}) \iff Q_i\in \text{Inj}(\mathscr{C}) \quad \forall \ i\in I.
%    \] 
%\end{Prop}
%
%%\begin{Prop}\label{Mendoza-1.10.20}
%%    Para una categoría abeliana $\mathscr{A}$ y $P\in\text{Obj}(\mathscr{A})$, las siguientes condiciones son equivalentes.
%%
%%    \begin{enumerate}[label=(\alph*)]
%%    
%%        \item $P\in\text{Proj}(\mathscr{A})$.
%%
%%        \item Todo epimorfismo $\alpha:X\twoheadrightarrow P$ en $\mathscr{A}$ es escindible.
%%    \end{enumerate}
%%\end{Prop}
%%
%%\begin{proof}
%%    $(a)\Rightarrow(b)$ Sea \begin{tikzcd} X \arrow[two heads]{r}[]{\alpha} &Y \end{tikzcd}. Luego, por ser $P$ proyectivo, se tiene
%%    \begin{center}
%%        \begin{tikzcd}
%%            &P \arrow[]{d}[]{1_P} \arrow[dotted]{dl}[swap]{\exists \ \alpha'} \\
%%            X \arrow[two heads]{r}[swap]{\alpha} &P,
%%        \end{tikzcd}
%%    \end{center}
%%    es decir, $\alpha\alpha'=1_P$. Por ende, $\alpha$ es un epimorfismo escindible. \\
%%
%%    $(b)\Rightarrow(a)$ Consideremos el diagrama en $\mathscr{A}$
%%    \begin{center}
%%        \begin{tikzcd}
%%           &P \arrow[]{d}[]{g} \\
%%            M \arrow[two heads]{r}[swap]{f} &N.
%%        \end{tikzcd}
%%    \end{center}
%%    Como $\mathscr{A}$ tiene productos fibrados, existe el producto fibrado de $f$ y $g$
%%    \begin{center}
%%        \begin{tikzcd}
%%            X \arrow[]{d}[swap]{g'} \arrow[]{r}[]{f'} &P \arrow[]{d}[]{g} \\
%%            M \arrow[two heads]{r}[swap]{f} &N.
%%        \end{tikzcd}
%%    \end{center}
%%    Ahora, como $f$ es un epimorfismo, por el inciso a de ?? \ref{Mendoza-1.10.15} tenemos que $f':X\to P$ es un epimorfismo. Entonces, por hipótesis, $f'$ es un epimorfismo escindible y por el inciso (b) de \ref{Mendoza-1.10.15} tenemos que $f'$ se factoriza a través de $f$. Por ende, se sigue que $P$ es proyectivo.
%%\end{proof}
%
%\begin{Def}\label{Def: Suficientes inyectivos}
%    Una categoría $\mathscr{C}$ tiene \emph{suficientes inyectivos} si para cualquier $X\in\text{Obj}(\mathscr{C})$ existe un monomorfismo $X\hookrightarrow Q$ con $Q\in\text{Inj}(\mathscr{C})$.
%\end{Def}

\section{Transformaciones naturales} \label{Sec: Transformaciones naturales}

\begin{Def}\label{Def: Transformación natural}
    Sean $F,G:\mathscr{C}\to \mathscr{D}$ funtores. Un \emph{morfismo de funtores} o \emph{transformación natural} $\eta:F\to G$ es una familia de morfismos $\eta:=\{F(C)\xrightarrow[]{\eta_C} G(C)\}_{C\in\text{Obj}(\mathscr{C})}$ en $\mathscr{D}$ tal que, para todo morfismo $X\xrightarrow[]{f} Y$ en $\mathscr{C}$, se tiene que $G(f)\circ\eta_X = \eta_Y\circ F(f)$; diagramáticamente,
    \begin{center}
        \begin{tikzcd}
            F(X) \arrow[]{d}[swap]{F(f)} \arrow[]{r}[]{\eta_X} &G(X) \arrow[]{d}[]{G(f)} \\
            F(Y) \arrow[]{r}[swap]{\eta_Y} &G(Y).
        \end{tikzcd}
    \end{center}
    La transformación natural $\eta:F\to G$ se suele denotar también como sigue
    \begin{center}
        \begin{tikzcd}
            &F \arrow[]{dd}[]{\eta} \\
            \mathscr{C} \arrow[bend left = 61]{rr}[]{} \arrow[bend right = 61]{rr}[]{} &&\mathscr{D}. \\
                        &G
        \end{tikzcd}
    \end{center}
    Denotaremos por $[\mathscr{C},\mathscr{D}]$ a la clase de todos los funtores\footnote{Recordamos que, al utilizar la palabra \emph{funtor}, suponemos que se trata de un funtor covariante, a menos que se especifique lo contrario explícitamente.} de $\mathscr{C}$ en $\mathscr{D}$. Para $F,G\in[\mathscr{C},\mathscr{D}]$, se denota por $\text{Nat}_{[\mathscr{C},\mathscr{D}]}(F,G)$ a la clase de las transformaciones naturales de $F$ en $G$.
\end{Def}

\begin{Obs}\label{Obs: Transformaciones naturales}

    Sean $\mathscr{C}$ y $\mathscr{D}$ categorías, $F,G,H\in[\mathscr{C},\mathscr{D}]$, $\eta\in\text{Nat}_{[\mathscr{C},\mathscr{D}]}(F,G)$ y $\rho\in\text{Nat}_{[\mathscr{C},\mathscr{D}]}(G,H)$.
    
    \begin{enumerate}[label=(\arabic*)]
    
        \item La familia de morfismos $1_F:F\to F$, dada por
            \[
                (1_F)_C := 1_{F(C)} \ \ \forall \ C\in\text{Obj}(\mathscr{C}),
            \] 
            es una transformación natural.

        \item Si definimos la composición $\rho\eta:F\to H$ como
            \[
                (\rho\eta)_C:=\rho_C\circ\eta_C \ \ \forall \ C\in\text{Obj}(\mathscr{C}),
            \] 
            entonces la composición de transformaciones naturales $\rho\eta$ es una transformación natural. Más aún, dado que por definición la composición de morfismos en una categoría es asociativa, se sigue que la composición de transformaciones naturales también lo es.

        \item Podemos considerar a la categoría de funtores de $\mathscr{C}$ en $\mathscr{D}$, denotada por $\mathscr{D}^\mathscr{C}$, donde $\text{Obj}(\mathscr{D}^\mathscr{C})$ es la clase de todos los funtores de $\mathscr{C}$ en $\mathscr{D}$, $\text{Mor}(\mathscr{D}^\mathscr{C})$ es la clase de todas las transformaciones naturales entre funtores de $\mathscr{C}$ en $\mathscr{D}$, la composición de transformaciones naturales se define como en (2) y las identidades, como en (1). En particular, del inciso (3) de la Observación \ref{Obs: Categoría opuesta} se sigue que $\mathscr{D}^{\mathscr{C}^\text{op}}$ es la categoría de funtores contravariantes de $\mathscr{C}$ en $\mathscr{D}$.

        \item Supongamos que $F(C)\xrightarrow[]{\eta_C} G(C)$ es un isomorfismo en $\mathscr{D}$, para todo $C\in\text{Obj}(\mathscr{C})$. Entonces la familia de morfismos $\eta^{-1}:G\to F$, dada por 
            \[
            (\eta^{-1})_C:=\eta_C^{-1} \ \ \forall \ C\in\text{Obj}(\mathscr{C}),
            \] 
            es una transformación natural. Más aún, $\eta^{-1}\eta = 1_F$ y $\eta\eta^{-1}=1_G$, por lo que $\eta:F\xrightarrow[]{\sim}G$ y $\eta^{-1}:G\xrightarrow[]{\sim}F$ son isomorfismos en $\mathscr{D}^\mathscr{C}$. Por ende, a las transformaciones naturales de este tipo se les conoce como \emph{isomorfismos naturales}.
    \end{enumerate}
\end{Obs}

\subsection*{Equivalencia de categorías} \label{Ssec: Equivalencia de categorías}

Dado que, en general, las categorías están compuestas de clases de objetos y de morfismos, la relación de isomorfismo entre categorías suele ser demasiado restrictiva, pues exige una correspondencia estricta entre las clases de objetos de ambas categorías. Los isomorfismos naturales nos ayudan a definir otra relación de equivalencia entre categorías menos restrictiva, que permite a ambas categorías tener un número arbitrario de ``copias isomorfas'' de un mismo objeto, llamada equivalencia de categorías. La idea intuitiva de esta noción es que, si dos categorías son equivalentes, una de ellas puede ser ``deformada'' en la otra añadiendo o removiendo la cantidad de ``copias isomorfas de objetos'' necesarias para hacer que las categorías sean isomorfas, en analogía con la equivalencia homotópica de espacios topológicos.

\begin{Def}\label{Def: Equivalencia de categorías}
    Sean $\mathscr{C}$ y $\mathscr{D}$ categorías. 

    \begin{enumerate}[label=(\alph*)]
    
        \item Un funtor $F:\mathscr{C}\to \mathscr{D}$ es una \emph{equivalencia de categorías} si existe un funtor $G:\mathscr{D}\to \mathscr{C}$ tal que $GF\simeq1_\mathscr{C}$ y $FG\simeq1_\mathscr{D}$. En tal caso, decimos que las categorías $\mathscr{C}$ y $\mathscr{D}$ son \emph{equivalentes}, y lo denotamos por $\mathscr{C}\cong\mathscr{D}$.

        \item Un funtor contravariante $F:\mathscr{C}\to \mathscr{D}$ es una \emph{dualidad de categorías} si existe un funtor contravariante $G:\mathscr{D}\to \mathscr{C}$ tal que $GF\simeq1_\mathscr{C}$ y $FG\simeq1_\mathscr{D}$. En tal caso, decimos que las categorías $\mathscr{C}$ y $\mathscr{D}$ son \emph{duales} una de la otra. 
    \end{enumerate}
\end{Def}

\begin{Ejem}
    Para cualquier categoría $\mathscr{C}$, el funtor de dualidad $D_\mathscr{C}:\mathscr{C}\to \mathscr{C}^\text{op}$ es una dualidad de categorías, con los isomorfismos naturales dados por las transformaciones naturales identidad. Por esta razón, la categoría opuesta $\mathscr{C}^\text{op}$ de $\mathscr{C}$ también se conoce como la \emph{categoría dual} de $\mathscr{C}$.
\end{Ejem}        

A continuación, veremos una caracterización útil de las equivalencias de categorías.

\begin{Prop}\label{Mendoza-Ejer.5}
    Sea $F: \mathscr{C}\to \mathscr{D}$ un funtor. Entonces, $F$ es una equivalencia de categorías si, y sólo si, $F$ es fiel, pleno y denso.
\end{Prop}

\begin{proof}
    $(\Rightarrow)$ Supongamos que $F$ es una equivalencia. Entonces, por definición, existe $G:\mathscr{D}\to \mathscr{C}$ tal que $FG\simeq1_\mathscr{D}$ y $GF\simeq1_\mathscr{C}$. Por ende, existen transformaciones naturales $\eta:GF\to 1_\mathscr{C}, \rho:FG\to 1_\mathscr{D}$ dadas por las familias de isomorfismos $\eta:=\{\eta_C:GF(C)\xrightarrow[]{\sim}1_\mathscr{C}(C)\}_{C\in\text{Obj}(\mathscr{C})}$ y $\rho=\{\rho_D:FG(D)\xrightarrow[]{\sim}1_\mathscr{D}(D)\}_{D\in\text{Obj}(\mathscr{D})}$ tales que para todo $C\xrightarrow[]{\alpha} C'\in\text{Mor}(\mathscr{C})$ y $D\xrightarrow[]{\beta} D'\in\text{Mor}(\mathscr{D})$ tenemos que $(1_\mathscr{C}(\alpha))\eta_C = \eta_{C'}(GF(\alpha))$ y $(1_\mathscr{D}(\beta))\rho_D = \rho_{D'}(FG(\beta))$. \\

    Sean $X,Y\in\text{Obj}(\mathscr{C})$ y $Z\in\text{Obj}(\mathscr{D})$. Supongamos que existen $f,g\in\text{Hom}_\mathscr{C}(X,Y)$ tales que $F(f)=F(g)$. Entonces, $GF(f)=GF(g)$. Observemos que
    \begin{align*}
        f\eta_X &= (1_\mathscr{C}(f))\eta_X \\
                &= \eta_Y(GF(f)) \\
                &= \eta_Y(GF(g)) \\
                &= (1_\mathscr{C}(g))\eta_X \\
                &= g\eta_X.
    \end{align*}
    Por ende, $f\eta_X\eta_X^{-1} = g\eta_X\eta_X^{-1}$, lo que implica que $f=g$. Por lo tanto, $F$ es fiel. Más aún, por la simetría de la definición de equivalencia se sigue que $G$ es fiel. \\

    Sea $h\in\text{Hom}_\mathscr{D}(F(X),F(Y))$. Entonces $G(h)\in\text{Hom}_\mathscr{C}(GF(X),GF(Y))$, por lo que $\gamma := \eta_Y (G(h))\eta_X^{-1}\in\text{Hom}_\mathscr{C}(X,Y)$. Observemos que $(1_\mathscr{C}(\gamma))\eta_X = \eta_Y(GF(\gamma))$ implica que $(1_\mathscr{C}(\gamma)) = \eta_Y(GF(\gamma))\eta_X^{-1}$. Por ende,
    \[
        \eta_Y(G(h))\eta_X^{-1} = \eta_Y(GF(\gamma))\eta_X^{-1}. 
    \] 
    Aplicando $\eta_Y^{-1}$ por la izquierda y $\eta_X$ por la derecha a la ecuación anterior, obtenemos
    \[
        G(h) = GF(\gamma).
    \] 
    Como $G$ es fiel, entonces $h=F(\gamma)$; es decir, existe $\gamma\in\text{Hom}_\mathscr{C}(X,Y)$ tal que $F(\gamma)=h$. Por ende, $F$ es pleno. \\

    Por último, observemos que $\rho_Z:FG(Z)\xrightarrow[]{\sim}1_\mathscr{D}(Z)$, por lo que $F(G(Z))\simeq Z$. Dado que $G(Z)\in\text{Obj}(\mathscr{C})$, se sigue que $F$ es denso. \\

    $(\Leftarrow)$ Supongamos que $F$ es fiel, pleno y denso. Sea $D\xrightarrow[]{\beta} D'\in\text{Mor}(\mathscr{D})$. Entonces, como $F$ es denso, existen $C,C'\in\mathscr{C}$ tales que $F(C)\simeq D$ y $F(C')\simeq D'$. Llamemos a estos isomorfismos $\varphi_C^D$ y $\varphi_{C'}^{D'}$, respectivamente. Entonces, $(\varphi_{C'}^{D'})^{-1}\beta\varphi_C^D\in\text{Hom}_\mathscr{D}(F(C),F(C'))$. Como $F$ es pleno, existe un morfismo $C\xrightarrow[]{\alpha} C'$ en $\mathscr{C}$ tal que $F(\alpha) = (\varphi_{C'}^{D'})^{-1}\beta\varphi_C^D$. Más aún, como $F$ es fiel, dicha $\alpha\in\text{Mor}(\mathscr{C})$ es única. Por lo tanto, existe una correspondencia biunívoca entre los morfismos de $\mathscr{C}$ y $\mathscr{D}$. \\

    Siguiendo la discusión anterior, sea $G:\mathscr{D}\to \mathscr{C}$ dado por $G(D\xrightarrow[]{\beta}D') = C\xrightarrow[]{\alpha}C'$, es decir, $G(D\xrightarrow[]{\beta}D')$ es el único morfismo en $\mathscr{C}$ tal que $F(G(\beta)) = (\varphi_{C'}^{D'})^{-1}\beta\varphi_C^D$. Veamos que $G$ es un funtor. En efecto, si $D'=D$ y $\beta=1_D$, entonces $(\varphi_{C'}^{D'})^{-1}\beta\varphi_C^D = (\varphi_C^D)^{-1} 1_D \varphi_C^D = (\varphi_C^D)^{-1}\varphi_C^D = 1_{F(C)}$ y, como $F1_C = 1_{F(C)}$ porque $F$ es un funtor, se sigue que $G(1_D) = 1_C = 1_{G(D)}$. \\

    Ahora, sean $D\xrightarrow[]{\beta_1} D', D'\xrightarrow[]{\beta_2} D''$ morfismos en $\mathscr{D}$. Entonces, existen $C\xrightarrow[]{\alpha_1} C', C'\xrightarrow[]{\alpha_2} C''$ en $\mathscr{C}$ tales que $G(D\xrightarrow[]{\beta_1}D') = C\xrightarrow[]{\alpha_1}C'$ y $G(D'\xrightarrow[]{\beta_2}D'') = C'\xrightarrow[]{\alpha_2}C''$, es decir, que cumplen las igualdades $F(\alpha_1) = (\varphi_{C'}^{D'})^{-1}\beta_1\varphi_C^D$ y $F(\alpha_2) = (\varphi_{C''}^{D''})^{-1}\beta_2\varphi_{C'}^{D'}$. Observemos que
    \begin{align*}
        (\varphi_{C''}^{D''})^{-1} \beta_2\beta_1 \varphi_C^D &= \big( (\varphi_{C''}^{D''})^{-1} \beta_2 \varphi_{C'}^{D'} \big) \big( (\varphi_{C'}^{D'})^{-1} \beta_1 \varphi_C^D\big) \\
                                                              &= F(\alpha_2)F(\alpha_1) \\
                                                              &= F(\alpha_2\alpha_1). \tag{pues $F$ es un funtor}
    \end{align*}
    Por lo tanto, $G(\beta_2\beta_1) = \alpha_2\alpha_1 = G(\beta_2)G(\beta_1)$, de donde concluimos que $G$ es un funtor. \\

    Notemos que, como $\varphi_C^D:F(C)\xrightarrow[]{\sim} D$ y $G(C)=D$, entonces $\varphi_C^D:FG(B)\xrightarrow[]{\sim}D$. Por lo tanto, definiendo $\rho_D:=\varphi_C^D$ para cada $D\in\text{Obj}(\mathscr{D})$, tenemos una familia de isomorfismos $\{\rho_D:FG(D)\xrightarrow[]{\sim}1_\mathscr{D}(D)\}_{D\in\text{Obj}(\mathscr{D})}$. Por construcción de $G$, para todo $D\xrightarrow[]{\beta} D'$ se tiene que $F(G(\beta)) = \rho_{D'}^{-1}\beta\rho_D$. Así, el siguiente diagrama conmuta
    \[\begin{tikzcd}
    FG(D) \arrow{r}{\rho_D} \arrow{d}[swap]{FG(\beta)} & D \arrow{d}{\beta} \\
    FG(D') \arrow{r}[swap]{\rho_{D'}}  & D'.
    \end{tikzcd}\]
    
    \noindent Entonces, $\rho$ es una transformación natural tal que $\rho_D$ es un isomorfismo para cada $D \in \text{Obj}(\mathscr{D})$. Por ende, tenemos que $\rho:FG\to 1_\mathscr{D}$ es un isomorfismo natural, por lo que $FG\simeq 1_\mathscr{D}$ en $\mathscr{D}^\mathscr{D}$. \\
    
    Ahora, como $F(C) \in \text{Obj}(\mathscr{D})$ para cada $C \in \text{Obj}(\mathscr{C})$, tenemos un isomorfismo $\rho_{F(C)}: FG(F(C))\xrightarrow[]{\sim} F(C)$ en $\mathscr{D}$. Como $F$ es un funtor fiel y pleno, existe un único $GF(C)\xrightarrow[]{\eta_C} C$ tal que $F(\eta_C)=\rho_{F(C)}.$ Definimos $\eta:= \{GF(C)\xrightarrow[]{\eta_C} C \}_{C \in \text{Obj}(\mathscr{C})}$. Veamos que $\eta : GF \to 1_{\mathscr{C}}$ es una transformación natural. Sea $C\xrightarrow[]{\alpha} C'$ un morfismo en $\mathscr{C}$. Entonces, $F(C)\xrightarrow[]{F(\alpha)} F(C')$ es un morfismo en $\mathscr{D}$ y por ser $\rho$ una transformación natural, se tiene el siguiente diagrama conmutativo
    \[\begin{tikzcd}
    FG(F(C)) \arrow{r}{\rho_{F(C)}} \arrow{d}[swap]{FG(F(\alpha))} & F(C) \arrow{d} {F(\alpha)} \\
    FG(F(C')) \arrow{r}[swap]{\rho_{F(C')}}                     & F(C'). 
    \end{tikzcd}\]
    Del diagrama anterior se sigue que $F(\alpha)\rho_{F(C)} = \rho_{F(C')} FG(F(\alpha))$. Por ende, tenemos que
\begin{align*}
    F( \alpha  \eta_C ) &= F(\alpha) F(\eta_C) \\
    &= F(\alpha) \rho_{F(C)} \\
    &=\rho_{F(C')} FG(F(\alpha)) \\
    &= F(\eta_{C'}) FG(F(\alpha))\\
    &= F(\eta_{C'} GF(\alpha)).
\end{align*}
    Luego, como $F$ es un funtor fiel, se sigue que $\alpha \eta_C = \eta_{C'} GF(\alpha).$ Por ende,  $\eta: GF\to 1_{\mathscr{C}}$ es una transformación natural. Más aún, como $\rho_{F(C)}$ es un isomorfismo, tenemos un morfismo $F(C)\xrightarrow[]{\rho_{F(C)}^{-1}} F(GF(C))$ y, por ser $F$ pleno, existe un morfismo $C\xrightarrow[]{\mu_{C}} GF(C)$ tal que $F(\mu_C)=\rho_{F(C)}^{-1}$. Notemos que 
\begin{align*}
    F(\eta_C  \mu_C) &= F(\eta_C) F(\mu_C) = \rho_{F(C)} \rho_{F(C)}^{-1} = 1_{F(C)} = F(1_C); \\
    F(\mu_C \eta_C) &= F(\mu_C) F(\eta_C) = \rho_{F(C)}^{-1} \rho_{F(C)} = 1_{FG(F(C))} = F(1_{GF(C)}). 
\end{align*}
Dado que $F$ es un funtor fiel, se tiene que $\eta_C \mu_C = 1_C$ y $\mu_C \eta_C = 1_{GF(C)}$, por lo que $\eta_C$ es un isomorfismo  para cada $C \in \text{Obj}(\mathscr{C})$. Por ende, $\eta$ es un isomorfismo natural, por lo que $GF \simeq 1_{\mathscr{C}}$ en $\mathscr{C}^\mathscr{C}$.
\end{proof}

% ¿Agregar observación de límites y colímites vistos como transformaciones naturales?

%\section{Localización}\label{Sec: Localización}
%
%Una localización de una categoría con respecto a una subclase de su clase de morfismos es una categoría en la que todos los morfismos de dicha subclase son isomorfismos, tal que cualquier categoría que cumpla esta misma propiedad se factorice a través de ella.
%
%\begin{Def}\label{Def: Localización de una categoría con respecto a una clase de morfismos}
%
%Sean $\mathscr{C}$ una categoría y $S\subseteq\text{Mor}(\mathscr{C})$ una clase de morfismos en $\mathscr{C}$. Una \emph{localización} de $\mathscr{C}$ \emph{con respecto a} $S$ es una categoría $\mathscr{L}$ junto con un funtor $Q:\mathscr{C}\to\mathscr{L}$ tal que $Q(s)$ es un isomorfismo para todo $s\in S$ y se cumple la siguiente propiedad universal: para cualquier categoría $\mathscr{D}$ y funtor $F:\mathscr{C}\to\mathscr{D}$ tal que $F(s)$ es un isomorfismo para todo $s\in S$, existe un único funtor $G:\mathscr{L}\to\mathscr{D}$ tal que $F=GQ$; diagramáticamente,
%\begin{center}
%    \begin{tikzcd}
%    \mathscr{C} \arrow{r}{F} \arrow{d}[swap]{Q} &\mathscr{D}. \\
%    \mathscr{L} \arrow[dotted]{ur}[swap]{\exists ! \ G}
%    \end{tikzcd}
%\end{center}
%
%\end{Def}
%
%\begin{Obs}\label{Observaciones de la localización}
%    Sean $\mathscr{C}$ una categoría, $S\subseteq\text{Mor}(\mathscr{C})$ y $\mathscr{L}, Q:\mathscr{C}\to \mathscr{L}$ una localización de $\mathscr{C}$ con respecto a $S$. Dado que, por definición, la propiedad universal de la localización es cumplida por el funtor $Q$, y la categoría $\mathscr{L}$ aparece implícitamente en la definición de éste, pues es su contradominio, podemos simplemente considerar al funtor $Q:\mathscr{C}\to \mathscr{L}$ como la localización de $\mathscr{C}$ con respecto a $S$.
%\end{Obs}
%
%\subsection*{Localización canónica} \label{Ssec: Localización canónica}
%
%Existe una manera canónica de definir la localización de una categoría con respecto a cualquier subclase de su clase de morfismos, como se muestra a continuación.
%
%\begin{Def}\label{Def: Camino}
%    Consideremos una categoría $\mathscr{C}$, una clase $S\subseteq\text{Mor}(\mathscr{C})$ de morfismos de $\mathscr{C}$ y un par de objetos $M,N$ de $\mathscr{C}$. Entonces, un \emph{camino} de longitud $n$ de $M$ a $N$ consiste de
%    \begin{enumerate}[label=(\alph*)]
%    
%        \item una función $L:\{0,1,...,n\}\to \text{Obj}(\mathscr{C})$ tal que $L_0=M$ y $L_n=N$, donde $L_i=L(i)$ para $0\le i\le n$;
%
%        \item una función $\Phi:\{(i,i+1) \mid 0\le i\le n-1\}\to \text{Mor}(\mathscr{C})$ tal que \[
%                \Phi(i,i+1) = L_i\xrightarrow[]{f_i}L_{i+1} \quad \lor \quad \Phi(i,i+1) = L_{i+1}\xleftarrow[]{s_i} L_i, \text{ con } s_i\in S.
%        \] 
%    \end{enumerate}
%    Por ende, formalmente, un camino es un par $(L,\Phi)$.
%\end{Def}
%
%\begin{Obs}\label{Observaciones de caminos}
%    Sean $\mathscr{C}$ una categoría y $S\subseteq\text{Mor}(\mathscr{C})$ una clase de morfismos de $\mathscr{C}$. Entonces, términos funtoriales, un camino de longitud finita es un diagrama finito que tiene a $\mathscr{C}$ como codominio y en el cual todas las flechas en alguno de los dos sentidos corresponden a morfismos en $S$.
%\end{Obs}
%
%\begin{Def}\label{Def: Transformaciones elementales de caminos}
%    Sean $\mathscr{C}$ una categoría, $S\subseteq\text{Mor}(\mathscr{C})$ una clase de morfismos de $\mathscr{C}$ y $M,N$ objetos de $\mathscr{C}$. Definimos las \emph{transformaciones elementales de caminos} como sigue:
%    \begin{align*}
%        \dotsb L_{i-1}\xrightarrow[]{f_{i-1}} L_i\xrightarrow[]{f_i} L_{i+1}\dotsb &\mapsto \dotsb L_{i-1}\xrightarrow[]{f_if_{i-1}}L_{i+1}\dotsb, \\ \\
%        \dotsb L\xrightarrow[]{s}P\xleftarrow[]{s}L\dotsb &\mapsto \dotsb L\xrightarrow[]{1_{L}}L\dotsb, \\ \\
%        \dotsb L\xleftarrow[]{s}P\xrightarrow[]{s}L\dotsb &\mapsto \dotsb L\xrightarrow[]{1_{L}}L\dotsb, \\ \\
%        \dotsb L\xrightarrow[]{1_{L}} L\xleftarrow[]{s} P\dotsb &\mapsto \dotsb L\xleftarrow[]{s} P\dotsb,
%    \end{align*}
%    donde $s\in S$. Dos caminos de $M$ a $N$ son \emph{equivalentes} si alguno de ellos puede ser obtenido a partir del otro mediante una cantidad finita de transformaciones elementales. Claramente, esto define una relación de equivalencia sobre el conjunto de caminos de $M$ a $N$.
%\end{Def}
%
%\begin{Def}\label{Def: Localización canónica}
%    Sean $\mathscr{C}$ una categoría y $S\subseteq\text{Mor}(\mathscr{C})$ una clase de morfismos de $\mathscr{C}$. Entonces definimos la categoría $\mathscr{C}[S^{-1}]$, que tiene como objetos a los objetos de $\mathscr{C}$ y, como morfismos, a las clases de equivalencia de caminos entre los objetos de $\mathscr{C}$. En partcular, el morfismo identidad de un objeto $M\in\text{Obj}(\mathscr{C}[S^{-1}])$ es $[M\xrightarrow[]{1_M}M]$.
%\end{Def}
%
%\begin{Teo}\label{Teo: Localización canónica}
%    Sean $\mathscr{C}$ una categoría y $S\subseteq\text{Mor}(\mathscr{C})$ una clase de morfismos de $\mathscr{C}$. Entonces, el funtor $Q:\mathscr{C}\to \mathscr{C}[S^{-1}]$ que manda a cada objeto en sí mismo y a cada morfismo $X\to Y$ en la clase de equivalencia del camino de longitud 1 correspondiente es una localización de $\mathscr{C}$ con respecto a $S$. Más aún, la categoría $\mathscr{C}[S^{-1}]$ es única hasta isomorfismos.
%\end{Teo}
%
%\begin{proof}
%
%    Sea $M\xrightarrow[]{s}N\in S$. Entonces, por definición, $Q(s) = [M\xrightarrow[]{s}N]$. Más aún, como $s\in S$, tenemos que $[N\xleftarrow[]{s}M]\in\mathscr{C}[S^{-1}]$. Observemos que
%    \begin{align*}
%        [M\xrightarrow[]{s}N][N\xleftarrow[]{s}M] &= [M\xrightarrow[]{s}N\xleftarrow[]{s}M] \\
%                                                  &= [M\xrightarrow[]{1_M}M], \\ \\
%        [N\xleftarrow[]{s}M][M\xrightarrow[]{s}N] &= [N\xleftarrow[]{s}M\xrightarrow[]{s}N] \\
%                                                  &= [N\xrightarrow[]{1_N}N].
%    \end{align*}
%    Por ende, $Q(s)$ es un isomorfismo para todo $s\in S$. \\
%
%    Supongamos que existen una categoría $\mathscr{D}$ y un funtor $F:\mathscr{C}\to \mathscr{D}$ tal que $F(s)$ es un isomorfsmo para todo $s\in S$. Consideremos a la correspondencia
%    \begin{align*}
%        G:\mathscr{C}[S^{-1}]&\to \mathscr{D}, \\
%        M&\mapsto M, \\
%        [(L,\Phi)]&\mapsto G(\Phi(n-1,n))\circ...\circ G(\Phi(2,1))\circ G(\Phi(1,0)),
%    \end{align*}
%    donde
%    \[
%        G(\Phi(i,i+1)) = \begin{cases} F(s_i)^{-1} &\text{si } \Phi(i,i+1) = L_i\xleftarrow[]{s_i} L_{i+1} \\ F(f_i) &\text{si } \Phi(i,i+1) = L_i\xrightarrow[]{f_i} L_{i+1}. \end{cases}
%    \] 
%    Es claro que $G:\mathscr{C}[S^{-1}]\to \mathscr{D}$ está bien definida, y es un funtor tal que $G\circ Q=F$. Más aún, por construcción, $G$ está determinado de forma única por $F$. \\
%
%    Supongamos que $Q':\mathscr{C}\to \mathscr{C}'$ es otra localización de $\mathscr{C}$ respecto a $S$. Entonces, por la propiedad universal de la localización, tenemos que existen funtores únicos $H:\mathscr{C}[S^{-1}]\to \mathscr{C}', H':\mathscr{C}'\to \mathscr{C}[S^{-1}]$ tales que el siguiente diagrama conmuta
%    \begin{equation}\label{eq: Unicidad de la localización}
%        \begin{tikzcd}
%            \mathscr{C} \arrow[]{r}[]{Q'} \arrow[]{dd}[swap]{Q} &\mathscr{C}'. \arrow[dotted, shift right]{ddl}[swap]{\exists! \ H'} \\ \\
%            \mathscr{C}[S^{-1}] \arrow[dotted, shift right]{uur}[swap]{\exists ! \ H}
%        \end{tikzcd}
%    \end{equation}
%    Ahora, del diagrama conmutativo (\ref{eq: Unicidad de la localización}), tenemos que los siguientes diagramas conmutan
%    \begin{center}
%        \begin{tikzcd}
%            \mathscr{C} \arrow[]{r}[]{Q} \arrow[]{dd}[swap]{Q} &\mathscr{C}[S^{-1}] \\ \\
%            \mathscr{C}[S^{-1}] \arrow[]{uur}[swap]{H'\circ H}
%        \end{tikzcd}
%        \hspace{2cm}
%        \begin{tikzcd}
%            \mathscr{C} \arrow[]{r}[]{Q} \arrow[]{dd}[swap]{Q} &\mathscr{C}'. \\ \\
%            \mathscr{C}' \arrow[]{uur}[swap]{H\circ H'}
%        \end{tikzcd}
%    \end{center}
%    Por otro lado, tenemos los diagramas conmutativos
%    \begin{center}
%        \begin{tikzcd}
%            \mathscr{C} \arrow[]{r}[]{Q} \arrow[]{dd}[swap]{Q} &\mathscr{C}[S^{-1}] \\ \\
%            \mathscr{C}[S^{-1}] \arrow[]{uur}[swap]{1_{\mathscr{C}[S^{-1}]}}
%        \end{tikzcd}
%        \hspace{2cm}
%        \begin{tikzcd}
%            \mathscr{C} \arrow[]{r}[]{Q} \arrow[]{dd}[swap]{Q} &\mathscr{C}'. \\ \\
%            \mathscr{C}' \arrow[]{uur}[swap]{1_{\mathscr{C}'}}
%        \end{tikzcd}
%    \end{center}
%    Por la propiedad universal de la localización, se sigue que $H'\circ H=1_{\mathscr{C}[S^{-1}]}$ y $H\circ H'=1_{\mathscr{C}'}$. Por lo tanto, la categoría $\mathscr{C}[S^{-1}]$ es única hasta isomorfismo.
%\end{proof}
%
%\begin{Obs}\label{Obs: Localización canónica}
%    Sean $\mathscr{C}$ una categoría y $S\subseteq\text{Mor}(\mathscr{C})$ una clase de morfismos de $\mathscr{C}$. En la demostración del Teorema \ref{Teo: Localización canónica}, hemos visto que, para todo $M\xrightarrow[]{s}N\in S$, se tiene que $[N\xleftarrow[]{s}M] = [M\xrightarrow[]{s}N]^{-1}$ en $\mathscr{C}[S^{-1}]$. Por ende, el camino de longitud uno $N\xleftarrow[]{s}M$ es un representante del inverso de la clase de equivalencia de $M\xrightarrow[]{s}N$, para cualquier $M\xrightarrow[]{s}N\in S$.
%\end{Obs}
%
%\subsection*{Categoría opuesta y localización} \label{Ssec: Categorías opuestas y localización}
%
%Sean $\mathscr{C}$ una categoría y $S$ una clase de morfismos en $\mathscr{C}$. Entonces, podemos considerar a la clase de morfismos $S^\text{op}:=D_\mathscr{C}(S)$ en $\mathscr{C}^\text{op}$. El siguiente resultado relaciona las localizaciones canónicas de $\mathscr{C}$ respecto a $S$ y $\mathscr{C}^\text{op}$ respecto a $S^\text{op}$.
%
%\begin{Prop}
%    Sean $S$ una clase de morfismos en una categoría $\mathscr{C}$, $S^\text{op}$ la correspondiente clase de morfismos en $\mathscr{C}^\text{op}$ y $Q:\mathscr{C}\to \mathscr{C}[S^{-1}], P:\mathscr{C}^\text{op}\to \mathscr{C}^\text{op}[(S^\text{op})^{-1}]$ las localizaciones canónicas respectivas. Entonces, las categorías $\mathscr{C}[S^{-1}]^\text{op}$ y $\mathscr{C}^\text{op}[(S^\text{op})^{-1}]$ son isomorfas.
%\end{Prop}
%
%\begin{proof}
%
%    Por el inciso (5) de la Observación \ref{Relaciones de dualidad} sabemos que la noción de isomorfismo es auto dual, de donde se sigue que el funtor $Q^\text{op}_\text{op}:\mathscr{C}^\text{op}\to \mathscr{C}[S^{-1}]^\text{op}$ es tal que $Q_\text{op}^\text{op}(s)$ es un isomorfismo para cada $s^\text{op}\in S^\text{op}$. Por ende, $Q_\text{op}^\text{op}$ se factoriza a través de la localización canónica $P:\mathscr{C}^\text{op}\to \mathscr{C}^\text{op}[(S^\text{op})^{-1}]$, por lo que existe un funtor único $G:\mathscr{C}^\text{op}[(S^\text{op})^{-1}]\to \mathscr{C}[S]^\text{op}$ tal que el siguiente diagrama conmuta
%\begin{center}
%    \begin{tikzcd}
%        \mathscr{C}^\text{op} \arrow[]{d}[swap]{P} \arrow[]{r}[]{Q_\text{op}^\text{op}} &\mathscr{C}[S^{-1}]^\text{op}. \\
%    \mathscr{C}^\text{op}[(S^\text{op})^{-1}] \arrow[dotted]{ur}[swap]{\exists! \ G}
%\end{tikzcd}
%\end{center}
%
%\noindent Como $P$ y $Q_\text{op}^\text{op}$ actúan como la identidad sobre objetos, entonces $G$ también lo hace. Ahora, sea
%\begin{equation}\label{eq: Camino 1}
%    \big[M\xrightarrow[]{f_0^\text{op}} L_1 \xrightarrow[]{f_1^\text{op}} \dotsb \xleftarrow[]{s_{i-1}^\text{op}} L_i\xrightarrow[]{f_i^\text{op}} \dotsb \xleftarrow[]{s_{n-2}^\text{op}} L_{n-1} \xleftarrow[]{s_{n-1}^\text{op}} N\big]
%\end{equation}
%un morfismo en $\mathscr{C}^\text{op}[(S^\text{op})^{-1}]$. Aplicando $Q_\text{op}^\text{op}$ al representante del morfismo anterior, obtenemos
%\begin{equation}\label{eq: Camino 2}
%    \bigg[ N\xleftarrow[]{s_{n-1}} L'_{1} \xleftarrow[]{s_{n-2}} \dotsb \xrightarrow[]{f_j} L'_j \xleftarrow[]{s_{j-1}} \dotsb \xrightarrow[]{f_1} L'_{n-1} \xrightarrow[]{f_0} M \bigg]^\text{op},
%\end{equation}
%donde $L'_k = L_{n-k}$ para $0\le k\le n$. Por ende, la imagen del morfismo (\ref{eq: Camino 1}) bajo $G$ está dada por (\ref{eq: Camino 2}), de donde se sigue el resultado\footnote{¿Elaborar más esta prueba?}.
%\end{proof}
%
%%\begin{Def}\label{Def: Localizacióñ de categorías}
%%    Sean $\mathscr{C}$ una categoría y $\Sigma$ una clase de morfismos en $\mathscr{C}$. Una \emph{localización} de $\mathscr{C}$ \emph{con respecto a} $\Sigma$ es una categoría $\mathscr{C}[\Sigma^{-1}]$, llamada \emph{categorías de fracciones}, y un funtor $Q=Q_\Sigma:\mathscr{C}\to \mathscr{C}[\Sigma^{-1}]$, llamado \emph{funtor de localización}, que satisface la siguiente propiedad universal.
%%    \begin{itemize}
%%    
%%        \item[(L1)] Para todo $\sigma\in\Sigma$, $Q(\sigma)$ es un isomorfismo.
%%
%%        \item[(L2)] Para todo funtor $F:\mathscr{C}\to \mathscr{D}$ tal que $F(\sigma)$ es un isomorfismo para todo $\sigma\in\Sigma$, existe un único funtor $\overline{F}:\mathscr{C}[\Sigma^{-1}]\to \mathscr{D}$ tal que $F = \overline{F}Q$; diagramáticamente,
%%            \begin{center}
%%                \begin{tikzcd}
%%                    \mathscr{C} \arrow[]{dr}[swap]{Q} \arrow[]{rr}[]{F} &&\mathscr{D} \\
%%                                                                        &\mathscr{C}[\Sigma^{-1}] \arrow[dotted]{ur}[swap]{\exists! \ \overline{F}}
%%                \end{tikzcd}
%%            \end{center}
%%            
%%    \end{itemize}
%%\end{Def}
%%
%%\begin{Teo}\label{Mendoza_CT-2.1}
%%    Para toda categoría
%%\end{Teo}
%
%\subsection*{Clases localizantes y cálculo de fracciones} \label{Ssec: Clases localizantes y cálculo de fracciones}
%
%\begin{Def}\label{Def: Fracciones}
%    Sean $\mathscr{C}$ una categoría, $S\subseteq\text{Mor}(\mathscr{C})$, y $(L,\Phi)$ un camino de longitud finita.
%    \begin{enumerate}[label=(\alph*)]
%    
%        \item Una \emph{fracción derecha} en $\mathscr{C}[S^{-1}]$ es una del tipo  $[f,s] := [M\xrightarrow[]{f}N\xleftarrow[]{s}L]$, con $f\in\text{Mor}(\mathscr{C})$ y $s\in S$.
%
%        \item Una \emph{fracción izquierda} en $\mathscr{C}[S^{-1}]$ es una del tipo  $[s',g] := [C\xleftarrow[]{s'}A\xrightarrow[]{g}B]$, con $s'\in S$ y $g\in\text{Mor}(\mathscr{C})$.
%    \end{enumerate}
%\end{Def}
%
%En general, una clase arbitraria $[(L,\Phi)]\in\mathscr{C}[S^{-1}]$ no tiene por qué ser una fracción derecha o izquierda. Para poder ``reducir'' clases arbitrarias de tal forma que podamos representarlas como fracciones derechas o izquierdas necesitaremos imponer condiciones adicionales en $S\subseteq\text{Mor}(\mathscr{C})$.
%
%\begin{Def}\label{Def: Sistema multiplicativo y cálculo de fracciones}
%    Sea $\mathscr{C}$ una categoría.
%
%    \begin{enumerate}[label=(\alph*)]
%    
%        \item Un \emph{sistema multiplicativo} es una clase de morfismos $S\subseteq\text{Mor}(\mathscr{C})$ tal que satisface las siguientes condiciones.
%
%            \begin{itemize}
%            
%                \item[(M1)] $1_X\in S$ para todo $X\in\text{Obj}(\mathscr{C})$.
%
%                \item[(M2)] Para cualesquier sucesión de morfismos componibles $X\xrightarrow[]{\alpha}Y\xrightarrow[]{\beta}Z$ en $\mathscr{C}$, tenemos que
%                    \[
%                    \alpha,\beta\in \implies \beta\alpha\in S.
%                    \] 
%            \end{itemize}
%
%        \item Una \emph{clase localizante a izquierda} en $\mathscr{C}$ es un sistema multiplicativo $S$ tal que satisface las siguientes condiciones.
%
%            \begin{itemize}
%            
%                \item[(CLI1)] Cada $X\xleftarrow[]{s}Z\xrightarrow[]{f}Y$ en $\mathscr{C}$, con $s\in S$, se puede completar al diagrama conmutativo en $\mathscr{C}$
%                    \begin{center}
%                        \begin{tikzcd}
%                            Z \arrow[]{d}[swap]{s} \arrow[]{r}[]{f} &Y \arrow[dotted]{d}[]{s'} \\
%                            X \arrow[dotted]{r}[swap]{g} &Z',
%                        \end{tikzcd}
%                        \hspace{1cm} con $s'\in S$.
%                    \end{center}
%                    
%                \item[(CLI2)] Sea \begin{tikzcd} X\arrow[shift left]{r}[]{f} \arrow[shift right]{r}[swap]{g} &Y\end{tikzcd} en $\mathscr{C}$. Entonces, si $X'\xrightarrow[]{s}X\in S$ es tal que $fs = gs$, existe $Y\xrightarrow[]{s'}Y'\in S$ tal que $s'f=s'g$.
%            \end{itemize}
%
%            En este caso, decimos que $S$ \emph{admite un cálculo de fracciones a izquierda}.
%
%        \item Una \emph{clase localizante a derecha} en $\mathscr{C}$ es un sistema multiplicativo $S$ tal que satisface las siguientes condiciones.
%
%            \begin{itemize}
%            
%                \item[(CLD1)] Cada $X\xrightarrow[]{g}Z\xleftarrow[]{s}Y$ en $\mathscr{C}$, con $s\in S$, se puede completar al diagrama conmutativo en $\mathscr{C}$
%                    \begin{center}
%                        \begin{tikzcd}
%                            Z \arrow[dotted]{d}[swap]{s'} \arrow[dotted]{r}[]{f} &Y \arrow[]{d}[]{s} \\
%                            X \arrow[]{r}[swap]{g} &Z',
%                        \end{tikzcd}
%                        \hspace{1cm} con $s'\in S$.
%                    \end{center}
%                    
%                \item[(CLD2)] Sea \begin{tikzcd} X\arrow[shift left]{r}[]{f} \arrow[shift right]{r}[swap]{g} &Y\end{tikzcd} en $\mathscr{C}$. Entonces, si $Y\xrightarrow[]{s}Y'\in S$ es tal que $sf = sg$, existe $X'\xrightarrow[]{s'}X\in S$ tal que $fs'=gs'$.
%            \end{itemize}
%
%            En este caso, decimos que $S$ \emph{admite un cálculo de fracciones a derecha}.
%
%        \item Una \emph{clase localizante} en $\mathscr{C}$ es un sistema multiplicativo $S$ tal que es una clase localizante a izquierda y a derecha en $\mathscr{C}$. En este caso, decimos que $S$ \emph{admite un cálculo de fracciones}.
%    \end{enumerate}
%\end{Def}
%
%\noindent Es claro de la definición anterior que las nociones de clase localizante a izquierda y clase localizante a derecha son duales y, por ende, que la noción de clase localizantes es auto dual.
%
%\begin{Obs}\label{Mendoza_CT-2.2}
%    Sean $\mathscr{C}$ una categoría y $S\subseteq\text{Mor}(\mathscr{C})$. 
%
%    \begin{enumerate}[label=(\arabic*)]
%    
%        \item Si $S$ es una clase localizante a izquierda, entonces toda clase de equivalencia en $\mathscr{C}[S^{-1}]$ es una fracción a izquierda.
%
%            En efecto: COMPLETAR.
%
%        \item Si $S$ es una clase localizante a derecha, entonces toda clase de equivalencia en $\mathscr{C}[S^{-1}]$ es una fracción a derecha.
%
%            En efecto: La demostración es dual a la de (1).
%    \end{enumerate}
%\end{Obs}
%
%\begin{Def}\label{Def: S-tejado a derecha}
%    
%    Sean $\mathscr{C}$ una categoría u $S\subseteq\text{Mor}(\mathscr{C})$.
%
%    \begin{enumerate}[label=(\alph*)]
%    
%        \item Sean $X,Y\in\text{Obj}(\mathscr{C})$. Un $S$-\emph{tejado a derecha} de $X$ a $Y$ es un par de morfismos $(s,f)$ como en el siguiente diagrama
%            \begin{center}
%                \begin{tikzcd}
%                    &X' \arrow[]{dl}[swap]{s} \arrow[]{dr}[]{f} \\
%                    X &&Y,
%                \end{tikzcd}
%            \end{center}
%            donde $s\in S$. Denotaremos por $[S,\mathscr{C}](X,Y)$ a la clase de todos los $S$-tejados a derecha de $X$ a $Y$.
%
%        \item Para $X,Y\in\text{Obj}(\mathscr{C})$, se define una relación $\sim$ en $[S,\mathscr{C}](X,Y)$ como sigue:
%            \[
%                (s,f) \sim (s',g) \iff \exists \ r,h\in\text{Mor}(\mathscr{C}) \text{ tales que } fr=gh \quad \land \quad s'h=sr\in S.
%            \] 
%            Diagramáticamente,
%            \begin{center}
%                \begin{tikzcd}
%                    &X' \arrow[]{dl}[swap]{s} \arrow[]{dr}[]{f} \\ 
%                    X &X''' \arrow[dotted]{u}[swap]{r} \arrow[dotted]{d}[]{h} &Y \\
%                      &X'' \arrow[]{ul}[]{s'} \arrow[]{ur}[swap]{g}
%                \end{tikzcd}
%                \hspace{1cm} o bien \hspace{1cm}
%                \begin{tikzcd}
%                    &&X''' \arrow[dotted]{dl}[swap]{r} \arrow[dotted]{dr}[]{h} \\
%                    &X' \arrow[]{dl}[swap]{s} \arrow[]{drrr}[]{f} &&X'' \arrow[]{dlll}[swap]{s'} \arrow[]{dr}[]{g} \\
%                    X &&&&Y.
%                \end{tikzcd}
%            \end{center}
%    \end{enumerate}
%\end{Def}
%
%\begin{Lema}\label{Mendoza_CT-2.2}
%    Sean $\mathscr{C}$ una categoría y $S\subseteq\text{Mor}(\mathscr{C})$ una clase localizante a derecha. Entonces, para cualesquiera $X,Y\in\mathscr{C}$, se tiene que $\sim$ es una relación de equivalencia\footnote{Algunas relaciones de equivalencia en otras partes del texto se dan por sentado. Pensar qué tan necesario es demostrar la de aquí.} en $[S,\mathscr{C}](X,Y)$.
%\end{Lema}
%
%\begin{proof}
%
%    Sean $X,Y\in\text{Obj}(\mathscr{C})$. \\
%
%    Sea $(s,f)$ un $S$-tejado a derecha de $X$ en $Y$. Entonces, tenemos que el siguiente diagrama en $\mathscr{C}$ conmuta
%            \begin{center}
%                \begin{tikzcd}
%                    &X' \arrow[]{dl}[swap]{s} \arrow[]{dr}[]{f} \\ 
%                    X &X' \arrow[dotted]{u}[swap]{1_{X'}} \arrow[dotted]{d}[]{1_{X'}} &Y, \\
%                      &X' \arrow[]{ul}[]{s} \arrow[]{ur}[swap]{f}
%                \end{tikzcd}
%            \end{center}
%            por lo que $(s,f)\sim(s,f)$. Por ende, la relación $\sim$ es reflexiva. \\
%
%            La simetría de la relación $\sim$ se sigue directamente de la definicióñ. \\
%
%            Por último, sean $(s,f)\sim(s',g)$ y $(s'g)\sim(s'',h)$. Entonces, tenemos los siguientes diagramas conmutativos en $\mathscr{C}$
%            \begin{equation}\label{Mendoza_CT-2.2-1}
%                \begin{tikzcd}
%                    &X_1 \arrow[]{dl}[swap]{s} \arrow[]{dr}[]{f} \\
%                    X &X' \arrow[]{u}[swap]{p} \arrow[]{d}[]{q} &Y, \\
%                      &X_2 \arrow[]{ul}[]{s'} \arrow[]{ur}[swap]{g}
%                \end{tikzcd}
%            \hspace{1cm}
%                \begin{tikzcd}
%                    &X_2 \arrow[]{dl}[swap]{s'} \arrow[]{dr}[]{g} \\
%                    X &X' \arrow[]{u}[swap]{r} \arrow[]{d}[]{v} &Y, \\
%                      &X_3 \arrow[]{ul}[]{s''} \arrow[]{ur}[swap]{h}
%                \end{tikzcd}
%            \end{equation}
%            donde $sp=s'q\in S$ y $s'r=s''v\in S$. Por (CLD1), tenemos el diagrama conmutativo en $\mathscr{C}$
%            \begin{equation}\label{Mendoza_CT-2.2-2}
%                \begin{tikzcd}
%                    X''' \arrow[dotted]{d}[swap]{s'''} \arrow[dotted]{r}[]{t} &X'' \arrow[]{d}[]{s'r} \arrow[]{r}[]{v} &X_3, \\
%                    X' \arrow[]{d}[swap]{q} \arrow[]{r}[swap]{sp} &X \\
%                    X_2
%                \end{tikzcd}
%            \end{equation}
%            Ahora, por el primer diagrama conmutativo en (\ref{Mendoza_CT-2.2-1}) y el cuadrado conmutativo en (\ref{Mendoza_CT-2.2-2}), tenemos que
%            \begin{align*}
%                s'(q s''') &= sps''' \\
%                           &= s'(rt).
%            \end{align*}
%            Luego, por (CLD2), existe $X^{\text{IV}}\xrightarrow[]{\sigma}X'''\in S$ tal que $qs'''\sigma = rt\sigma$. Consideremos el siguiente diagrama en $\mathscr{C}$:
%            \begin{center}
%                \begin{tikzcd}
%                    &X_1 \arrow[]{dl}[swap]{s} \arrow[]{dr}[]{f} \\
%                    X &X^{\text{IV}} \arrow[]{u}[swap]{ps'''\sigma} \arrow[]{d}[]{vt\sigma} &Y. \\
%                      &X_3 \arrow[]{ul}[]{s''} \arrow[]{ur}[swap]{h}
%                \end{tikzcd}
%            \end{center}
%            COMPLETAR.
%\end{proof}
%
%
%
%%\subsection*{Subcategorías y localización} \label{Ssec: Subcategorías y localización}

\begin{Prop}\label{Mendoza_CT-Ejer.8}
    
    Sean $\mathscr{C}$ una categoría localmente pequeña y $A,B\in\text{Obj}(\mathscr{C})$. Entonces, $h:A\simeq B$ en $\mathscr{C}$ si, y sólo si, $\text{Hom}_\mathscr{C}(-,h):\text{Hom}_\mathscr{C}(-,A)\simeq \text{Hom}_\mathscr{C}(-,B)$ en $\text{Sets}^\mathscr{C}$.
\end{Prop}

\begin{proof}\leavevmode

    $(\Rightarrow)$ Supongamos que $A\xrightarrow[]{h}B$ es un isomorfismo en $\mathscr{C}$. Dado que los funtores preservan isomorfismos, trivialmente se verifica que el siguiente diagrama en $\mathscr{C}$ conmuta para cualquier $X\xrightarrow[]{f}Y\in\text{Mor}(\mathscr{C})$
    \begin{center}
        \begin{tikzcd}
            \text{Hom}_\mathscr{C}(Y,A) \arrow[]{d}[swap]{\text{Hom}_\mathscr{C}(f,A)} \arrow[]{r}[]{\text{Hom}_\mathscr{C}(B,h)} &\text{Hom}_\mathscr{C}(Y,B) \arrow[]{d}[]{\text{Hom}_\mathscr{C}(f,B)} \\
            \text{Hom}_\mathscr{C}(X,A) \arrow[]{r}[swap]{\text{Hom}_\mathscr{C}(B,h)} &\text{Hom}_\mathscr{C}(X,B),
        \end{tikzcd}
    \end{center}
    por lo que $\text{Hom}_\mathscr{C}(-,H):\text{Hom}_\mathscr{C}(-,A)\to \text{Hom}_\mathscr{C}(-,B)$ es un isomorfismo natural. \\

    $(\Leftarrow)$ Supongamos que $\text{Hom}_\mathscr{C}(-,A)\xrightarrow[]{\text{Hom}_\mathscr{C}(-,h)}\text{Hom}_\mathscr{C}(-,B)$ es un isomorfismo natural en $\text{Sets}^\mathscr{C}$. Entonces, tenemos el siguiente diagrama conmutativo en Sets
    \begin{center}
        \begin{tikzcd}
            \text{Hom}_\mathscr{C}(B,A) \arrow[]{d}[swap]{\text{Hom}_\mathscr{C}(1_B,A)} \arrow[]{r}{\text{Hom}_\mathscr{C}(B,h)}[swap]{\sim} &\text{Hom}_\mathscr{C}(B,B) \arrow[]{d}{\text{Hom}_\mathscr{C}(1_B,B)} \\
            \text{Hom}_\mathscr{C}(B,A) \arrow[]{r}{\sim}[swap]{\text{Hom}_\mathscr{C}(B,h)} &\text{Hom}_\mathscr{C}(B,B).
        \end{tikzcd}
    \end{center}
    Dado que $\text{Hom}_\mathscr{C}(B,h)$ es un isomorfismo en Sets y, por tanto, una biyección, existe $h'\in\text{Hom}_\mathscr{C}(B,A)$ tal que $\text{Hom}_\mathscr{C}(B,h)(h') = hh' = 1_B$. Por el inciso (4) de la Observación \ref{Obs: Morfismos especiales}, tenemos que $h$ es un epimorfismo en $\mathscr{C}$, lo que implica que $\text{Hom}_\mathscr{C}(h,A):\text{Hom}_\mathscr{C}(B,A)\to \text{Hom}_\mathscr{C}(A,A)$ es un epimorfismo en Sets, por lo que existe $h''\in\text{Hom}_\mathscr{C}(B,A)$ tal que $\text{Hom}_\mathscr{C}(h,A)(h'') = h''h=1_A$. Luego, del inciso (1) de la Observación \ref{Obs: Morfismos especiales}, se sigue que $A\xrightarrow[]{h}B$ es un isomorfismo en $\mathscr{C}$.
\end{proof}


\subsection*{El Lema de Yoneda} \label{Ssec: Lema de Yoneda}

\begin{Teo}[Lema de Yoneda] \label{Teo: Lema de Yoneda}
    Sean $\mathscr{C}$ una categoría localmente pequeña y $F:\mathscr{C}^\text{op}\to \text{Sets}, G:\mathscr{C}\to \text{Sets}$ funtores. Entonces, para cualquier $A\in\text{Obj}(\mathscr{C})$, tenemos las biyecciones
    \begin{align*}
        \upsilon:\text{Nat}_{[\mathscr{C}^\text{op},\text{Sets}]}(\text{Hom}(-,A),F) &\rightarrow F(A), \\
        \tau&\mapsto \tau_A(1_A); \\ \\
        y:\text{Nat}_{[\mathscr{C},\text{Sets}]}(\text{Hom}(A,-),G) &\rightarrow G(A), \\
        \eta&\mapsto \eta_A(1_A);
    \end{align*}
    donde $\text{Hom}(-,-):\mathscr{C}^\text{op}\times\mathscr{C}\to \text{Sets}$ es el bifuntor Hom.
\end{Teo}

\begin{proof}

    %A continuación, probaremos que $\upsilon$ es una biyección. El caso de $y$ es totalmente análogo. \\

    Sea $\eta:\text{Hom}(-,A)\to F$ una transformación natural. Dado que $\eta_A:\text{Hom}(A,A)\to F(A)$ es un morfismo en Sets, tenemos que $\upsilon(\eta) = \eta_A(1_A)\in F(A)$, por lo que $\upsilon$ está bien definida. \\

    Por definición de transformación natural, para cada $B\in\text{Obj}(\mathscr{C}^\text{op})$ y $f^\text{op}\in\text{Hom}_{\mathscr{C}^\text{op}}(A,B)$ tenemos el diagrama conmutativo
    \begin{center}
        \begin{tikzcd}
            \text{Hom}(A,A) \arrow[]{d}[swap]{\text{Hom}(f^\text{op},A)} \arrow[]{r}[]{\eta_A} &F(A) \arrow[]{d}[]{F(f^\text{op})} \\
            \text{Hom}(B,A) \arrow[]{r}[swap]{\eta_B} &F(B)
        \end{tikzcd}
    \end{center}
    en Sets, de donde se sigue que
    \begin{align*}
        F(f^\text{op})\eta_A(1_A) &= \eta_B\text{Hom}(f^\text{op},A)(1_A) \\
                        &= \eta_B(f^\text{op}1_A) \\
                        &= \eta_Bf^\text{op},
    \end{align*}
    es decir,
    \begin{equation}\label{eq: Lema de Yoneda 1}
        \eta_B f^\text{op} = F(f^\text{op})\eta_A(1_A).
    \end{equation}
    Análogamente, si $\mu:\text{Hom}(-,A)\to F$ es otra transformación natural, tenemos que $\mu_B f^\text{op} = F(f^\text{op})\mu_A(1_A)$. Entonces, tenemos que
    \begin{align*}
        \upsilon(\eta) = \upsilon(\mu) &\implies \eta_A(1_A) = \mu_A(1_A) \\
                                          &\implies F(f^\text{op})\eta_A(1_A) = F(f^\text{op})\mu_A(1_A) \quad \hspace{0.2mm} \forall \ f^\text{op}\in\text{Hom}_{\mathscr{C}^\text{op}}(A,B) \\
                            &\implies \eta_Bf^\text{op} = \mu_Bf^\text{op} \quad \quad \quad \quad \quad \quad \quad \quad \forall \ f^\text{op}\in \text{Hom}_{\mathscr{C}^\text{op}}(A,B) \\
                            &\implies \eta_Bg = \mu_Bg \quad \quad \quad \quad \quad \quad \quad \quad \quad \quad \hspace{-0.05mm} \forall \ g\in \text{Hom}_{\mathscr{C}}(B,A) \\
                            &\implies \eta_B = \mu_B \hspace{24.8mm} \quad \quad \quad \quad \forall \ B\in\text{Obj}(\mathscr{C}) \\
                            &\implies \eta = \mu,
    \end{align*}
    por lo que $\upsilon$ es inyectiva. \\

    Ahora, sea $x\in F(A)$. Para todo $B\in\text{Obj}(\mathscr{C})$, definimos una función
    \[
        \eta_B:\text{Hom}(B,A)\to F(B), \ g \mapsto (F(g))(x),
    \]
    la cual es un morfismo en Sets. Veamos que $\eta := \{\text{Hom}(-,A)(B)\xrightarrow[]{\eta_B} F(B)\}_{B\in\text{Obj}(\mathscr{C})}$ es una transformación natural, es decir que, para cualquier morfismo $B\xrightarrow[]{h^\text{op}}C$ en $\mathscr{C}^\text{op}$, el diagrama en Sets
    \begin{equation}\label{eq: Lema de Yoneda 2}
        \begin{tikzcd}
            \text{Hom}(B,A) \arrow[]{d}[swap]{\text{Hom}(h^\text{op},A)} \arrow[]{r}[]{\eta_B} &F(B) \arrow[]{d}[]{F(h^\text{op})} \\
            \text{Hom}(C,A) \arrow[]{r}[swap]{\eta_C} &F(C)
        \end{tikzcd}
    \end{equation}
    conmuta. En efecto, sea $B\xrightarrow[]{h^\text{op}} C\in\text{Mor}(\mathscr{C}^\text{op})$. Entonces, para todo $f\in\text{Hom}(B,A)$, tenemos que
    \begin{align*}
        (F(h^\text{op})\eta_B)(f) &= F(h^\text{op})(\eta_B(f)) \\
                        &= F(h^\text{op})(F(f)(x))  \\
                        &= (F(h^\text{op})F(f))(x) \\
                        &= (F(h^\text{op}f))(x) \tag{$F$ es un funtor} \\
                        &= \eta_C(h^\text{op}f) \\
                        &= (\eta_C\text{Hom}(h^\text{op},A))(f),
    \end{align*}
    por lo que el diagrama (\ref{eq: Lema de Yoneda 2}) conmuta. Finalmente, dado que
    \begin{align*}
        \upsilon(\eta) &= \eta_A(1_A) \\
                &= (F(1_A))(x) \\
                &= 1_{F(A)}(x) \\
                &= x,
    \end{align*}
    se sigue que $\upsilon$ es suprayectiva. \\

    Por otro lado, sea $\tau:\text{Hom}(A,-)\to F$ una transformación natural. Dado que $\tau_A:\text{Hom}(A,A)\to F(A)$ es un morfismo en Sets, tenemos que $y(\tau) = \tau_A(1_A)\in F(A)$, por lo que $y$ está bien definida. \\

    Por definición de transformación natural, para cada $B\in\text{Obj}(\mathscr{C})$ y $f\in\text{Hom}(A,B)$ tenemos el diagrama conmutativo
    \begin{center}
        \begin{tikzcd}
            \text{Hom}(A,A) \arrow[]{d}[swap]{\text{Hom}(A,f)} \arrow[]{r}[]{\tau_A} &F(A) \arrow[]{d}[]{F(f)} \\
            \text{Hom}(A,B) \arrow[]{r}[swap]{\tau_B} &F(B)
        \end{tikzcd}
    \end{center}
    en Sets, de donde se sigue que
    \begin{align*}
        F(f)\tau_A(1_A) &= \tau_B\text{Hom}(A,f)(1_A) \\
                        &= \tau_B(f1_A) \\
                        &= \tau_Bf,
    \end{align*}
    es decir,
    \begin{equation}\label{eq: Lema de Yoneda 3}
        \tau_B f = F(f)\tau_A(1_A).
    \end{equation}
    Análogamente, si $\sigma:\text{Hom}(A,-)\to F$ es otra transformación natural, entonces $\sigma_B f = F(f)\sigma_A(1_A)$. Luego,
    \begin{align*}
        y(\tau) = y(\sigma) &\implies \tau_A(1_A) = \sigma_A(1_A) \\
                            &\implies F(f)\tau_A(1_A) = F(f)\sigma_A(1_A) \quad \quad \forall \ f\in \text{Hom}(A,B) \\
                            &\implies \tau_Bf = \sigma_Bf \quad \quad \quad \quad \quad \quad \hspace{10.4mm} \forall \ f\in \text{Hom}(A,B) \\
                            &\implies \tau_B = \sigma_B \hspace{21.6mm} \quad \quad \quad \quad \forall \ B\in\text{Obj}(\mathscr{C}) \\
                            &\implies \tau = \sigma.
    \end{align*}

    Ahora, sea $x\in F(A)$. Para todo $B\in\text{Obj}(\mathscr{C})$, definimos una función
    \[
        \tau_B:\text{Hom}(A,B)\to F(B), \ f \mapsto (F(f))(x),
    \]
    la cual es un morfismo en Sets. Veamos que $\tau := \{\text{Hom}(A,-)(B)\xrightarrow[]{\tau_B} F(B)\}_{B\in\text{Obj}(\mathscr{C})}$ es una transformación natural, es decir que, para cualquier morfismo $B\xrightarrow[]{g}C$ en $\mathscr{C}$, el diagrama en Sets
    \begin{equation}\label{eq: Lema de Yoneda 4}
        \begin{tikzcd}
            \text{Hom}(A,B) \arrow[]{d}[swap]{\text{Hom}(A,g)} \arrow[]{r}[]{\tau_B} &F(B) \arrow[]{d}[]{F(g)} \\
            \text{Hom}(A,C) \arrow[]{r}[swap]{\tau_C} &F(C),
        \end{tikzcd}
    \end{equation}
    conmuta. En efecto, sea $B\xrightarrow[]{g} C\in\text{Mor}(\mathscr{C})$. Entonces, para todo $f\in\text{Hom}(A,B)$, tenemos que
    \begin{align*}
        (F(g)\tau_B)(f) &= F(g)(\tau_B(f)) \\
                        &= F(g)(F(f)(x))  \\
                        &= (F(gf))(x)  \\
                        &= \tau_C(gf) \\
                        &= (\tau_C\text{Hom}(A,g))(f),
    \end{align*}
    por lo que el diagrama (\ref{eq: Lema de Yoneda 4}) conmuta. Finalmente, vemos que
    \begin{align*}
        y(\tau) &= \tau_A(1_A) \\
                &= 1_{F(A)}(x) \\
                &= x.
    \end{align*}
\end{proof}

\begin{Obs}\label{Obs: Lema de Yoneda}
    En la demostración del Teorema \ref{Teo: Lema de Yoneda} (Lema de Yoneda), la ecuación (\ref{eq: Lema de Yoneda 3}) nos dice que la acción de $\tau_B$ sobre un elemento $f\in\text{Hom}(A,B)$ arbitrario está totalmente determinada por el elemento $\tau_A(1_A)\in F(A)$. Conversamente, el elemento $\tau_A(1_A)\in F(A)$ genera una transformación natural de $\text{Hom}(-,A)$ a $F$.
\end{Obs}

%\section{Lema de Yoneda} \label{Sec: Lema de Yoneda}
%
%En la teoría de categorías, el Lema de Yoneda es un resultado engañosamente simple, hasta el punto de poder parecer una afirmación trivial, pero con implicaciones extremadamente profundas que se extienden hacia todos los confines de la teoría. Es por eso que, en vez de simplemente escribir el resultado y su demostración, en esta sección intentaremos hacer una breve presentación de una construcción previa que motiva el resultado, así como de algunas de las primeras consecuencias del Lema de Yoneda a nivel conceptual.
%
%
%\subsection*{La perspectiva conjuntista, la perspectiva categórica y la perspectiva de Yoneda}\label{Ssec: La perspectiva conjuntista, la perspectiva categórica y la perspectiva de Yoneda}
%
%Usualmente, al estudiar matemáticas utilizando el lenguaje de la teoría de conjuntos, adoptamos el paradigma de buscar entender un concepto matemático \emph{a través de los elementos de algún conjunto subyaciente a él}. Algunos ejemplos de esto incluyen el estudio de un grupo a través de cómo se relacionan los elementos de su conjunto subyaciente mediante las operaciones de dicha estructura algebraica, así como el estudio de un espacio topológico a través de cómo se relacionan los elementos de su topología mediante las operaciones de dicha estructura topológica. \\
%
%En cambio, la perspectiva que adoptamos al usar el lenguaje de la teoría de categorías es que, en vez de estudiar objetos desde \emph{adentro} de ellos, debemos estudiarlos desde \emph{afuera}. En particular, el paradigma que se adopta es el de entender un concepto matemático \emph{a través de sus relaciones con otros conceptos similares a él}. Para poder tomar esta perspectiva, los conceptos a tratar deben poder ser vistos como objetos en una misma categoría en el sentido de la Definición \ref{Def: Categoría} \textemdash lo que, por supuesto, limita el rango de aplicación de la teoría\textemdash \ y, en este caso, sus relaciones se entienden como los morfismos de dicha categoría. De esta manera, podemos estudiar un objeto $X$ en una categoría a través de, por ejemplo, los morfismos que tienen a $X$ como codominio; podemos pensar esto como obtener información sobre $X$ a través del ``punto de vista'' que tienen sobre él objetos en la misma categoría. \\
%
%La perspectiva de Yoneda va más alla y afirma que un objeto en una categoría está \emph{totalmente determinado} por sus relaciones con otros objetos en la misma categoría. De forma más precisa, Yoneda afirma que, en categorías localmente pequeñas, \emph{toda} la información que podemos obtener sobre un objeto $A$ (en términos categóricos) está dada por \emph{todos} los ``puntos de vista`` que tienen sobre él los objetos en la misma categoría. Más aún, también afirma que, en este tipo de categorías, \emph{toda} la información que podemos obtener sobre $A$ está dada por \emph{todos} los ``puntos de vista'' que tiene $A$ sobre los objetos en la misma categoría. A estos dos resultados se les conoce como el Encaje de Yoneda, en sus versiones contravariante y covariante, respectivamente, por razones que aclararemos a continuación. Más adeltante veremos que, concretamente, el Lema de Yoneda es una poderosa generalización de estos dos resultados.
%
%\subsection*{El funtor de Yoneda} \label{Ssec: El funtor de Yoneda}
%
%Sean $\mathscr{C}$ una categoría localmente pequeña y $A\in\text{Obj}(\mathscr{C})$ un objeto a estudiar. Los ``puntos de vista'' que tiene un objeto arbitrario $B\in\text{Obj}(\mathscr{C})$ sobre $A$ están dados por $\text{Hom}_\mathscr{C}(B,A)$ y, como $\mathscr{C}$ es localmente pequeña, forman un conjunto. Dado que $\text{Hom}_\mathscr{\mathscr{C}}(B,A) = \text{Hom}_\mathscr{C}(-,A)(B)$, podemos pensar a este conjunto de ``puntos de vista'' como la aplicación del funtor Hom-contravariante $\text{Hom}_\mathscr{C}(-,A)\in\text{Obj}\big(\text{Sets}^\mathscr{C}\big)$ al objeto arbitrario $B\in\text{Obj}(\mathscr{C})$. Observemos ahora el siguiente resultado.
%
%\begin{Lema}\label{Lema: Funtor de Yoneda covariante}
%    Sea $\mathscr{C}$ una categoría localmente pequeña. Entonces, la correspondencia
%    \begin{align*}
%        Y_\ast:\mathscr{C} &\to \text{Sets}^{\mathscr{C}}, \\
%        A &\mapsto \text{Hom}_\mathscr{C}(-,A), \\
%        \big( A\xrightarrow[]{h}B \big) &\mapsto \big( \text{Hom}_\mathscr{C}(-,A)\xrightarrow[]{\text{Hom}_\mathscr{C}(-,h)} \text{Hom}_\mathscr{C}(-,B) \big)
%    \end{align*}
%    es un funtor covariante.
%\end{Lema}
%
%\begin{proof}
%    (F1) se sigue de la definición del funtor Hom-contravariante, pues $\mathscr{C}$ es localmente pequeña, mientras que (F2), (F3) y (F4) se siguen de la definición del funtor Hom-covariante. 
%\end{proof}
%
%El funtor $Y_\ast$ definido en el Lema \ref{Lema: Funtor de Yoneda covariante} se conoce como el \emph{funtor de Yoneda}. Notamos que, a pesar de que este funtor es covariante, envía objetos de una categoría en el funtor Hom-\emph{contra}variante correspondiente a dicho objeto, ya sea un objeto o un morfismo. El siguiente resultado formaliza la perspectiva de Yoneda discutida en la sección anterior afirmando que \emph{dos objetos en una categoría localmente pequeña son isomorfos si, y sólo si, sus funtores Hom-contravariantes respectivos lo son} y, en particular, nos muestra que \emph{el isomorfismo entre dichos funtores se obtiene al aplicar el funtor de Yoneda al isomorfismo entre los objetos}.
%
%
%\begin{Prop}[Encaje de Yoneda] \label{Prop: Encaje de Yoneda}
%    El funtor de Yoneda $Y_\ast:\mathscr{C}\to \text{Sets}^\mathscr{C}$ es fiel y pleno.
%\end{Prop}
%
%\begin{proof}
%
%    Sean $A\xrightarrow[]{f}B, A\xrightarrow[]{g}B\in\text{Mor}(\mathscr{C})$. Observemos que
%    \begin{align*}
%        Y_\ast(f) = Y_\ast(g) &\implies \text{Hom}_\mathscr{C}(-,f) = \text{Hom}_\mathscr{C}(-,g) \\
%                              &\implies \text{Hom}_\mathscr{C}(A,f) = \text{Hom}_\mathscr{C}(A,g) \\
%                                                                  &\implies \text{Hom}_\mathscr{C}(A,f)(1_A) = \text{Hom}_\mathscr{C}(A,g)(1_A) \\
%                                                                  &\implies f1_A = g1_A \\
%                                                                  &\implies f=g,
%    \end{align*}
%    por lo que el funtor $Y_\ast$ es fiel. Por otro lado, sean $F,G\in\text{Obj}(\text{Sets}^\mathscr{C})$ y $F\xrightarrow[]{\eta} G\in\text{Mor}(\text{Sets}^\mathscr{C})$ una transformación natural. Entonces, por definición, para cada $A\xrightarrow[]{f}B\in\text{Mor}(\mathscr{C})$ tenemos que el diagrama en Sets
%    \begin{center}
%        \begin{tikzcd}
%            F(A) \arrow[]{d}[swap]{F(f)} \arrow[]{r}[]{\eta_A} &G(A) \arrow[]{d}[]{G(f)} \\
%            F(B) \arrow[]{r}[swap]{\eta_B} &G(B)
%        \end{tikzcd}
%    \end{center}
%    conmuta, de donde se sigue que COMPLETAR.
%    
%\end{proof}
%
%\subsection*{El funtor de Yoneda contravariante} \label{Ssec: El otro funtor de Yoneda contravariante}
%
%En la Proposición \ref{Mendoza_CT-Ejer.8} vimos que la información categórica de un objeto en una categoría localmente pequeña (hasta isomorfismo) está dada por los ``puntos de vista`` que tienen sobre él los objetos en la misma categoría (nuevamente, hasta isomorfismo). Anteriormente, mencionamos que lo mismo es válido para los ``puntos de vista'' que tiene dicho objeto sobre los objetos en la misma categoría, lo que podemos considerar como el funtor Hom-covariante correspondiente al objeto. Esto implicaría la existencia de una correspondencia
%\begin{align}\label{eq: Funtor de Yoneda contravariante}
%    Y^\ast:\mathscr{C}&\to \text{Sets}^\mathscr{C} \nonumber \\
%    A&\mapsto \text{Hom}_\mathscr{C}(A,-), \\
%    \big( A\xrightarrow[]{h}B \big) &\mapsto \big(\text{Hom}_\mathscr{C}(B,-)\xrightarrow[]{\text{Hom}_\mathscr{C}(h,-)} \text{Hom}_\mathscr{C}(A,-)\big), \nonumber
%\end{align}
%que invierte el sentido de los morfismos y tiene propiedades similares al funtor de Yoneda. En efecto, las demostraciones de los siguientes resultados son totalmente análogas a las del Lema \ref{Lema: Funtor de Yoneda covariante} y las Proposiciones \ref{Mendoza_CT-Ejer.8} y \ref{Prop: Encaje de Yoneda}\footnote{También es posible hacer las demostraciones por dualidad utilizando los tres resultados citados y el hecho de que, por definición de categoría opuesta, $\text{Hom}_{\mathscr{C}^\text{op}}(B,A) = \text{Hom}_\mathscr{C}(A,B)$ para cualesquiera $A,B\in\text{Obj}(\mathscr{C})$, lo que implica que $\text{Hom}_\mathscr{C}(A,-) = \text{Hom}_{\mathscr{C}^\text{op}}(-,A)$.}.
%
%\begin{Lema}\label{Lema: Funtor de Yoneda contravariante}
%    Sea $\mathscr{C}$ una categoría localmente pequeña. Entonces, la correspondencia (\ref{eq: Funtor de Yoneda contravariante}) es un funtor contravariante.
%\end{Lema}
%
%\begin{Prop}\label{Mendoza_CT-Ejer.8*}
%    Sean $\mathscr{C}$ una categoría localmente pequeña y $A,B\in\text{Obj}(\mathscr{C})$. Entonces, $h:A\simeq B$ en $\mathscr{C}$ si, y sólo si, $\text{Hom}_\mathscr{C}(h,-):\text{Hom}_\mathscr{C}(A,-)\simeq \text{Hom}_\mathscr{C}(B,-)$ en $\text{Sets}^\mathscr{C}$.
%\end{Prop}
%
%\begin{Prop}\label{Prop: Encaje de Yoneda contravariante}
%    Sea $\mathscr{C}$ una categoría localmente pequeña. Entonces, el funtor del Lema \ref{Mendoza_CT-Ejer.8*} es fiel y pleno.
%\end{Prop}
%
%El funtor $Y^\ast$ dado por la correspondencia (\ref{eq: Funtor de Yoneda contravariante}) se conoce como el \emph{funtor de Yoneda contravariante}. Notamos que, en este caso, a pesar de que este funtor es contravariante, envía objetos de una categoría en el funtor Hom-\emph{co}variante correspondiente a dicho objeto. Para hacer más clara la distinción, al funtor $Y_\ast$ del Lema \ref{Lema: Funtor de Yoneda covariante} también se le conoce como el \emph{funtor de Yoneda covariante}. 
%
%\subsection*{El Lema de Yoneda} \label{Ssec: El Lema de Yoneda}
%
%Una vez establecida la funtorialidad de $Y_\ast$ y $Y^\ast$ por los Lemas \ref{Lema: Funtor de Yoneda covariante} y \ref{Lema: Funtor de Yoneda contravariante}, respectivamente, los demás resultados que hemos visto en esta sección se siguen directamente como corolarios del siguiente resultado.
%
%\begin{Prop}\label{Prop: Lema de Yoneda para funtores Hom}
%    Sean $\mathscr{C}$ una categoría localmente pequeña. Entonces, para cualesquiera $A,B\in\text{Obj}(\mathscr{C})$, tenemos las biyecciones
%    \begin{align*}
%        \upsilon:\text{Nat}_{[\mathscr{C},\text{Sets}]}(\text{Hom}(-,A),\text{Hom}(-,B)) &\rightarrow \text{Hom}(A,B), \\
%        \tau&\mapsto \tau_A(1_A); \\ \\
%        y:\text{Nat}_{[\mathscr{C},\text{Sets}]}(\text{Hom}(A,-),\text{Hom}(B,-)) &\rightarrow \text{Hom}(B,A), \\
%        \eta&\mapsto \eta_A(1_A).
%    \end{align*}
%    %En particular,
%    %\begin{align*}
%    %    \tau = \text{Hom}(-,h) &\iff \tau_A(1_A) = h, \\
%    %    \eta = \text{Hom}(h',-) &\iff \eta_A(1_A) = h'.
%    %\end{align*}
%\end{Prop}
%
%\begin{proof}
%
%    Probaremos que $y$ es una biyección, pues el caso de $\upsilon$ es totalmente análogo. \\
%
%    Sea $\eta:\text{Hom}(A,-)\to \text{Hom}(B,-)$ una transformación natural. Dado que $\eta_A:\text{Hom}(A,A)\to \text{Hom}(B,A)$, tenemos que $y(\eta) = \eta_A(1_A)\in \text{Hom}(B,A)$, por lo que $y$ está bien definida. \\
%
%    Por definición de transformación natural, para cada $B\in\text{Obj}(\mathscr{C})$ y $f\in\text{Hom}(A,B)$ tenemos el diagrama conmutativo
%    \begin{center}
%        \begin{tikzcd}
%            \text{Hom}(A,A) \arrow[]{d}[swap]{\text{Hom}(A,f)} \arrow[]{r}[]{\eta_A} &\text{Hom}(B,A) \arrow[]{d}[]{\text{Hom}(B,f)} \\
%            \text{Hom}(A,B) \arrow[]{r}[swap]{\eta_B} &\text{Hom}(B,B)
%        \end{tikzcd}
%    \end{center}
%    en Sets, de donde se sigue que
%    \begin{align*}
%        \text{Hom}(B,f)\eta_A(1_A) &= \eta_B\text{Hom}(A,f)(1_A) \\
%                        &= \eta_B(f1_A) \\
%                        &= \eta_Bf,
%    \end{align*}
%    es decir,
%    \begin{equation}\label{eq: Lema de Yoneda para funtores Hom}
%        \eta_B f = \text{Hom}(B,f)\eta_A(1_A).
%    \end{equation}
%    Análogamente, si $\sigma:\text{Hom}(A,-)\to \text{Hom}(B,-)$ es otra transformación natural, entonces tenemos que $\sigma_B f = \text{Hom}(B,f)\sigma_A(1_A)$. Luego,
%    \begin{align*}
%        y(\eta) = y(\sigma) &\implies \eta_A(1_A) = \sigma_A(1_A) \\
%                            &\implies \eta_Bf = \sigma_Bf \quad \quad \quad \forall \ f\in \text{Hom}(A,B) \\
%                            &\implies \eta_B = \sigma_B \hspace{0.6mm} \quad \quad \quad \quad \forall \ B\in\text{Obj}(\mathscr{C}) \\
%                            &\implies \eta = \sigma,
%    \end{align*}
%    por lo que $y$ es inyectiva. \\
%
%    Por otro lado, sea $x\in \text{Hom}(A,B)$. Para todo $B\in\text{Obj}(\mathscr{C})$, definimos una función
%    \begin{align*}
%        \eta_B:\text{Hom}(A,B)&\to \text{Hom}(B,B), \\
%        f &\mapsto (\text{Hom}(B,f))(x),
%    \end{align*}
%    la cual es un morfismo en Sets. Veamos que $\eta := \{\eta_B:\text{Hom}(A,-)(B)\to \text{Hom}(B,-)(B)\}_{B\in\text{Obj}(\mathscr{C})}$ es una transformación natural, es decir, que para cualquier morfismo $B\xrightarrow[]{g}C$ en $\mathscr{C}$, el diagrama en Sets
%    \begin{equation}\label{eq: Lema de Yoneda para funtores Hom 2}
%        \begin{tikzcd}
%            \text{Hom}(A,B) \arrow[]{d}[swap]{\text{Hom}(A,g)} \arrow[]{r}[]{\eta_B} &\text{Hom}(B,B) \arrow[]{d}[]{\text{Hom}(B,g)} \\
%            \text{Hom}(A,C) \arrow[]{r}[swap]{\eta_C} &\text{Hom}(B,C)
%        \end{tikzcd}
%    \end{equation}
%    conmuta. En efecto, sea $B\xrightarrow[]{g} C\in\text{Mor}(\mathscr{C})$. Entonces, para todo $f\in\text{Hom}(A,B)$, tenemos que
%    \begin{align*}
%        (\text{Hom}(B,g)\eta_B)(f) &= \text{Hom}(B,g)(\eta_B(f)) \\
%                        &= \text{Hom}(B,g)(\text{Hom}(B,f)(x))  \\
%                        &= (\text{Hom}(B,gf))(x) \tag{$\text{Hom}(B,-)$ es un funtor} \\
%                        &= \eta_C(gf) \\
%                        &= (\eta_C\text{Hom}(A,g))(f),
%    \end{align*}
%    por lo que el diagrama (\ref{eq: Lema de Yoneda para funtores Hom 2}) conmuta. Finalmente, dado que
%    \begin{align*}
%        y(\eta) &= \eta_A(1_A) \\
%                &= (F(1_A))(x) \\
%                &= 1_{F(A)}(x) \\
%                &= x,
%    \end{align*}
%    se sigue que $y$ es suprayectiva.
%\end{proof}
%
%\begin{Obs}\label{Obs: Lema de Yoneda para funtores Hom}
%    El Lema \ref{Lema: Funtor de Yoneda covariante} y las Proposiciones \ref{Mendoza_CT-Ejer.8} y \ref{Prop: Encaje de Yoneda} se siguen directamente de la Proposición \ref{Prop: Lema de Yoneda para funtores Hom}. Dualmente, el Lema \ref{Lema: Funtor de Yoneda contravariante} y las Proposiciones \ref{Mendoza_CT-Ejer.8*} y \ref{Prop: Encaje de Yoneda contravariante} también se siguen de la Proposición \ref{Prop: Lema de Yoneda para funtores Hom}. \\
%
%    En efecto, 
%\end{Obs}
%
%A pesar de la evidente profundidad de la Proposición \ref{Prop: Lema de Yoneda para funtores Hom}, ésta es apenas un caso particular del Lema de Yoneda.
%
%\begin{Teo}[Lema de Yoneda] \label{Teo: Lema de Yoneda}
%    Sean $\mathscr{C}$ una categoría localmente pequeña y $F:\mathscr{C}\to \text{Sets}$ un funtor covariante. Entonces, para cualquier $A\in\text{Obj}(\mathscr{C})$, tenemos las biyecciones
%    \begin{align*}
%        \upsilon:\text{Nat}_{[\mathscr{C},\text{Sets}]}(\text{Hom}(-,A),F) &\rightarrow F(A), \\
%        \tau&\mapsto \tau_A(1_A); \\ \\
%        y:\text{Nat}_{[\mathscr{C},\text{Sets}]}(\text{Hom}(A,-),F) &\rightarrow F(A), \\
%        \eta&\mapsto \eta_A(1_A);
%    \end{align*}
%    donde $\text{Hom}(-,-):\mathscr{C}^\text{op}\times\mathscr{C}\to \text{Sets}$ es el bifuntor Hom-covariante.
%\end{Teo}
%
%\begin{proof}
%
%    La demostración de que $y$ es una biyección es idéntica a la de la Proposición \ref{Prop: Lema de Yoneda para funtores Hom}, reemplazando al funtor $\text{Hom}(B,-)$ por $F$ (y, en consecuencia, $\text{Hom}(B,A)$ por $F(A)$ y $\text{Hom}(B,f)$ por $F(f)$, etcétera). Nuevamente, el caso de $\upsilon$ es completamente análogo.
%
%%    Probaremos que $y$ es una biyección, pues el caso de $\upsilon$ es totalmente análogo. \\
%%
%%    Sea $\eta:\text{Hom}(A,-)\to F$ una transformación natural. Dado que $\eta_A:\text{Hom}(A,A)\to F(A)$, tenemos que $y(\eta) = \eta_A(1_A)\in F(A)$, por lo que $y$ está bien definida. \\
%%
%%    Por definición de transformación natural, para cada $B\in\text{Obj}(\mathscr{C})$ y $f\in\text{Hom}(A,B)$ tenemos el diagrama conmutativo
%%    \begin{center}
%%        \begin{tikzcd}
%%            \text{Hom}(A,A) \arrow[]{d}[swap]{\text{Hom}(A,f)} \arrow[]{r}[]{\eta_A} &F(A) \arrow[]{d}[]{F(f)} \\
%%            \text{Hom}(A,B) \arrow[]{r}[swap]{\eta_B} &F(B)
%%        \end{tikzcd}
%%    \end{center}
%%    en Sets, de donde se sigue que
%%    \begin{align*}
%%        F(f)\eta_A(1_A) &= \eta_B\text{Hom}(A,f)(1_A) \\
%%                        &= \eta_B(f1_A) \\
%%                        &= \eta_Bf,
%%    \end{align*}
%%    es decir,
%%    \begin{equation}\label{eq: Lema de Yoneda}
%%        \eta_B f = F(f)\eta_A(1_A).
%%    \end{equation}
%%    Análogamente, si $\sigma:\text{Hom}(A,-)\to F$ es otra transformación natural, entonces tenemos que $\sigma_B f = F(f)\sigma_A(1_A)$. Luego,
%%    \begin{align*}
%%        y(\eta) = y(\sigma) &\implies \eta_A(1_A) = \sigma_A(1_A) \\
%%                            &\implies \eta_Bf = \sigma_Bf \quad \quad \quad \forall \ f\in \text{Hom}(A,B) \\
%%                            &\implies \eta_B = \sigma_B \hspace{0.6mm} \quad \quad \quad \quad \forall \ B\in\text{Obj}(\mathscr{C}) \\
%%                            &\implies \eta = \sigma,
%%    \end{align*}
%%    por lo que $y$ es inyectiva. \\
%%
%%    Por otro lado, sea $x\in F(A)$. Para todo $B\in\text{Obj}(\mathscr{C})$, definimos una función
%%    \begin{align*}
%%        \eta_B:\text{Hom}(A,B)&\to F(B), \\
%%        f &\mapsto (F(f))(x),
%%    \end{align*}
%%    la cual es un morfismo en Sets. Veamos que $\eta := \{\eta_B:\text{Hom}(A,-)(B)\to F(B)\}_{B\in\text{Obj}(\mathscr{C})}$ es una transformación natural, es decir que, para cualquier morfismo $B\xrightarrow[]{g}C$ en $\mathscr{C}$, el diagrama en Sets
%%    \begin{equation}\label{eq: Lema de Yoneda 2}
%%        \begin{tikzcd}
%%            \text{Hom}(A,B) \arrow[]{d}[swap]{\text{Hom}(A,g)} \arrow[]{r}[]{\eta_B} &F(B) \arrow[]{d}[]{F(g)} \\
%%            \text{Hom}(A,C) \arrow[]{r}[swap]{\eta_C} &F(C),
%%        \end{tikzcd}
%%    \end{equation}
%%    conmuta. En efecto, sea $B\xrightarrow[]{g} C\in\text{Mor}(\mathscr{C})$. Entonces, para todo $f\in\text{Hom}(A,B)$, tenemos que
%%    \begin{align*}
%%        (F(g)\eta_B)(f) &= F(g)(\eta_B(f)) \\
%%                        &= F(g)(F(f)(x))  \\
%%                        &= (F(g)F(f))(x) \\
%%                        &= (F(gf))(x) \tag{$F$ es un funtor} \\
%%                        &= \eta_C(gf) \\
%%                        &= (\eta_C\text{Hom}(A,g))(f),
%%    \end{align*}
%%    por lo que el diagrama (\ref{eq: Lema de Yoneda 2}) conmuta. Finalmente, dado que
%%    \begin{align*}
%%        y(\eta) &= \eta_A(1_A) \\
%%                &= (F(1_A))(x) \\
%%                &= 1_{F(A)}(x) \\
%%                &= x,
%%    \end{align*}
%%    se sigue que $y$ es suprayectiva.
%\end{proof}
%
%\begin{Obs}\label{Obs: Lema de Yoneda}
%    %En la demostración del Teorema \ref{Teo: Lema de Yoneda} (Lema de Yonda), la ecuación (\ref{eq: Lema de Yoneda}) nos dice que la acción de $\eta_B$ sobre un elemento $f\in\text{Hom}(A,B)$ arbitrario está totalmente determinada por el elemento $\eta_A(1_A)\in F(A)$. Conversamente, el elemento $\eta_A(1_A)\in F(A)$ genera una transformación natural de $\text{Hom}(-,A)$ a $F$.
%    La versión general de la ecuación (\ref{eq: Lema de Yoneda para funtores Hom}), que aparece en la demostración del Teorema \ref{Teo: Lema de Yoneda} (Lema de Yonda) \textemdash nuevamente, reemplazando $\text{Hom}(f,B)$ por $F(f)$\textemdash, es
%    \[
%        \eta_B f = F(f)\eta_A(1_A).
%    \] 
%    Esta ecuación nos dice es que la acción de $\eta_B$ sobre un elemento $f\in\text{Hom}(A,B)$ arbitrario está totalmente determinada por el elemento $\eta_A(1_A)\in F(A)$ y, conversamente, que el elemento $\eta_A(1_A)\in F(A)$ genera una transformación natural de $\text{Hom}(-,A)$ a $F$.
%\end{Obs}

\end{document}
