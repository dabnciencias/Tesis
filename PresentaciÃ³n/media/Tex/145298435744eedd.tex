\documentclass[preview]{standalone}

\usepackage[english]{babel}
\usepackage[utf8]{inputenc}
\usepackage[T1]{fontenc}
\usepackage{lmodern}
\usepackage{amsmath}
\usepackage{amssymb}
\usepackage{dsfont}
\usepackage{setspace}
\usepackage{tipa}
\usepackage{relsize}
\usepackage{textcomp}
\usepackage{mathrsfs}
\usepackage{calligra}
\usepackage{wasysym}
\usepackage{ragged2e}
\usepackage{physics}
\usepackage{xcolor}
\usepackage{microtype}
\DisableLigatures{encoding = *, family = * }
\linespread{1}

\begin{document}

\begin{center}
\justifying \textbf{Definición} Un \emph{par de cotorsión} $(\mathcal{U}, \mathcal{V})$ en $\mathscr{B}$es un par $(\mathcal{U}, \mathcal{V})$ de subcategorías plenas de $\mathscr{B}$ que cumple las siguientes condiciones.\begin{enumerate} \item[(1)] $\text{Ext}^1_\mathscr{B} (\mathcal{U}, \mathcal{V}) = 0$. \item[(1)] $\text{Ext}^1_\mathscr{B} (U, X) = 0 \quad \forall \ U\in\mathcal{U} \Rightarrow X\in\mathcal{V}$. \item[(1)] $\text{Ext}^1_\mathscr{B} (Z, V) = 0 \quad \forall \ V\in\mathcal{V} \Rightarrow Z\in\mathcal{U}$. \end{enumerate}
\end{center}

\end{document}
