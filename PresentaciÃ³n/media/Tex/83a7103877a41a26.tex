\documentclass[preview]{standalone}

\usepackage[english]{babel}
\usepackage[utf8]{inputenc}
\usepackage[T1]{fontenc}
\usepackage{lmodern}
\usepackage{amsmath}
\usepackage{amssymb}
\usepackage{dsfont}
\usepackage{setspace}
\usepackage{tipa}
\usepackage{relsize}
\usepackage{textcomp}
\usepackage{mathrsfs}
\usepackage{calligra}
\usepackage{wasysym}
\usepackage{ragged2e}
\usepackage{physics}
\usepackage{xcolor}
\usepackage{microtype}
\DisableLigatures{encoding = *, family = * }
\linespread{1}

\begin{document}

\begin{center}
\justifying \textbf{Definición} Para cualesquiera $A,C\in\text{Obj}(\mathscr{A})$, una $\mathbb{E}$-\emph{extensión} es un elemento $\delta\in\mathbb{E}(C,A)$. Por ende, formalmente, una $\mathbb{E}$-extensión es una terna $(A,\delta,C)$. En particular, decimos que el elemento $0\in\mathbb{E}(C,A)$ es la $\mathbb{E}$-\emph{extensión escindible}.
\end{center}

\end{document}
