\documentclass[preview]{standalone}

\usepackage[english]{babel}
\usepackage[utf8]{inputenc}
\usepackage[T1]{fontenc}
\usepackage{lmodern}
\usepackage{amsmath}
\usepackage{amssymb}
\usepackage{dsfont}
\usepackage{setspace}
\usepackage{tipa}
\usepackage{relsize}
\usepackage{textcomp}
\usepackage{mathrsfs}
\usepackage{calligra}
\usepackage{wasysym}
\usepackage{ragged2e}
\usepackage{physics}
\usepackage{xcolor}
\usepackage{microtype}
\DisableLigatures{encoding = *, family = * }
\linespread{1}

\begin{document}

\begin{center}
\begin{itemize}
                        \item[$\bullet$] Existe una generalización simultánea de las categorías exactas y las categorías trianguladas, dada por las categorías extrianguladas.
                        \item[$\bullet$] Algunas categorías exactas pueden ser vistas como extrianguladas y vice versa. Por otro lado, toda categoría triangulada puede ser vista como extriangulada y algunas extrianguladas pueden ser vistas como trianguladas.
                        \item[$\bullet$] Existen categorías extrianguladas que no son exactas ni trianguladas.
                        \item[$\bullet$] Se pueden definir pares de cotorsión (completos) en este nuevo contexto.
                    \end{itemize}
\end{center}

\end{document}
