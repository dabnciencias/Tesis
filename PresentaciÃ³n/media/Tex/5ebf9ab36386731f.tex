\documentclass[preview]{standalone}

\usepackage[english]{babel}
\usepackage[utf8]{inputenc}
\usepackage[T1]{fontenc}
\usepackage{lmodern}
\usepackage{amsmath}
\usepackage{amssymb}
\usepackage{dsfont}
\usepackage{setspace}
\usepackage{tipa}
\usepackage{relsize}
\usepackage{textcomp}
\usepackage{mathrsfs}
\usepackage{calligra}
\usepackage{wasysym}
\usepackage{ragged2e}
\usepackage{physics}
\usepackage{xcolor}
\usepackage{microtype}
\DisableLigatures{encoding = *, family = * }
\linespread{1}

\begin{document}

\begin{center}
\textbf{Definición} Una \emph{realización} de $\mathbb{E}$ es una correspondencia $\mathfrak{s}$ que asocia a cada $\mathbb{E}$-extensión $\delta\in\mathbb{E}(C,A)$ una clase de equivalencia $\mathfrak{s}(\delta) = [A\xrightarrow{x}B\xrigharrow{y}C]$ tal que se cumple la siguiente condición.
\end{center}

\end{document}
