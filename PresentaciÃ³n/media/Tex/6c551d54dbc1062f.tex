\documentclass[preview]{standalone}

\usepackage[english]{babel}
\usepackage[utf8]{inputenc}
\usepackage[T1]{fontenc}
\usepackage{lmodern}
\usepackage{amsmath}
\usepackage{amssymb}
\usepackage{dsfont}
\usepackage{setspace}
\usepackage{tipa}
\usepackage{relsize}
\usepackage{textcomp}
\usepackage{mathrsfs}
\usepackage{calligra}
\usepackage{wasysym}
\usepackage{ragged2e}
\usepackage{physics}
\usepackage{xcolor}
\usepackage{microtype}
\DisableLigatures{encoding = *, family = * }
\linespread{1}

\begin{document}

\begin{center}
\justifying \textbf{Definición} Un \emph{par de cotorsión completo} $(\mathcal{U}, \mathcal{V})$ en $(\mathscr{A}, \mathscr{E})$ es un par $(\mathcal{U}, \mathcal{V})$ de subcategorías aditivas plenas de $\mathscr{A}$, cerradas por isomorfismos en $\mathscr{A}$, tal que cumple las siguientes condiciones.\begin{itemize} \item[(1)] $\text{Ext}^1_{(\mathscr{A}, T, \Delta)}(\mathcal{U}, \mathcal{V}) = 0$. \item[(2)] Para cualquier $C\in\mathscr{A}$, existe un triángulo distinguido \hfill \break $V\to U\to C\to T(V)$ tal que $U\in\mathcal{U}$ y $V\in\mathcal{V}$. \end{itemize}
\end{center}

\end{document}
