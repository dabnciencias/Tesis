\documentclass[preview]{standalone}

\usepackage[english]{babel}
\usepackage[utf8]{inputenc}
\usepackage[T1]{fontenc}
\usepackage{lmodern}
\usepackage{amsmath}
\usepackage{amssymb}
\usepackage{dsfont}
\usepackage{setspace}
\usepackage{tipa}
\usepackage{relsize}
\usepackage{textcomp}
\usepackage{mathrsfs}
\usepackage{calligra}
\usepackage{wasysym}
\usepackage{ragged2e}
\usepackage{physics}
\usepackage{xcolor}
\usepackage{microtype}
\DisableLigatures{encoding = *, family = * }
\linespread{1}

\begin{document}

\begin{center}
\flushleft \textbf{Proposición} Sean $\mathscr{C}$ una $\mathbb{Z}$-categoría y , $\mathbb{E}:\mathscr{C}^\text{op}\times\mathscr{C}\to\text{Ab}$ es un bifuntor aditivo. Entonces, las siguientes condiciones se satisfacen. \begin{enumerate} \item[(a)] $a\cdot(\delta_1+\delta_2) = a\cdot\delta_1 + a\cdot\delta_2 \ \ \ \forall \ a\in\text{Hom}_\mathscr{C}(A,A'), \ \delta_1,\delta_2\in\mathbb{E}(C,A)$. \item[(b)] $(a_1+a_1)\cdot\delta = a_1\cdot\delta + a_2\cdot\delta \ \ \ \forall \ a_1,a_2\in\text{Hom}_\mathscr{C}(A,A'), \ \delta\in\mathbb{E}(C,A)$. \item[(c)] $1_A\cdot\delta = \delta \ \ \ \forall \ \delta\in\mathbb{E}(C,A)$. \item[(d)] $(a'a)\cdot\delta = a'\cdot(a\cdot\delta) \ \ \ \forall \ a\in\text{Hom}_\mathscr{C}(A,A'), a'\in\text{Hom}_\mathscr{C}(A',A''), \delta\in\mathbb{E}(C,A)$. \item[(e)] $(\delta_1+\delta_2)\cdot c = \delta_1\cdot c + \delta_2\cdot c \ \ \ \forall \ c\in\text{Hom}_\mathscr{C}(C',C), \ \delta_1,\delta_2\in\mathbb{E}(C,A)$. \item[(f)] $\delta\cdot(c_1+c_2)=\delta\cdot c_1 + \delta\cdot c_2 \ \ \ \forall \ c_1,c_2\in\text{Hom}_\mathscr{C}(C',C), \ \delta\in\mathbb{E}(C,A)$. \item[(g)] $\delta\cdot1_C  = \delta \ \ \ \forall \ \delta\in\mathbb{E}(C,A)$. \item[(h)] $\delta\cdot(cc') = (\delta\cdot c)\cdot c' \ \ \ \forall \ c\in\text{Hom}_\mathscr{C}(C',C), c'\in\text{Hom}_\mathscr{C}(C'',C'), \delta\in\mathbb{E}(C,A)$. \item[(i)] $a\cdot(\delta\cdot c) = (a\cdot\delta)\cdot c \ \ \ \forall \ a\in\text{Hom}_\mathscr{C}(A,A'), c\in\text{Hom}_\mathscr{C}(C',C), \delta\in\mathbb{E}(C,A)$. \item[(j)] $0\cdot\delta = 0 = \delta\cdot0 \ \ \ \forall \ \delta\in\mathbb{E}(C,A)$. \end{enumerate}
\end{center}

\end{document}
