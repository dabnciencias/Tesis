\documentclass[preview]{standalone}

\usepackage[english]{babel}
\usepackage[utf8]{inputenc}
\usepackage[T1]{fontenc}
\usepackage{lmodern}
\usepackage{amsmath}
\usepackage{amssymb}
\usepackage{dsfont}
\usepackage{setspace}
\usepackage{tipa}
\usepackage{relsize}
\usepackage{textcomp}
\usepackage{mathrsfs}
\usepackage{calligra}
\usepackage{wasysym}
\usepackage{ragged2e}
\usepackage{physics}
\usepackage{xcolor}
\usepackage{microtype}
\DisableLigatures{encoding = *, family = * }
\linespread{1}

\begin{document}

\begin{center}
\justifying \textbf{Proposición} Sean $A\xrightarrow{x} B\xrightarrow{y} C\xrightdasharrow{\delta}$ un $\mathbb{E}$-triángulo. Entonces, para todo $X\in\mathscr{A}$, se tiene que las sucesiones $$\resizebox{\hsize}{!}{$\text{Hom}_\mathscr{A}(-,A) \xrightarrow{\text{Hom}_\mathscr{A}(-,x)} \text{Hom}_\mathscr{A}(-,B) \xrightarrow{\text{Hom}_\mathscr{A}(-,y)} \text{Hom}_\mathscr{A}(-,C) \xrightarrow{\delta_\sharp} \mathbb{E}(-,A) \xrightarrow{\mathbb{E}(-,x)} \mathbb{E}(-,B) \xrightarrow{\mathbb{E}(-,y)} \mathbb{E}(-,C)$,}$$ $$\resizebox{\hsize}{!}{$\text{Hom}_\mathscr{A}(C,-) \xrightarrow{\text{Hom}_\mathscr{A}(y,-)} \text{Hom}_\mathscr{A}(B,-) \xrightarrow{\text{Hom}_\mathscr{A}(x,-)} \text{Hom}_\mathscr{A}(A,-) \xrightarrow{\delta^\sharp} \mathbb{E}(C,-) \xrightarrow{\mathbb{E}(y^\text{op},-)} \mathbb{E}(B,-) \xrightarrow{\mathbb{E}(x^\text{op},-)} \mathbb{E}(A,-)$}$$
\end{center}

\end{document}
