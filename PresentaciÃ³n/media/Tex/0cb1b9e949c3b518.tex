\documentclass[preview]{standalone}

\usepackage[english]{babel}
\usepackage[utf8]{inputenc}
\usepackage[T1]{fontenc}
\usepackage{lmodern}
\usepackage{amsmath}
\usepackage{amssymb}
\usepackage{dsfont}
\usepackage{setspace}
\usepackage{tipa}
\usepackage{relsize}
\usepackage{textcomp}
\usepackage{mathrsfs}
\usepackage{calligra}
\usepackage{wasysym}
\usepackage{ragged2e}
\usepackage{physics}
\usepackage{xcolor}
\usepackage{microtype}
\DisableLigatures{encoding = *, family = * }
\linespread{1}

\begin{document}

\begin{center}
\flushleft \textbf{Observación} La categoría producto de dos $\mathbb{Z}$-categorías es una $\mathbb{Z}$-categoría. En particular, si $\mathscr{C}$ es una $\mathbb{Z}$-categoría, entonces $\mathscr{C}^\text{op}\times\mathscr{C}$ también lo es.
\end{center}

\end{document}
