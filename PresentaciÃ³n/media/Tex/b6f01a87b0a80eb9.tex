\documentclass[preview]{standalone}

\usepackage[english]{babel}
\usepackage[utf8]{inputenc}
\usepackage[T1]{fontenc}
\usepackage{lmodern}
\usepackage{amsmath}
\usepackage{amssymb}
\usepackage{dsfont}
\usepackage{setspace}
\usepackage{tipa}
\usepackage{relsize}
\usepackage{textcomp}
\usepackage{mathrsfs}
\usepackage{calligra}
\usepackage{wasysym}
\usepackage{ragged2e}
\usepackage{physics}
\usepackage{xcolor}
\usepackage{microtype}
\DisableLigatures{encoding = *, family = * }
\linespread{1}

\begin{document}

\begin{center}
\flushleft \textbf{Recordatorio} Sean $\mathscr{C}$ y $\mathscr{D} \mathbb{Z}$-categorías. Un funtor (covariante) F:\mathscr{C}\to\mathscr{D} es \emph{aditivo} si $F:\text{Hom}_\mathscr{C}(X,Y)\to \text{Hom}_\mathscr{D}(F(X), F(Y))$ es un morfismo de grupos abelianos para cualesquiera $X,Y\in\mathscr{C}$.
\end{center}

\end{document}
