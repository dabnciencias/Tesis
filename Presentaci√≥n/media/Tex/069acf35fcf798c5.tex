\documentclass[preview]{standalone}

\usepackage[english]{babel}
\usepackage[utf8]{inputenc}
\usepackage[T1]{fontenc}
\usepackage{lmodern}
\usepackage{amsmath}
\usepackage{amssymb}
\usepackage{dsfont}
\usepackage{setspace}
\usepackage{tipa}
\usepackage{relsize}
\usepackage{textcomp}
\usepackage{mathrsfs}
\usepackage{calligra}
\usepackage{wasysym}
\usepackage{ragged2e}
\usepackage{physics}
\usepackage{xcolor}
\usepackage{microtype}
\DisableLigatures{encoding = *, family = * }
\linespread{1}

\begin{document}

\begin{center}
\flushleft \textbf{Definición} Sean $\mathscr{C}$ una $\mathbb{Z}$-categoría, $\mathbb{E}:\mathscr{C}^\text{op}\times\mathscr{C}\to\text{Ab}$ es un bifuntoraditivo y $A,A',C,C'\in\mathscr{C}$. Definicmos las correspondencias \begin{align*} \text{Hom}_\mathscr{C}(A,A')\times\mathbb{E}(C,A) &\to \mathbb{E}(C,A'), \\ (a,\delta) &\mapsto a\cdot\delta:=\mathbb{E}(C,a)(\delta), \\ \\ \mathbb{E}(C,A)\times\text{Hom}_\mathscr{C}(C',C) &\to \mathbb{E}(C',A), \\ (\delta,c) &\mapsto \delta\cdot c:=\mathbb{E}(c^\text{op},A)(\delta). \end{align*}
\end{center}

\end{document}
