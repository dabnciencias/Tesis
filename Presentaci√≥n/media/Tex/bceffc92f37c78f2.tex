\documentclass[preview]{standalone}

\usepackage[english]{babel}
\usepackage[utf8]{inputenc}
\usepackage[T1]{fontenc}
\usepackage{lmodern}
\usepackage{amsmath}
\usepackage{amssymb}
\usepackage{dsfont}
\usepackage{setspace}
\usepackage{tipa}
\usepackage{relsize}
\usepackage{textcomp}
\usepackage{mathrsfs}
\usepackage{calligra}
\usepackage{wasysym}
\usepackage{ragged2e}
\usepackage{physics}
\usepackage{xcolor}
\usepackage{microtype}
\DisableLigatures{encoding = *, family = * }
\linespread{1}

\begin{document}

\begin{center}
\justifying \textbf{Nota} En todo lo que sigue, $\mathscr{A}$ representará una categoría aditiva y $\mathbb{E}$, un bifuntor aditivo de $\mathscr{A}^\text{op}\times\mathscr{A}$en Ab.
\end{center}

\end{document}
